% To compile, you must use latex. In TeXstudio, you must use dvi->* chain.
\documentclass[12pt, b5paper, oldfontcommands, draft]{memoir}
% Margins
\usepackage[b5paper,total={6in, 9in}]{geometry}
\geometry{margin=2.4cm, top=2.5cm}

% Citations at the end of each chapter
\usepackage[sectionbib,square,comma]{natbib}
\usepackage{chapterbib}
\bibliographystyle{bib/genbibstyle}

% Required for LaTeXDraw
\usepackage[usenames,dvipsnames]{pstricks}
\usepackage{pstricks-add}
\usepackage{epsfig}
\usepackage{pst-grad} % For gradients
\usepackage{pst-plot} % For axes
\usepackage[space]{grffile} % For spaces in paths
\usepackage{etoolbox} % For spaces in paths
\makeatletter % For spaces in paths
\patchcmd\Gread@eps{\@inputcheck#1 }{\@inputcheck"#1"\relax}{}{}
\makeatother

% Required for grammar trees
\usepackage{pst-node,pst-tree}

%\usepackage{createspace}
%\usepackage{geometry}
%\usepackage{graphicx}
%\usepackage{tikz}

%\usepackage[utf8]{inputenc}
%\usepackage{fix-cm}

\usepackage{tgheros} % "TEX Gyre Heros", Helvetica, code: qhv
%\usepackage{tgbonum} % "TEX Gyre Bonum", ITCBookman, code: qbk
% Keep commented. Somehow ptm still loads, while uncommenting messes up Greek letters.
%\usepackage{mathptmx} % Times, code: ptm

\usepackage[utf8]{inputenc}
\usepackage[T1]{fontenc}
\usepackage[english]{babel}
\usepackage{anyfontsize}
%\usepackage{microtype}
\usepackage{sfmath} % Sans Serif math fonts
\renewcommand{\rmdefault}{ptm} % Times
\renewcommand{\sfdefault}{qhv} % Helvetica
\renewcommand{\familydefault}{\rmdefault}

\usepackage[fleqn]{amsmath}
\usepackage{mathrsfs} % Math script
\usepackage{mathtools}
%\usepackage{textgreek} % Greek characters in text mode.
%\newcommand{\tl}{λ}
\newcommand{\tlb}[1]{$\boldsymbol{\lambda}$\,#1.\,}
\newcommand{\tl}{$\boldsymbol{\lambda}$}
\newcommand{\ta}{$\boldsymbol{\alpha}$}
\newcommand{\tb}{$\boldsymbol{\beta}$}
\newcommand{\te}{$\boldsymbol{\eta}$}
\newcommand{\td}{$\boldsymbol{\delta}$}
\newcommand{\tr}{$\boldsymbol{\rho}$}
\newcommand{\conversion}[1]{$\underset{\boldsymbol{#1}}{\leftrightarrow}$}
\newcommand{\reduction}[1]{$\underset{\boldsymbol{#1}}{\rightarrow}$}
\newcommand{\antireduction}[1]{$\underset{\boldsymbol{#1}}{\leftarrow}$}
%\DeclareUnicodeCharacter{03BB}{\lt}
%\usepackage{newunicodechar}
%\newunicodechar{λ}{\lt}

% For source code highlighting.
%\usepackage[no-math]{fontspec}
%\usepackage{minted}
%\usemintedstyle{mathematica}

%\usepackage{tabu} % Tables
%\usepackage{subcaption} % Subfigures (side-by-side)
%\usepackage{wrapfig}


% Get rid of those hideous arrows
\usepackage{chemarrow}
\renewcommand{\rightarrow}{\rarrowfill{1.2em}}
\renewcommand{\to}{\rarrowfill{1.2em}}
\renewcommand{\leftarrow}{\larrowfill{1em}}
\renewcommand{\leftrightarrow}{\larrowfill{1em}\hspace{-0.75em}\rarrowfill{1em}}
\newcommand{\xdownarrow}[1]{%
    {\left\downarrow\vbox to #1{}\right.\kern-\nulldelimiterspace}
}

% enumerations
\usepackage{paralist}
%\setlist{noitemsep,topsep=5pt,parsep=0pt,partopsep=5pt}

% Bibliography management
%\usepackage[backref=true,style=numeric,sorting=none]{biblatex}
%\bibliography{references}
%\usepackage[sectionbib,authoryear]{natbib}
%\usepackage{chapterbib}
\usepackage[hidelinks]{hyperref} % links (after biblatex)

%\usepackage{fullpage} %Comment this out before submission.
%Only used for DRAFTS to include the time of compilation as a versioning system
\usepackage{datetime}

% Whitespace adjustments
\newcommand{\vs}{\vspace{.5\baselineskip}}
\newcommand{\pg}{\clearpage}

% We create a math mode for the Miranda code, because there is occasional need
% for mathematical operators.
%\usepackage{sansmath}
%\usepackage{mathastext}
%\DeclareMathVersion{sfmath}
%\DeclareSymbolFont{sfletters}{OT1}{qhv}{m}{up}
%\SetSymbolFont{letters}{sfmath}{OT1}{qhv}{m}{up}
%\DeclareSymbolFont{sfoperators}{OT1}{qhv}{m}{up}
%\SetSymbolFont{operators}{sfmath}{OT1}{qhv}{m}{up}
%\SetMathAlphabet\mathit{sfmath}{OT1}{qhv}{m}{sl}
%\SetMathAlphabet\mathrm{sfmath}{OT1}{qhv}{m}{n}

% This is the mlcoded environment, a math environment in upright Helvetica.
\newenvironment{mlcoded}{
	\sffamily\spaceskip=2\fontdimen2\font
	\list{}{\rightmargin20pt\leftmargin20pt}\item[] % This line is how the quote environment is defined.
%	\noindent\par\vspace{0.5\baselineskip}
%	\begin{quote}
}{
	\endlist
%	\end{quote}
%	\vspace{0.5\baselineskip}\noindent\\
}
\newenvironment{mlalign}{
	\begin{mlcoded}\setlength{\tabcolsep}{2pt}
%    \vspace{-.3cm}
	\begin{tabular}{rl}
}{
	\end{tabular}
%    \vspace{-.2cm}
	\end{mlcoded}
}

\usepackage{environ}

\NewEnviron{mlnumbered}{%
    \begin{equation}
    \text{\hspace{-0.9em}\ml{\BODY}}%
    \end{equation}%
}{%
%what goes here?
}

% Inline Miranda code: "this is \ml{f x} applied to...."
%\newcommand{\ml}[1]{{\sansmath$#1$}}
\newcommand{\ml}[1]{{\sffamily #1}}

% For Roman Numerals
%\makeatletter
%\newcommand*{\rom}[1]{\expandafter\@slowromancap\romannumeral #1@}
%\makeatother

%%%%% Title page %%%%%
\makeatletter
\newcommand*{\maketitlepg}{{
		\null
		\let\cleardoublepage\clearpage
		\thispagestyle{empty}
		\leavevmode
		\normalfont
		\flushleft
		%	\parindent=0pt
		\vspace*{\drop}
		{\noindent\HUGE\fontfamily{qhv}\selectfont\uppercase \@title}\par
		\rule{\linewidth}{3pt}

		\vspace{2cm}
		{\LARGE\fontfamily{qhv}\selectfont\@author}\par
		%	\rule{\unitlength}{1.6pt}
		{\noindent\fontfamily{qhv}\selectfont\members\par\null}
		\vfill
		{\small\date{\today}}
		%	\vspace*{\drop}
		\cleardoublepage
}}
\makeatother

% This is how to do this in memoir:
\setsecnumdepth{subsubsection}
%\setcounter{secnumdepth}{4}
%\setcounter{tocdepth}{4}

\makeindex

%%%%% Other headings %%%%%

% Headings
\usepackage{fancyhdr}
\pagestyle{fancy}

\renewcommand{\chaptermark}[1]{\markboth{#1}{}}
\renewcommand{\sectionmark}[1]{\markright{#1}}
%\renewcommand{\subsectionmark}[1]{\markright{#1}}
\fancyhf{}
\fancyhead[LE,RO]{\footnotesize \thepage}
\fancyhead[LO]{\footnotesize \em Section \thesection\; \rightmark}
%\fancyhead[LO]{\fancyplain{}\slshape{\rightmark}}
\fancyhead[RE]{\footnotesize \em\chaptername \;\thechapter\; \leftmark}
%\fancyfoot[CE,CO]{\leftmark}
%\fancyfoot[LE,RO]{\thepage}

% For spelling out numbers: \numberstringnum{4} or \Numberstringnum{4}
\usepackage{fmtcount}
% For underlining the chapter number
\usepackage{calc}

\usepackage{caption}
\captionsetup{labelsep=quad,font={sf}}

% Section Formatting
\usepackage{titlesec}


% Dedication
\newcommand{\dedication}{To Dorothy}
% Dedication formatting, very similar to chapter and part formatting below
\newcommand{\dedicationstyle}[1]{{
		\normalfont\fontsize{20}{20}\selectfont \sffamily
		\makebox[2pt][l]{%\hspace-1.7cm
					\rule[-6pt]{\widthof{ttttttt\dedication}}{3pt}

		}
		%	\upshape\Large\sffamily
		\hspace{\widthof{ttttt}}
		\vspace{-5pt}
		#1
}}


% Chapter title formatting
\titleformat{\chapter}[display]{\normalfont\fontsize{30}{20}\selectfont\itshape \sffamily}{
   \makebox[2pt][l]{\hspace{-1.7cm}
	\rule[-6pt]{\widthof{ttttt\Numberstringnum{\thechapter}}}{3pt}
	}\Numberstringnum{\thechapter}
}{.7cm}
{
%	{\vspace{-1.1\baselineskip}\hrule height 3pt width 1.25in \relax}\vspace{\baselineskip}
	\upshape\Large\sffamily\MakeUppercase
}[]
\titlespacing*{\chapter}{ 1.7cm}{ 1cm}{1cm}  %[ right sep ]
\newcommand{\chapterbreak}{\clearpage\thispagestyle{empty}}

% Part title formatting - identical to chapter formatting
\titleformat{\part}[display]{\normalfont\fontsize{30}{20}\selectfont\itshape \sffamily}{
	\makebox[2pt][l]{\hspace{-1.7cm}
		\rule[-6pt]{\widthof{tttttPart \thepart}}{3pt}
	}Part \thepart
}{.7cm}
{
	%	{\vspace{-1.1\baselineskip}\hrule height 3pt width 1.25in \relax}\vspace{\baselineskip}
	\upshape\LARGE\sffamily\MakeUppercase
}[]
\titlespacing*{\part}{ 1.7cm}{ 1cm}{3cm}  %[ right sep ]
\titleclass{\part}{top}
\newcommand{\partbreak}{\clearpage\thispagestyle{empty}}
% Section heading spacing
% \titleformat{<command>}[<shape>]{<format>}{<label>}{<sec>}{<before-code>}[<after-code>]
\titleformat{\section}{\bf\sffamily}{\thesection}{8pt}{}
\titleformat{\subsection}{\sffamily}{\thesubsection}{8pt}{}
%\titleformat{\subsubsection}{\large}{\textsf{\thesubsubsection}}{8pt}{\textsf}

% Removes page numbers on first part/chapter/title pages. Since it clobbers the plain page
% style, we can't use that style anywhere.
\makeatletter
\let\ps@plain\ps@empty
\makeatother

\title{{The Implementation\\ of Functional\\ \vspace{10pt}Programming Languages}}

%\subtitle{ }

\author{Simon L. Peyton Jones}

\newcommand{\members}{\leavevmode Department of Computer Science\\ University College London\par
	\vspace*{\baselineskip}
	\textit{with chapters by}\\
	Philip Wadler, Programming Research Group, Oxford\\
	Peter Hancock, Metier Management Systems Ltd\\
	David Turner, University of Kent, Canterbury
}

\newlength{\drop}
\drop=0.1\textheight


\newenvironment{numbered}{
	\vs
	\begin{compactenum}[(i)]
}{
	\end{compactenum}
	\vs
}

\newenvironment{references}{\begin{description}\itemsep -5pt \footnotesize
}{\end{description}}

% Boxed asides.
\usepackage{float}
\newcommand{\boxedfigure}[2]{
	\begin{figure}[H]
		\centering

		{%
			\setlength{\fboxrule}{1pt}%
			\setlength{\fboxsep}{10pt}%
			\fbox{%
				\begin{minipage}{0.9\textwidth}
					\small
					\setlength{\parindent}{10pt}
					\setlength{\parskip}{0mm plus 0mm minus 0mm}
					#1
			\end{minipage}%
			}%
		}%

		% \noindent\fbox{%
		% 	\parbox{0.9\textwidth}{%
		% 			#1
		% 	}%
		% }

		\caption{\textsf #2}
	\end{figure}
}


\newcommand{\plainbox}[1]{
    {\centering

        {%
            \vspace{0.5\baselineskip}
            \setlength{\fboxrule}{1pt}%
            \setlength{\fboxsep}{10pt}%
            \fbox{%
                \begin{minipage}{0.9\textwidth}
                    \small
                    \setlength{\parindent}{10pt}
                    \setlength{\parskip}{0mm plus 0mm minus 0mm}
                    #1
                \end{minipage}%
            }%
            \vspace{0.5\baselineskip}
        }%

    }
}


\newcommand{\titledbox}[3]{%
\plainbox{%
{\centering

#1

}
\vspace{0.4\baselineskip}

\noindent\textit{#2}
\vspace{0.4\baselineskip}

\noindent #3
}%
}%

\newcommand{\theorembox}[2]{\titledbox{THEOREM}{\noindent #1}{\noindent #2}}
\newcommand{\definitionbox}[2]{\titledbox{DEFINITION}{\noindent #1}{\noindent #2}}

\newenvironment{pcorollary}{\begin{quote}\textit{Corollary.\;}}{\end{quote}}
\newenvironment{pproof}{\begin{quote}\textit{Proof.\;}}{\end{quote}}
\newcommand{\eval}{\ml{\bfseries Eval}}

\def\bracketadjust{\hspace{-1.2pt}}
\newcommand{\evalbb}[1]{\ml{\mbox{\bfseries Eval[\bracketadjust[\ }}\ml{#1}\ml{\mbox{\bfseries\ ]\bracketadjust]}}}

\begin{document}

\frontmatter

\maketitlepg

% Make Dedication
\begin{flushleft}
	\thispagestyle{empty}
	\vspace*{\drop}
%	\huge
%	\dedication
	\dedicationstyle{\dedication}
\end{flushleft}
\clearpage

\tableofcontents
%\listoffigures
%\listoftables

\setlength{\parindent}{10pt}
\setlength{\parskip}{0mm plus 0mm minus 0mm}

\chapter[Preface]{}

% The Preface "chapter" is typeset as if the word preface were the number instead of the title,
% except it is in all caps as well.
\vspace{-3.5cm}
\makebox[2pt][l]{
{\normalfont\fontsize{20}{20}\selectfont\sffamily
\hspace{0.75cm}
\makebox[2pt][l]{ \hspace{-1.5cm}
	\rule[-6pt]{\widthof{mmPREFACE}}{3pt}
}PREFACE}
}

\vspace{1.5cm}

\noindent This book is about implementing functional programming languages using
graph reduction.

Functional languages have become the focus of much active research in
recent years
%\citep{backus_can_1978} \citep{peyton-jones_directions_1984}
[Backus, 1978] [Peyton Jones, 1984], but their acceptance has
been delayed by the inefficiency of their available implementations when
compared with more conventional languages.

This situation has changed recently, with the advent of rather fast implementations of functional languages such as Cardelli's ML [Cardelli, 1983],
Fairbairn's Ponder [Fairbairn, 1982], and the Chalmers Lazy ML compiler
[Johnsson, 1984]. These implementations rival the speed of compilers for
more conventional languages.

There appear to be two main approaches to the efficient implementation of
functional languages. The first is an environment-based scheme, exemplified
by Cardelli's ML implementation, which derives from the experience of the
Lisp community. The other is graph reduction, a much newer technique first
invented by Wadsworth [Wadsworth, 1971], and on which the Ponder and
Lazy ML implementations are founded. Despite the radical differences in
beginnings, the most sophisticated examples of each approach show remark-
able similarities.

The techniques of graph reduction are to be found scattered amongst the
proceedings of various conferences and workshops, and it is one purpose of
this book to collect some of this work together. The book is intended to have
two main applications:
\begin{compactenum}[(i)]
\item As a course text for part of an undergraduate or postgraduate course on
the implementation of functional languages.
\item As a handbook for those attempting to write a functional language
implementation based on graph reduction.
\end{compactenum}
The material is presented in a fairly informal tutorial fashion, the idea being to
convey the framework and some of the intuitions that will render the original
sources more accessible.

Chapters 5 and 7 were written by Philip Wadler, of the Programming
Research Group, Oxford, and Chapter 4 was written in collaboration with
him. Chapters 8 and 9 were written by Peter Hancock, of Metier Management
Systems Ltd, and currently at the Programming Research Group, Oxford. I
gratefully acknowledge their patience in writing and rewriting their drafts.

I am extremely grateful to a number of other people who have made
significant technical contributions to the book. David Turner's help was
invaluable, in offering comments on the parts of the book that relate to
Miranda. The Appendix which gives an introduction to Miranda, was written
by him. (Miranda is a trademark of Research Software Limited.)

Much of the information about Miranda in the first part of the book is based
on a prerelease version of the Miranda system, and I am grateful to Research
Software Limited for the permission to include this material.

Simon Finn, of the University of Stirling, made a number of penetrating
observations about the treatment of pattern-matching, which resulted in
significant improvements in Chapters 4 and 6.

Several long discussions with Thomas Johnsson of Chalmers University,
Goteborg, radically changed the shape of the G-machine chapters, and
Chapter 21 was written in collaboration with him. John Fairbairn and Stuart
Wray of Cambridge also made important contributions in this area.

Paul Hudak and Robert Keller helped me in writing the last section of
Chapter 24.

Many other people have given me welcome help and encouragement, and
have helped enormously by reading the text and making constructive
suggestions. These include Lennart Augustsson, Philip Boswell, Geoff Burn,
Nigel Chapman, Chris Clack, Pierre-Louis Curien, Dan Friedman, Kevin
Hammond, Peter Hancock, Chris Hankin, John Hughes, Thomas Johnsson,
Dick Kieburtz, Phil Molyneux, Benedict Nixon, Martin Packer, Ellen Poon,
Jon Salkild, William Stoye, David Turner, Phil Wadler, John Washbrook and
Russel Winder. I am particularly grateful to John Washbrook for his unfailing
support.

\vspace{-\baselineskip}
\begin{flushright}
Simon L. Peyton Jones\\
University College London\\
London WC1E 6BT\\
simonpj@uk.ac.ucl.cs
\end{flushright}

%\pg
\vspace{-2\baselineskip}

%\bibliography{bib/SPLJ}

\section*{References}

\begin{references}
\item Backus, J. 1978. Can programming be liberated from the von Neumann style? A
functional style and its algebra of programs. \textit{Communications of the ACM}. Vol.~21, no.~8, pp. 613-41.
\item Cardelli, L. 1983. The functional abstract machine. \textit{Polymorphism}. Vol.~1,~no.~1.
\item Fairbairn, J. 1982. Ponder and its type system. \textit{Technical Report 31}. Computer Lab.,
Cambridge. November.
\item Johnsson, T. 1984. Efficient compilation of lazy evaluation. In \textit{Proceedings of the ACM
Conference on Compiler Construction, Montreal}, pp. 58-69. June.
\item Peyton Jones, S.L. 1984. Directions in functional programming research. In SERC
\textit{Distributed Computing Systems}, pp. 220-49. Duce (editor). Peter Peregrinus.
\item Wadsworth, C.P. 1971. Semantics and pragmatics of the lambda calculus, Chapter 4.
PhD thesis, Oxford.

\end{references}


\mainmatter


\chapter{Introduction}

This book is about implementing functional programming languages using
\textit{lazy graph reduction}, and it divides into three parts.

The first part describes how to translate a high-level functional language
into an intermediate language, called the lambda calculus, including detailed
coverage of pattern-matching and type-checking. The second part begins with
a simple implementation of the lambda calculus, based on graph reduction,
and then develops a number of refinements and alternatives, such as super-
combinators, full laziness and SK combinators. Finally, the third part
describes the G-machine, a sophisticated implementation of graph reduction,
which provides a dramatic increase in performance over the implementations
described earlier.

One of the agreed advantages of functional languages is their semantic
simplicity. This simplicity has considerable payoffs in the book. Over and
over again we are able to make semi-formal arguments for the correctness of
the compilation algorithms, and the whole book has a distinctly mathematical
flavor - an unusual feature in a book about implementations.

Most of the material to be presented has appeared in the published
literature in some form (though some has not), but mainly in the form of
conference proceedings and isolated papers. References to this work appear
at the end of each chapter.

\section{Assumptions}
This book is about implementations, not languages, so we shall make no
attempt to extol the virtues of functional languages or the functional
programming style. Instead we shall assume that the reader is familiar with
functional programming; those without this familiarity may find it heavy
going. A brief introduction to functional programming may be found in
Darlington [1984], while Henderson [1980] and Glaser et al. [1984] give more
substantial treatments. Another useful text is Abelson and Sussman [1985]
which describes Scheme, an almost-functional dialect of Lisp.

An encouraging consensus seems to be emerging in the basic features of
high-level functional programming languages, exemplified by languages such
as SASL [Turner, 1976], ML [Gordon et al., 1979], KRC [Turner, 1982],
Hope [Burstall et al., 1980], Ponder [Fairbairn, 1985], LML [Augustsson,
1984], Miranda [Turner, 1985] and Orwell [Wadler, 1985]. However, for the
sake of definiteness, we use the language Miranda as a concrete example
throughout the book (When used as the name of a programming language,
`Miranda' is a trademark of Research Software Limited.) A brief intro-
duction to Miranda may be found in the appendix, but no serious attempt is
made to give a tutorial about functional programming in general, or Miranda
in particular. For those familiar with functional programming, however, no
difficulties should arise.

Generally speaking, all the material of the book should apply to the other
functional languages mentioned, with only syntactic changes. The only
exception to this is that we concern ourselves almost exclusively with the
implementation of languages with \textit{non-strict semantics} (such as SASL, KRC,
Ponder, LML, Miranda and Orwell). The advantages and disadvantages of
this are discussed in Chapter 11, but it seems that graph reduction is probably
less attractive than the environment-based approach for the implementation
of languages with strict semantics; hence the focus on non-strict languages.
However, some functional languages are strict (ML and Hope, for example),
and while much of the book is still relevant to strict languages, some of the
material would need to be interpreted with care.

Generally speaking, all the material of the book should apply to the other
functional languages mentioned, with only syntactic changes. The only
exception to this is that we concern ourselves almost exclusively with the
implementation of languages with \textit{non-strict semantics} (such as SASL, KRC,
Ponder, LML, Miranda and Orwell). The advantages and disadvantages of
this are discussed in Chapter 11, but it seems that graph reduction is probably
less attractive than the environment-based approach for the implementation
of languages with strict semantics; hence the focus on non-strict languages.
However, some functional languages are strict (ML and Hope, for example),
and while much of the book is still relevant to strict languages, some of the
material would need to be interpreted with care.

The emphasis throughout is on an informal approach, aimed at developing
understanding rather than at formal rigor. It would be an interesting task to
rewrite the book in a formal way, giving watertight proofs of correctness at
each stage.

\section{Part I: Compiling High-level Functional Languages}
It has been widely observed that most functional languages are quite similar to
each other, and differ more in their syntax than their semantics. In order to
simplify our thinking about implementations, the first part of this book shows
how to translate a high-level functional program into an \textit{intermediate language}
which has a very simple syntax and semantics. Then, in the second and third
parts of the book, we will show how to implement this intermediate language
using graph reduction. Proceeding in this way allows us to describe graph
reduction in considerable detail, but in a way that is not specific to any
particular high-level language.

The intermediate language into which we will translate the high-level
functional program is the notation of the lambda calculus (Figure~\ref{fig:implfunctprog}). The
lambda calculus is an extremely well-studied language, and we give an introduction
to it in Chapter 2.


\boxedfigure{
\centering

\framebox{\
    \begin{minipage}{6cm}
        \centering

        High-level language program\
    \end{minipage}
}\\
\hspace{4cm}$\Bigg\downarrow$ \quad
\begin{minipage}{4cm}
    {Part I}
\end{minipage}\\
\framebox{\
    \begin{minipage}{6cm}
        \centering

        Program expressed in lambda notation
    \end{minipage}
    \ }\\
\hspace{4cm}$\Bigg\downarrow$  \quad
\begin{minipage}{4cm}
    {Parts II and III}
\end{minipage}\\
\framebox{\
    \begin{minipage}{6cm}
        \centering

        Concrete implementation
    \end{minipage}
    \ }\\

}{Implementing a functional program}



The lambda calculus is not only simple, it is also sufficiently expressive to
allow us to translate any high-level functional language into it. However,
translating some high-level language constructs into the lambda notation is
less straightforward than it at first appears, and the rest of Part I is concerned
with this translation.

Part I is organized as follows. First of all, in Chapter 3, we define a language
which is a superset of the lambda calculus, which we call the enriched lambda
calculus. The extra constructs provided by the enriched lambda calculus are
specifically designed to allow a straightforward translation of a Miranda
program into an expression in the enriched lambda calculus, and Chapter 3
shows how to perform this translation for simple Miranda programs.

After a brief introduction to pattern-matching, Chapter 4 then extends the
translation algorithm to cover more complex Miranda programs, and gives a
formal semantics for pattern-matching. Subsequently, Chapter 7 rounds out
the picture, by showing how Miranda's ZF expressions can also be translated
in the same way. (Various advanced features of Miranda are not covered,
such as algebraic types with laws, abstract data types, and modules.)

Much of the rest of Part I concerns the transformation of enriched lambda
calculus expressions into the ordinary lambda calculus subset, a process which
is quite independent of Miranda. This language-independence was one of the
reasons for defining the enriched lambda calculus language in the first place.
Chapter 5 shows how expressions involving pattern-matching constructs may
be transformed to use case-expressions, with a considerable gain in efficiency.
Then Chapter 6 shows how all the constructs of the enriched lambda calculus,
including case-expressions, may be transformed into the ordinary lambda
calculus.

Part I concludes with Chapter 8 which discusses type-checking in general,
and Chapter 9 in which a type-checker is constructed in Miranda.

\section{Part II: Graph Reduction}

The rest of the book describes how the lambda calculus may be implemented
using a technique called \textit{graph reduction}. It is largely independent of the later
chapters in Part I, Chapters 2--4 being the essential prerequisites.

As a foretaste of things to come, we offer the following brief introduction to
graph reduction. Suppose that the function \ml{f} is defined (in Miranda) like this:
\begin{mlcoded}
f\, x = (x + 1) $*$ (x -- 1)
\end{mlcoded}
This definition specifies that \ml{f} is a function of a single argument \ml{x}, which
computes `\ml{(x + 1) $*$ (x -- 1)}'. Now suppose that we are required to evaluate
\begin{mlcoded}
f\, 4
\end{mlcoded}
that is, the function \ml{f} applied to \ml{4}. We can think of the program like this:
\begin{center}
    \begin{forest}
        [\ml{@}
            [\ml{f}]
            [\ml{4}]
        ]
    \end{forest}
\end{center}
where the \ml{@} stands for function application. Applying \ml{f} to \ml{4} gives
\begin{center}
    \begin{forest}
        [\ml{$*$}
            [$+$
                [\ml{4}]
                [\ml{1}]
            ]
            [$-$
                [\ml{4}]
                [\ml{1}]
            ]
        ]
    \end{forest}
\end{center}
(Note: in the main text we will use a slightly different representation for
applications of \ml{$*$}, \ml{$+$} and \ml{$-$}, but this fact is not significant here.) We may now
execute the addition and the subtraction (in either order), giving
\begin{center}
    \begin{forest}
    [\ml{$*$}
        [\ml{5}]
        [\ml{3}]
    ]
\end{forest}
\end{center}
Finally we can execute the multiplication, to give the result
\begin{center}
\ml{15}
\end{center}

From this simple example we can see that:
\begin{numbered}
\item \textit{Executing} a functional program consists of \textit{evaluating} an expression.
\item A functional program has a natural representation as a \textit{tree} (or, more
generally, a \textit{graph}).
\item Evaluation proceeds by means of a sequence of simple steps, called
\textit{reductions}. Each reduction performs a local transformation of the graph
(hence the term \textit{graph reduction}).
\item Reductions may safely take place in a variety of orders, or indeed in
parallel, since they cannot interfere with each other.
\item Evaluation is complete when there are no further reducible expressions.
\end{numbered}
Graph reduction gives an appealingly simple and elegant model for the
execution of a functional program, and one that is radically different from the
execution model of a conventional imperative language.

We begin in Chapter 10 by discussing the representation of a functional
program as a graph. The next two chapters form a pair which discusses first the
question of deciding which reduction to perform next (Chapter 11), and then
the act of performing the reduction (Chapter 12).

Chapters 13 and 14 introduce the powerful technique of \textit{supercombinators},
which is the key to the remainder of the book. This is followed in Chapter 15
with a discussion of \textit{full laziness}, an aspect of lazy evaluation; this chapter can
be omitted on first reading since later material does not depend on it.

Chapter 16 then presents \textit{SK combinators}, an alternative implementation
technique to supercombinators. Hence, this chapter can be understood
independently of Chapters 13--15. Thereafter, however, we concentrate on
supercombinator-based implementations.

Part II concludes with a chapter on \textit{garbage collection}.

\section{Part Ill: Advanced Graph Reduction}

It may seem at first that graph reduction is inherently less efficient than more
conventional execution models, at least for conventional von Neumann
machines. The bulk of Part III is devoted to an extended discussion of the
G-machine, which shows how graph reduction can be compiled to a form that
is suitable for \textit{direct execution} by ordinary sequential computers.

In view of the radical difference between graph reduction on the one hand,
and the linear sequence of instructions executed by conventional machines on
the other, this may seem a somewhat surprising achievement. This (fairly
recent) development is responsible for a dramatic improvement in the speed
of functional language implementations.

Chapters 18 and 19 introduce the main concepts of the G-machine, while
Chapters 20 and 21 are devoted entirely to optimizations of the approach.
The book concludes with three chapters that fill in some gaps, and offer
some pointers to the future.

Chapter 22 introduces \textit{strictness analysis}, a compile-time program analysis
method which has been the subject of much recent work, and which is crucial
to many of the optimizations of the G-machine.

Perhaps the major shortcoming of functional programming languages,
from the point of view of the programmer, is the difficulty of estimating the
space and time complexity of the program. This question is intimately bound
up with the implementation, and we discuss the matter in Chapter 23.
graph reduction.

Finally, the book concludes with a chapter on parallel implementations of
graph reduction.

\pg

\section*{References}

\begin{references}
	\item Abelson, H., and Sussman, G.J. 1985. \textit{Structure and Interpretation of Computer
	Programs}. MIT Press.
	\item  Augustsson, L. 1984. A compiler for lazy ML. \textit{Proceedings of the ACM Symposium on
	Lisp and Functional Programming, Austin}. August, pp. 218--27.
	\item Burstall, R.M., MacQueen, D.B., and Sanella, D.T. 1980. \textit{Hope: an experimental
	applicative language}. CSR-62--80. Department of Computer Science, University of
	Edinburgh. May.
	\item  Darlington, J. 1984. Functional programming. In \textit{Distributed Computing}. Duce
(Editor). Academic Press.
	\item  Fairbairn, J. 1985. Design and implementation of a simple typed language based on the
lambda calculus. PhD thesis, \textit{Technical Report 75}. University of Cambridge. May.
	\item  Glaser, H., Hankin, C., and Till, D. 1984. \textit{Principles of Functional Programming}.
Prentice-Hall.
	\item  Gordon, M.J., Milner, A.J., and Wadsworth, C.P. 1979. \textit{Edinburgh LCF}. LNCS 78.
Springer Verlag.
	\item  Henderson, P. 1980. \textit{Functional Programming}. Prentice-Hall.
	\item  Turner, D.A. 1976. \textit{The SASL language manual}. University of St Andrews. December.
	\item  Turner, D.A. 1982. Recursion equations as a programming language. In \textit{Functional
Programming and Its Applications}, Darlington et al. (editors), pp. 1-28. Cam-
bridge University Press.
	\item  Turner, D.A. 1985. Miranda -- a non-strict functional language with polymorphic
types. In \textit{Conference on Functional Programming Languages and Computer
Architecture, Nancy}, pp. 1-16. Jouannaud (editor), LNCS 201. Springer Verlag.
	\item  Wadler, P. 1985. \textit{Introduction to Orwell}. Programming Research Group, University of
Oxford.
\end{references}


\part[Compiling High-Level Functional Languages]{Compiling High-Level\\ Functional Languages}

\chapter{The Lambda Calculus}

This chapter introduces the lambda calculus, a simple language which will be
used throughout the rest of the book as a bridge between high-level functional
languages and their low-level implementations. The reasons for introducing
the lambda calculus as an intermediate language are:
\begin{numbered}
\item It is a simple language, with only a few, syntactic constructs, and simple
semantics. These properties make it a good basis for a discussion of
implementations, because an implementation of the lambda calculus only
has to support a few constructs, and the simple semantics allows us to
reason about the correctness of the implementation.
\item It is an expressive language, which is sufficiently powerful to express all
functional programs (and indeed, all computable functions). This means
that if we have an implementation of the lambda calculus, we can
implement any other functional language by translating it into the lambda
calculus.
\end{numbered}
In this chapter we focus on the syntax and semantics of the lambda calculus
itself, before turning our attention to high-level functional languages in the
next chapter.

\section{The Syntax of the Lambda Calculus}

Here is a simple expression in the lambda calculus:
\begin{mlcoded}
(+ 4 5)
\end{mlcoded}
All function applications in the lambda calculus are written in \textit{prefix} form, so,
for example, the function + precedes its arguments 4 and 5. A slightly more
complex example, showing the (quite conventional) use of brackets, is
\begin{mlcoded}
(+ ($*$ 5 6) (+ 8 3))
\end{mlcoded}
In both examples, the outermost brackets are redundant, but have been
added for clarity (see Section 2.1.2).

From the implementation viewpoint, a functional program should be
thought of as an \textit{expression}, which is `executed' by \textit{evaluating} it. Evaluation
proceeds by repeatedly selecting a \textit{reducible expression} (or \textit{redex}) and
reducing it. In our last example there are two redexes: ($*$ 5 6) and (+ 8 3).
The whole expression \ml{(+ ($*$ 5 6) ($*$ 8 3))} is not a redex, since a \ml{+} needs to be
applied to two \textit{numbers} before it is reducible. Arbitrarily choosing the first
redex for reduction, we write
\begin{mlcoded}
(+ ($*$ 5 6) ($*$ 8 3)) $\to$ (+ 30 ($*$ 8 3))
\end{mlcoded}
where the \ml{$\to$} is pronounced `reduces to'. Now there is only one redex, \ml{(* 8 3)},
which gives
\begin{mlcoded}
(+ 30 ($*$ 8 3)) $\to$ (+ 30 24)
\end{mlcoded}
This reduction creates a new redex, which we now reduce
\begin{mlcoded}
(+ 30 24) $\to$ 54
\end{mlcoded}
\indent When there are several redexes we have a choice of which one to reduce
first. This issue will be addressed later in this chapter.

\subsection{Function Application and Currying}
In the lambda calculus, function application is so important that it is denoted
by simple juxtaposition; thus we write
\begin{mlcoded}
f x
\end{mlcoded}
to denote `the function \ml{f} applied to the argument \ml{x}'. How should we express
the application of a function to several arguments? We could use a new
notation, like \ml{(f (x,y))}, but instead we use a simple and rather ingenious
alternative. To express `the sum of 3 and 4' we write
\begin{mlcoded}
	((+ 3) 4)
\end{mlcoded}
The expression \ml{(+ 3)} denotes the function that adds \ml{3} to its argument. Thus
the whole expression means `the function \ml{+} applied to the argument \ml{3}, the
result of which is a function applied to \ml{4}'. (In common with all functional
programming languages, the lambda calculus allows a function to return a
function as its result.)

This device allows us to think of all functions as having a \textit{single argument
only}. It was introduced by Schonfinkel [1924] and extensively used by Curry
[Curry and Feys, 1958]; as a result it is known as \textit{currying}.

\subsection{Use of Brackets}

In mathematics it is conventional to omit redundant brackets to avoid
cluttering up expressions. For example, we might omit brackets from the
expression
\begin{mlcoded}
	(ab) + ((2c)/d)
\end{mlcoded}
to give
\begin{mlcoded}
	ab + 2c/d
\end{mlcoded}
The second expression is easier to read than the first, but there is a danger that
it may be ambiguous. It is rendered unambiguous by establishing conventions
about the precedence of the various functions (for example, multiplication
binds more tightly than addition)..

Sometimes brackets cannot be omitted, as in the expression:
\begin{mlcoded}
	(b + c)/a
\end{mlcoded}
Similar conventions are useful when writing down expressions in the
lambda calculus. Consider the expression:
\begin{mlcoded}
	((+ 3) 2)
\end{mlcoded}
By establishing the convention that function application associates to the left,
we can write the expression more simply as:
\begin{mlcoded}
	(+ 3 2)
\end{mlcoded}
or even
\begin{mlcoded}
	+ 3 2
\end{mlcoded}

We performed some such abbreviations in the examples given earlier. As a
more complicated example, the expression:
\begin{mlcoded}
	((f ((+ 4) 3)) (g x))
\end{mlcoded}
is fully bracketed and unambiguous. Following our convention, we may omit
redundant brackets to make the expression easier to read, giving:
\begin{mlcoded}
	f (+ 4 3) (g x)
\end{mlcoded}
No further brackets can be omitted. Extra brackets may, of course, be
inserted freely without changing the meaning of the expression; for example
\begin{mlcoded}
	(f (+ 4 3) (g x))
\end{mlcoded}
is the same expression again.

\subsection{Built-in Functions and Constants}
In its purest form the lambda calculus does not have built-in functions such as
\ml{+}, but our intentions are practical and so we extend the pure lambda calculus
with a suitable collection of such built-in functions.

These include arithmetic functions (such as \ml{+}, \ml{--}, $*$, \ml{/}\,) and constants (\ml{0}, \ml{1},
\ldots), logical functions (such as \ml{AND}, \ml{OR}, \ml{NOT}) and constants (\ml{TRUE},
\ml{FALSE}), and character constants (\ml{`a'}, \ml{`b'}, \ldots). For example
\begin{mlalign}
	-- 5 4 &$\to$ 1\\
AND TRUE FALSE &$\to$ FALSE
\end{mlalign}
We also include a conditional function, \ml{IF}, whose behavior is described by the
reduction rules:
\begin{mlcoded}
IF TRUE E\textsubscript{t} E\textsubscript{f} $\to$ E\textsubscript{t}\\
IF FALSE E\textsubscript{t} E\textsubscript{f} $\to$ E\textsubscript{f}
\end{mlcoded}

We will initially introduce data constructors into the lambda calculus by
using the built-in functions \ml{CONS} (short for \ml{CONSTRUCT}\,), \ml{HEAD} and \ml{TAIL}
(which behave exactly like the Lisp functions \ml{CONS}, \ml{CAR} and \ml{CDR}\,). The
constructor \ml{CONS} builds a compound object which can be taken apart with
\ml{HEAD} and \ml{TAIL}. We may describe their operation by the following rules:
\begin{mlalign}
HEAD (CONS a b) &$\to$ a\\
TAIL (CONS a b) &$\to$ b
\end{mlalign}
We also include \ml{NIL}, the empty list, as a constant. The data constructors will
be discussed at greater length in Chapter 4.

The exact choice of built-in functions is, of course, somewhat arbitrary, and
further ones will be added as the need arises.

\subsection{Lambda Abstractions}

The only functions introduced so far have been the built-in functions (such as
\ml{+} and \ml{CONS}\,). However, the lambda calculus provides a construct, called a
lambda abstraction, to denote new (non-built-in) functions. A lambda
abstraction is a particular sort of expression which denotes a function. Here is
an example of a lambda abstraction:
\begin{mlcoded}
	(\tlb{x}+ x 1)
\end{mlcoded}
The \tl says `here comes a function', and is immediately followed by a variable,
\ml{x} in this case; then comes a . followed by the \textit{body} of the function, \ml{(+ x 1)} in
this case. The variable is called the \textit{formal parameter}, and we say that the \tl
binds it. You can think of it like this:
\begin{quote}\setlength{\tabcolsep}{3pt}
\begin{tabular}{llllll}
	\ml{(\tl}	&\ml{x}  &\ml{.}  &\ml{+}  &\ml{x} &\ml{1}\,)  \\
	\;\;$\uparrow$	&$\uparrow$  &$\uparrow$  &$\uparrow$  &$\uparrow$ &$\uparrow$ \\
	That function of &\ml{x}  & which  &adds  &\ml{x} to &\ml{1} 
\end{tabular}
\end{quote}
A lambda abstraction \textit{always} consists of all the four parts mentioned: the \tl,
the formal parameter, the \ml{.} and the body.

A lambda abstraction is rather similar to a function definition in a
conventional language, such as C:
\begin{mlcoded}
	Inc( x )\\
	int x;\\
	\{ return( x + 1 ); \}
\end{mlcoded}
The formal parameter of the lambda abstraction corresponds to the formal
parameter of the function, and the body of the abstraction is an expression
rather than a sequence of commands. However, functions in conventional
languages must have a name (such as \ml{Inc}), whereas lambda abstractions are
`anonymous' functions.

The body of a lambda abstraction extends \textit{as far to the right as possible}, so
that in the expression
\begin{mlcoded}
	(\tlb{x}+ x 1) 4
\end{mlcoded}
the body of the \ml{\tl{}x} abstraction is \ml{(+ x 1)}, not just \ml{+}. As usual, we may add
extra brackets to clarify, thus
\begin{mlcoded}
	(\tlb{x}(+ x 1)) 4
\end{mlcoded}
When a lambda abstraction appears in isolation we may write it without any
brackets:
\begin{mlcoded}
	\tlb{x}+ x 1
\end{mlcoded}

\subsection{Summary}
We define a \textit{lambda expression} to be an expression in the lambda calculus, and
Figure 2. 1 summarizes the forms which a lambda expression may take. Notice
that a lambda \textit{abstraction} is not the same as a lambda \textit{expression}; in fact the
former is a particular instance of the latter.

<exp> :: = <constant>
Built-in constants
<variable>
Variable names
<exp> <exp>
A <variable>.<exp>
Applications
Lambda abstractions
This is the abstract syntax of lambda expressions. In order to write down
such an expression in concrete form we use brackets to disambiguate its
structure (see Section 2.1.2).
We will use lower-case letters for variables (e.g. x, f), and upper-case
letters to stand for whole lambda expressions (e.g. M, E).
The choice of constants is rather arbitrary; we assume integers and
booleans (e.g. 4, TRUE), together with built-in functions to manipulate
them (e.g. AND, IF, +). We also assume built-in list-processing functions
(e.g. CONS, HEAD).
Figure 2.1 Syntax of a lambda expression (in BNF)
\chapter[Translating a high-level functional language into the lambda calculus][Translating a high-level functional language into the lambda calculus]{Translating a high-level\\ functional language into the\\ lambda calculus}

In the next few chapters we will describe how to translate a high-level functional language into the lambda calculus. We can regard this translation in two ways:

\begin{numbered}
    \item As a description of the semantics of the language, giving the meaning of each of its constructs in terms of lambda expressions, whose meaning is well understood. This is precisely the approach taken by \textit{denotational semantics} [Gordon 1979].
    \item As a step in the implementation of the high-level language, by expressing all its constructs in terms of the lambda notation.
\end{numbered}

For the sake of definiteness we use a subset of the language Miranda [Turner 1985], but the techniques apply to any functional language. An introduction to Miranda can be found in the Appendix.
\vfill

\plainbox{
    \footnotesize
    {\centering

    \textbf{Disclaimer:}

    }
    \noindent In this book Miranda is used as an example of a modern functional programming language, to illustrate various points about the implementation of functional programming languages in general. This book is not intended to be a source of reference for the definition of Miranda. Note that:
    \begin{numbered}
        \item Miranda has a number of features, both major and minor, which are not discussed here at all.
        \item The material about Miranda in this book was based on a prerelease version of the Miranda system and may therefore be inaccurate by the time it is published.
    \end{numbered}

    The Miranda functional programming system is a product of Research Software Limited, and a full description of the language and its programming environment is in preparation by them.
}

\section{The Overall Structure of the Translation Process}
Miranda is a powerful, high-level functional language, providing a rich set of programming constructs. The purpose of the next few chapters is to demonstrate how some of these constructs can be translated into the lambda calculus. Specifically, we will discuss structured data types, pattern-matching, conditional equations and ZF expressions. Miranda includes a number of other constructs, such as abstract data types and structured data types with laws, which we will not study in this book.

Even so, the translation we describe is a substantial task, and we begin by outlining the structure of the translation process. It might be possible to translate a program directly from Miranda into the lambda calculus, but this would be an extremely complicated translation, so we will take a more step-by-step approach. In order to do this, it is convenient to regard much of the translation as a process of successively \textit{transforming} one program into another, until finally the result is a program in the lambda notation. (We are here using `translation' to suggest a process which takes a program in one language and produces a program in another, while a `transformation' produces a program in the \textit{same} language.)

Two ways of organizing the translation then suggest themselves:

\begin{numbered}
    \item We could perform most of the translation by successive transformations of one Miranda program into another, each transformation performing a simplification step. We would complete the process by translating the resulting (simple) Miranda program into the lambda calculus. The idea is that the earlier transformations would have done all the hard work, so the final step should consist of little more than a change of syntax.

    \item Alternatively, we could begin the translation by performing a simple syntactic translation of the Miranda program into an enriched version of the lambda calculus. This enriched lambda calculus would include the ordinary lambda calculus as a subset, but would also include extra constructs, chosen so that the first step consists of little more than a change of syntax. Then we could do most of the hard work by successively transforming the expression into simpler and simpler forms, until it becomes an ordinary lambda expression, free from any of the extra constructs.
\end{numbered}

Initially, the first method looks more attractive than the second, because it does not require us to define a new language (the enriched lambda calculus). However, we choose to follow the second course of action for the following reasons:

\begin{numbered}
    \item Miranda is designed to be a language for programmers, not compilers, and it lacks certain features that are desirable for a transformation-based compiler. (The particular features lacking are lambda abstractions and the ability to qualify any expression with local definitions. This is not a criticism of Miranda -- it just has a different purpose.)

    \item To a much greater extent than is the case for imperative languages, functional languages are largely syntactic variations of one another, with relatively few semantic differences. Using the second method allows the transformations we present to be applied easily to other languages, by altering only the translation of the high-level language into the enriched lambda calculus.
\end{numbered}

Figure 3.1 depicts the overall plan of action. We will use the term \textit{ordinary lambda calculus} to refer to the language described in Chapter 2, and \textit{enriched lambda calculus} to refer to the language introduced here.

The enriched lambda calculus is simply the ordinary lambda calculus augmented with extra constructs, chosen to allow an easy translation from Miranda. For each construct we will:

\begin{numbered}
    \item say what it looks like (give its syntax);
    \item say what it means (give its semantics).
\end{numbered}

\boxedfigure{
    \centering\sffamily\footnotesize

    \begin{minipage}{10cm}
        \centering

        \framebox{\ Miranda program\ }\\
        \hspace{4cm}$\Bigg\downarrow$ \begin{minipage}{4cm}
            {A simple translation\\ (specific to Miranda)}
        \end{minipage}\\
        \framebox{\
            \begin{minipage}{4.5cm}
                {Expression in the\\ enriched lambda calculus}
            \end{minipage}
            \ }\\
        \hspace{4cm}$\Bigg\downarrow$ \begin{minipage}{4cm}
            {Multiple transformations\\ (independent of Miranda)}
        \end{minipage}\\
        \framebox{\
            \begin{minipage}{4.5cm}
                {Expression in the\\ ordinary lambda calculus}
            \end{minipage}
            \ }\\

    \end{minipage}

}{Translation of Miranda into the lambda calculus}


The semantics for each construct can be given by providing a simple transformation which shows how to express that construct in terms of the ordinary lambda calculus. Then we could, in principle, translate from Miranda into the ordinary lambda calculus by first translating into the enriched lambda calculus, and then using the semantics of each construct repeatedly to transform the expression into an ordinary lambda expression.


While this method generates correct results, far greater efficiency is attainable by using more complicated transformations, but we can always confirm their correctness by reference to the inefficient version.

\section{The Enriched Lambda Calculus}
The enriched lambda calculus is a superset of the ordinary lambda calculus, so that any expression in the ordinary lambda calculus is also an expression in the enriched lambda calculus. The syntax for function application, lambda
abstractions, constants and built-in functions therefore remains exactly as described in Chapter 2. Likewise, all functions are written in prefix form, and the same conventions hold concerning brackets.

The only difference from the ordinary lambda calculus is the provision of four extra constructs. They are:

\begin{numbered}
    \item \ml{let}-expressions and \ml{letrec}-expressions;
    \item pattern-matching lambda abstractions;
    \item the infix operator \fatbar;
    \item case-expressions.
\end{numbered}

Of these, we will only describe the first here. The other three all concern pattern-matching, and cannot be defined before the discussion of pattern-matching itself. This is given in Chapter 4, and the remaining three constructs are defined there.

Figure 3.2 summarizes the syntax of the enriched lambda calculus for future reference.

\subsection{Simple let-expressions}
One of the main constructs in any functional language is the \textit{definition}, whereby a name is bound to a value. This mechanism is provided in the enriched lambda calculus, using \ml{let}-expressions and \ml{letrec}-expressions.

We begin by defining \textit{simple} \ml{let}-expressions. They are called `simple' by contrast with \textit{pattern-matching} \ml{let}-expressions, which we deal with later. A simple \ml{let}-expression has the following syntax:
\begin{mlcoded}
    let v = B in E
\end{mlcoded}

\boxedfigure{
\hspace{-2em}
\begin{tabular}{llll}
    <exp> & ::= & <constant> & Constants \\
          & | & <variable>   & Variables \\
          & | & <exp> <exp>  & Applications \\
          & | & \tl{} <variable> . <exp>   & Lambda abstractions \\
          & | & \ml{let\phantom{rec}} <pattern> = <exp> \ml{in} <exp> & \ml{Let}-expressions \\
          & | & \ml{letrec} <pattern> = <exp>  & \ml{Letrec}-expressions \\
          &   & \phantom{\ml{letrec}} $\cdots$ & \\
          &   & \phantom{\ml{letrec}} <pattern> = <exp>  &  \\
          &   & \ml{in} <exp> & \\
          & | & <exp> \fatbar{} <exp>  & Fat bar \\
          & | & \ml{case} <variable> \ml{of} & \ml{Case}-expression \\
          &   & <pattern>\quad $\Rightarrow$\quad <exp> & \\
          &   & $\cdots$ & \\
          &   & <pattern>\quad $\Rightarrow$\quad <exp> & \\
\end{tabular}

\hspace{-2em}
\begin{tabular}{lllll}
    <pattern> & ::= & <constant> &  & Constant patterns\\
              & |   & <variable> &  & Variable patterns\\
              & |   & <constructor> & <pattern> ,\qquad\qquad  & Constructor patterns\\
              &     &  & $\cdots$  & \\
              &     &  & <pattern>  & \\
\end{tabular}

}{Syntax of enriched lambda expressions}


\noindent where the \ml{v} is a variable, and \ml{B} and \ml{E} are expressions in the (enriched) lambda notation.

It introduces a definition for a variable \ml{v}, which binds \ml{v} to \ml{B} in \ml{E}. The definition is in scope with \ml{E} but not \ml{B}. We say that the `\ml{v = B}' is the \textit{definition of the} \ml{let}, the \ml{v} is the variable \textit{bound by the} \ml{let}, and the \ml{B} is the \textit{definition body}.

For example, consider the following \ml{let}-expression:
\begin{mlcoded}
    let x $=$ 3 in ($*$ x x)
\end{mlcoded}
Intuitively, the value of this expression is found by substituting \ml{3} for \ml{x} in the body \ml{($*$ x x)}, and then evaluating the body, giving the result \ml{9}:
\begin{mlalign}
    & let x $=$ 3 in ($*$ x x)\\
    $\rightarrow$ & $*$ 3 3\\
    $\rightarrow$ & 9
\end{mlalign}

A \ml{let}-expression is an expression like any other, and can be used in the same way as any other expression. For example,
\begin{mlalign}
    & $+$ 1 (let x $=$ 3 in ($*$ x x))\\
    $\rightarrow$ & $+$ 1 ($*$ 3 3)\\
    $\rightarrow$ & $+$ 1 9\\
    $\rightarrow$ & 10
\end{mlalign}

For the same reason, \ml{let}-expresions can be nested:
\begin{mlalign}
    & let x $=$ 3 in (let y $=$ 4 in ($*$ x y))\\
    $\rightarrow$ & let y $=$ 4 in ($*$ 3 y)\\
    $\rightarrow$ & $*$ 3 4\\
    $\rightarrow$ & 12
\end{mlalign}

As a matter of convenience, we also allow ourselves to write multiple definitions in the same let; thus:
\begin{letalign}
    let & x $=$ 3\\
        & y $=$ 4\\
    in  & $*$ x y
\end{letalign}

This expression means precisely the same as the previous one. We define a \ml{let}-expresion with several definitions to mean the same as the nested set of \ml{let}-expresions which defines the same variables in the same order, one per \ml{let}-expresion. (Syntactically, it would have been possible to specify that multiple definitions are separated with semicolons, but layout will suffice for our purposes.)

Earlier in this section we developed an informal reduction rule for \ml{let}-expresions. This involved \textit{substitution} and is very reminiscent of the \tb{}-reduction rule, which also uses substitution. For example, to evaluate
\begin{mlcoded}
    (\tlb{x}$*$ x x) 3
\end{mlcoded}

we substitute \ml{3} for \ml{x} in the body \ml{($*$ x x)}, and then evaluate the body. Generalizing this idea, we can now define the semantics of a simple \ml{let}-expresion as follows:
\begin{mlcoded}
    (let v $=$ B in E) $=$ ((\tlb{v}E) B)
\end{mlcoded}

(We use the symbol $=$ to denote the equivalence of two expressions.) That is all that is needed to define its semantics! By repeated application of this equivalence, we could eliminate all simple \ml{let}-expresions from an expression, in favor of lambda abstractions.

\subsection{Simple \ml{letrec}-expresions}
The syntax of a \textit{simple} \ml{letrec}-expresion is similar to that of a simple \ml{let}-expresion:
\begin{letalign}
    letrec & v$_1 =$ E$_1$ \\
           & v$_2 =$ E$_2$ \\
           & $\cdots$ \\
           & v$_n =$ E$_n$ \\
    in & \\
    & E
\end{letalign}

where the \ml{v$_i$} are variables, and \ml{E}, \ml{E$_1$}, ..., \ml{E$_n$} are expressions in the (enriched) lambda notation. We will sometimes abbreviate `\ml{letrec}-expresion' to `letrec' (and `\ml{let}-expresion' to `\ml{let}'), where no ambiguity arises.

The term `\ml{letrec}' is short for `let recursively', and it introduces possibly recursive bindings for a number of variables \ml{v$_i$}. The difference between \ml{let}s and \ml{letrec}s is that the \ml{v$_i$} are in scope in the \ml{E$_i$} (as well as \ml{E}) of a \ml{letrec}. To take an example, the expression
\noindent
\begin{letalign}
    letrec & factorial $=$ \tlb{n}IF ($=$ n 0) 1 ($*$ n (factorial ($-$ n 1)))\\
    in & factorial 4
\end{letalign}
defines a recursive function \ml{factorial}, and applies it to the argument \ml{4}. The value of the expression is thus \ml{24}.

Like \ml{let}-expresions, \ml{letrec}-expresions can appear embedded anywhere in an expression. Unlike \ml{let}-expresions, however, it is essential to allow multiple definitions in a \ml{letrec}-expresion, so as to permit mutual recursion. This is demonstrated by the following example:
\begin{letalign}
    letrec & f $= \ldots$ f $\ldots$ g $\ldots$ \\
           & g $= \ldots$ f $\ldots$ \\
    in $\ldots$ &
\end{letalign}

Here, \ml{f} refers to itself and \ml{g}, and \ml{g} refers to \ml{f}. This cannot be transformed into a nested pair of \ml{letrec}s, because then either \ml{g} would not be in scope in the body of \ml{f}, or vice versa.

It is easy to provide a semantics for a \ml{letrecs} with only a single definition, using the \ml{Y} operator developed in Section 2.4. In particular,
\begin{mlcoded}
    (letrec v $=$ B in E) $=$ (let v $=$ Y (\tlb{v}B) in E)
\end{mlcoded}
The use of \ml{Y} renders the definition non-recursive, so we can then use a \ml{let}-expresion, whose semantics has already been defined.

The case of multiple definitions requires the use of pattern-matching, and so is postponed until Chapter 6.

\subsection{Pattern-matching let$-$ and \ml{letrec}-expresions}
We will also allow \textit{patterns}, as well as variables, to appear on the left-hand side of definitions in \ml{let}s and \ml{letrec}s. We have not yet defined what a pattern is, so we postpone the topic until Chapter 6. However, a variable is just a simple form of pattern, so simple \ml{let(rec)}-expressions are just simple forms of pattern-matching \ml{let(rec)}-expressions.

\subsection{Let(rec)s versus Lambda Abstractions}
So far we have regarded the ordinary lambda calculus as the target language, into which we will transform the program, and \ml{let(rec)}-expressions as intermediate embellishments. However, there are strong efficiency reasons for including \textit{simple} \ml{let(rec)}-expressions in the target language, rather than transforming them into the ordinary lambda calculus.

Specifically, the transformation of a \ml{let}-expresion
\begin{mlcoded}
    let v $=$ B in E
\end{mlcoded}
into the application of a lambda abstraction
\begin{mlcoded}
    (\tlb{v}E) B
\end{mlcoded}
is using a sledgehammer (lambda abstraction) to crack a nut (\ml{let}-expresions). The lambda abstraction \ml{(\tlb{v}.E)} could be applied to many arguments, but it is in fact only ever applied to one, namely \ml{B}. The generality of lambda abstraction is not required, and the special case (that of application to a unique argument) can be exploited by the more sophisticated compilers described later in this book.

This issue manifests itself in a number of ways:
\begin{numbered}
    \item Miranda is a polymorphically typed language, and in Chapter 8 we give an algorithm for type-checking programs. Unfortunately, it is not possible to type-check the program once it has been transformed into the ordinary lambda calculus, but the addition of simple \ml{let(rec)}-expressions is sufficient to solve the problem.
    \item In all implementations except the very simplest, \ml{let}-expressions can be evaluated very much more efficiently than the corresponding application of a lambda abstraction. This applies to all the implementations described from Chapter 14 onwards.
    \item A related problem is that the transformation of \ml{letrec}-expressions into the ordinary lambda calculus compels us to use \ml{Y} to express recursion. The resulting expression is not an efficient implementation, and a more
    sophisticated compiler may wish to handle recursion in a different way
    (see Chapter 14). Keeping the recursion explicit using \ml{letrec} allows scope
    for these optimizations.
\end{numbered}

To summarize, all our implementations, except the very simplest, will require the program to be transformed into the ordinary lambda calculus augmented with simple \ml{let(rec)}-expressions. This approach makes a dramatic contribution to the efficiency of the resulting implementations. On the other hand, little seems to be gained by augmenting the language still further.

\section{Translating Miranda into the Enriched Lambda Calculus}
A program consists of a set of definitions, together with an expression to be evaluated. To keep these two components of the program separate we will use a box, like this:

\begin{center}
\setlength{\tabcolsep}{18pt}
\renewcommand{\arraystretch}{1.5}
\begin{tabular}{|c|}
    \hline
    Set of definitions \\
    \hdashline
    Expression to be evaluated \\
    \hline
\end{tabular}
\end{center}

\noindent For example, we could compute twice the square of 5 with the following Miranda program:
\begin{center}
\setlength{\tabcolsep}{18pt}
\renewcommand{\arraystretch}{1.5}
    \begin{tabular}{|c|}
        \hline
        \ml{square n $=$ n*n } \\
        \hdashline
        \ml{2 $*$ (square 5)} \\
        \hline
    \end{tabular}
\end{center}

\noindent(Note: Miranda is an interactive language, and defines a `program' to be a set of definitions, while the `expression to be evaluated' is typed by the user. For the rest of this book, however, we will use `program' to mean `a set of definitions together with an expression to be evaluated'.) Proceeding informally, we can translate this Miranda program into the enriched lambda calculus quite easily, to produce the expression
\begin{letalign}
    let &square $=$ \tlb{n}$*$ n n\\
    in &($*$ 2 (square 5))
\end{letalign}

We now introduce some notation to help describe the translation process. Consider the translation of the Miranda expression \ml{(2 $*$ (square 5))} into the lambda expression \ml{($*$ 2 (square 5))}. We may regard this translation process as a \textit{function} \metafn{TE}, which takes the Miranda expression as its input, and produces the lambda expression as its output. We write the translation like this:
\begin{mlcoded}
    \metafnbb{TE}{2 $*$ (square 5)} $\equiv *$ 2 (square 5)
\end{mlcoded}
The double square brackets \doublebracket{} are used to enclose the Miranda expression,
to emphasize that the argument to \metafn{TE} is a \textit{syntactic} object. This convention was used in Chapter 2, but the difference on this occasion is that the result of the translation is a syntactic object also, and we use \ml{$\equiv$} rather than \ml{$=$} to remind us of this fact. We call \metafn{TE} a \textit{translation scheme}.

We also need another translation scheme \metafn{TD}, which translates Miranda definitions into definitions suitable for a \ml{letrec}. For example,

\begin{mlcoded}
    \metafnbb{TD}{square n $=$ n*n} $\equiv$ square $=$ \tlb{n}$*$ n n
\end{mlcoded}

Here we see another reason for using \ml{$\equiv$} when writing translation schemes: it avoids confusion with \ml{$=$} symbols in the program being translated. We can now generalize the translation scheme as follows. Given the Miranda program

\begin{center}
    \setlength{\tabcolsep}{18pt}
    \renewcommand{\arraystretch}{1.5}
    \begin{tabular}{|c|}
        \hline
        \ml{\strut Definition$_1$} \\
        $\vdots$\\
        \ml{\strut Definition$_n$} \\
        \hdashline
        \ml{\strut Expression} \\
        \hline
    \end{tabular}
\end{center}
we generate the following (enriched) lambda expression:
\begin{mlcoded}
    letrec \\
    \phantom{ww}\metafnbb{TD}{Definition$_1$} \\
    \phantom{ww}$\cdots$ \\
    \phantom{ww}\metafnbb{TD}{Definition$_n$} \\
    in\\
    \phantom{ww}\metafnbb{TE}{Expression}
\end{mlcoded}
In the previous example we used a \ml{let} instead of a \ml{letrec}, but Miranda definitions are all potentially recursive, so we must use a \ml{letrec} in general (later work will optimize this--Section 6.2.8).

What we have now done is to reduce the translation problem to one of defining the two translation schemes \metafn{TD} and \metafn{TE}. We will define them for simple cases in the succeeding two sections, and then lay out the plan of the next few chapters, which will extend them to cover more complicated cases.

For the moment, we completely avoid the question of declarations of new types and type-checking. The former will be introduced in Chapter 4 and the latter in Chapter 8.

\subsection{The \metafn{TE} Translation Scheme}

The translation scheme \metafn{TE} is a function, which takes a Miranda expression as its argument, and produces an equivalent lambda expression as its result, thus:

\begin{center}
    \framebox{\strut\ Miranda expression\ } $\xrightarrow{\qquad\text{\normalsize \metafn{TE}}\qquad}$ \framebox{\strut\ Lambda expression\ }
\end{center}

\noindent We will describe \metafn{TE} by case analysis, giving a rule for each possible form of a
Miranda expression.


%\backmatter

%\include{appendix}

\end{document}
