\chapter[Preface]{}

% The Preface "chapter" is typeset as if the word preface were the number instead of the title,
% except it is in all caps as well.
\vspace{-3.5cm}
\makebox[2pt][l]{
{\normalfont\fontsize{20}{20}\selectfont\sffamily
\hspace{0.75cm}
\makebox[2pt][l]{ \hspace{-1.5cm}
	\rule[-6pt]{\widthof{mmPREFACE}}{3pt}
}PREFACE}
}

\vspace{1.5cm}

\noindent This book is about implementing functional programming languages using
graph reduction.

Functional languages have become the focus of much active research in
recent years
%\citep{backus_can_1978} \citep{peyton-jones_directions_1984}
[Backus, 1978] [Peyton Jones, 1984], but their acceptance has
been delayed by the inefficiency of their available implementations when
compared with more conventional languages.

This situation has changed recently, with the advent of rather fast implementations of functional languages such as Cardelli's ML [Cardelli, 1983],
Fairbairn's Ponder [Fairbairn, 1982], and the Chalmers Lazy ML compiler
[Johnsson, 1984]. These implementations rival the speed of compilers for
more conventional languages.

There appear to be two main approaches to the efficient implementation of
functional languages. The first is an environment-based scheme, exemplified
by Cardelli's ML implementation, which derives from the experience of the
Lisp community. The other is graph reduction, a much newer technique first
invented by Wadsworth [Wadsworth, 1971], and on which the Ponder and
Lazy ML implementations are founded. Despite the radical differences in
beginnings, the most sophisticated examples of each approach show remark-
able similarities.

The techniques of graph reduction are to be found scattered amongst the
proceedings of various conferences and workshops, and it is one purpose of
this book to collect some of this work together. The book is intended to have
two main applications:
\begin{compactenum}[(i)]
\item As a course text for part of an undergraduate or postgraduate course on
the implementation of functional languages.
\item As a handbook for those attempting to write a functional language
implementation based on graph reduction.
\end{compactenum}
The material is presented in a fairly informal tutorial fashion, the idea being to
convey the framework and some of the intuitions that will render the original
sources more accessible.

Chapters 5 and 7 were written by Philip Wadler, of the Programming
Research Group, Oxford, and Chapter 4 was written in collaboration with
him. Chapters 8 and 9 were written by Peter Hancock, of Metier Management
Systems Ltd, and currently at the Programming Research Group, Oxford. I
gratefully acknowledge their patience in writing and rewriting their drafts.

I am extremely grateful to a number of other people who have made
significant technical contributions to the book. David Turner's help was
invaluable, in offering comments on the parts of the book that relate to
Miranda. The Appendix which gives an introduction to Miranda, was written
by him. (Miranda is a trademark of Research Software Limited.)

Much of the information about Miranda in the first part of the book is based
on a prerelease version of the Miranda system, and I am grateful to Research
Software Limited for the permission to include this material.

Simon Finn, of the University of Stirling, made a number of penetrating
observations about the treatment of pattern-matching, which resulted in
significant improvements in Chapters 4 and 6.

Several long discussions with Thomas Johnsson of Chalmers University,
Goteborg, radically changed the shape of the G-machine chapters, and
Chapter 21 was written in collaboration with him. John Fairbairn and Stuart
Wray of Cambridge also made important contributions in this area.

Paul Hudak and Robert Keller helped me in writing the last section of
Chapter 24.

Many other people have given me welcome help and encouragement, and
have helped enormously by reading the text and making constructive
suggestions. These include Lennart Augustsson, Philip Boswell, Geoff Burn,
Nigel Chapman, Chris Clack, Pierre-Louis Curien, Dan Friedman, Kevin
Hammond, Peter Hancock, Chris Hankin, John Hughes, Thomas Johnsson,
Dick Kieburtz, Phil Molyneux, Benedict Nixon, Martin Packer, Ellen Poon,
Jon Salkild, William Stoye, David Turner, Phil Wadler, John Washbrook and
Russel Winder. I am particularly grateful to John Washbrook for his unfailing
support.

\vspace{-\baselineskip}
\begin{flushright}
Simon L. Peyton Jones\\
University College London\\
London WC1E 6BT\\
simonpj@uk.ac.ucl.cs
\end{flushright}

%\pg
\vspace{-2\baselineskip}

%\bibliography{bib/SPLJ}

\section*{References}

\begin{references}
\item Backus, J. 1978. Can programming be liberated from the von Neumann style? A
functional style and its algebra of programs. \textit{Communications of the ACM}. Vol.~21, no.~8, pp. 613-41.
\item Cardelli, L. 1983. The functional abstract machine. \textit{Polymorphism}. Vol.~1,~no.~1.
\item Fairbairn, J. 1982. Ponder and its type system. \textit{Technical Report 31}. Computer Lab.,
Cambridge. November.
\item Johnsson, T. 1984. Efficient compilation of lazy evaluation. In \textit{Proceedings of the ACM
Conference on Compiler Construction, Montreal}, pp. 58-69. June.
\item Peyton Jones, S.L. 1984. Directions in functional programming research. In SERC
\textit{Distributed Computing Systems}, pp. 220-49. Duce (editor). Peter Peregrinus.
\item Wadsworth, C.P. 1971. Semantics and pragmatics of the lambda calculus, Chapter 4.
PhD thesis, Oxford.

\end{references}
