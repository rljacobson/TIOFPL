\chapter{Polymorphic Type-Checking}
\chapterauthors{Peter Hancock}
\vspace{3cm}

\noindent
In common with several other modern programming languages, Miranda
has the property that a programmer need not specify the types of the objects
defined in his program. The compiler can work out those types, if the program
can be consistently typed at all. The part of the compiler that does this is
usually called the `type-checker'. It attempts to infer the types of expressions
in the program from their contexts. This kind of type-checking was first
implemented for the language ML, around 1976. The type discipline was first
expounded by Milner [1978].

Whether or not a type-checker requires information from the programmer
to check that a program is well typed, type-checking is of great value in
drawing the programmer's attention to a variety of errors, from trivial slips in
program entry, to gross logical blunders. It helps us to write robust programs.

Another advantage of type-checking is that it helps to build faster
implementations of programming languages. If a program is passed by the
type-checker, then no type error should occur at run-time, such as the use of
an integer as if it were a function, a boolean as if it were an integer, or a
function as if it were a tuple. In Milner's words, well-typed expressions do not
`go wrong': at run-time we will never misinterpret the representation of an
expression. By omitting run-time checks for such errors, the implementation
of a language can be made simpler and faster. Of course, any implementation
should still provide for diagnosis of its own internal errors.

The purpose of this chapter is to explain in some detail how a type-checker
works. Then, in Chapter 9, we put the ideas into practice by constructing a
type-checker for a simple functional language. The type-checker is constructed
in Miranda, in the hope that the development of such a functional
program may itself be of some additional interest.

Given the informal spirit of this book, and its concentration on setting up
intuitions rather than on attaining impregnable conceptual rigor, it is not
appropriate to proceed `from the ground up'. Instead, we shall assume that
the reader already has some understanding of the notion of a type, and wishes
to see how that notion can be applied in practice. Nevertheless, some
cautionary remarks may be in order, and they are made at the end of the
chapter.

This chapter is organized as follows. Section 8.1 reviews some basic
concepts, and notations for types. Section 8.2 illustrates the concept of
polymorphism, using several examples. Section 8.3 shows in an informal way
how types may be inferred from the structure of a definition. Section 8.4 sets
out the language for which we will build a type-checker. Section 8.5 considers
the detailed type structure of expressions in the language, and attempts to
clarify the rules of type inference, which are summarized in Section 8.6.
Section 8.7 contains the cautionary remarks referred to before.

\textit{Important note:} The type-checker described here is actually somewhat
more liberal than that of the Miranda compiler itself, in that it will succeed in
type-checking some programs which the Miranda compiler would reject. This
difference is explained in Section 8.5.5. The Miranda type-checker is also
considerably more sophisticated than the one we describe here, because it
supports features, such as abstract data types and a module structure, which
are beyond the scope of this book.

\section{Informal Notation for Types}

The types with which we are concerned in functional programming include
ground types such as characters, numbers and booleans, types of tuples, lists
and, of course, functions. To talk about these types, we will use the following
notation. Capital letters will be used for type variables. A type variable A
stands for a type in much the same way that a numerical variable n stands for a
number in mathematics. Lower-case letters will be used for the elements of
types. The notation
\begin{mlcoded}
a \hastype{} A
\end{mlcoded}
means that \ml{a} has type \ml{A}. For example, \ml{42\hastype num}, \ml{'f'\hastype char}, where \ml{num} is the
type of numbers, and \ml{char} is the type of characters. (Note: the notation used
for types in this chapter differs from that of Miranda -- in Miranda an upper
case letter cannot stand for a type.)

\subsection{Tuples}

Given types \ml{A} and \ml{B}, \ml{(A,B)} is the type of ordered pairs \ml{(a,b)} where \ml{a\hastype A}, and
\ml{b\hastype B}. Using Descartes' terminology, \ml{a} is the first \textit{coordinate} of \ml{(a,b)}, and \ml{b} is
the second. More generally, if $n \geq 2$ and \ml{A$_1$}, \ldots , \ml{A$_n$} are types, then
\begin{mlcoded}
    (A$_1$, \ldots, A$_n$)
\end{mlcoded}
is the type whose values are of the form of tuples
\begin{mlcoded}
(a$_1$, \ldots, a$_n$)
\end{mlcoded}
where \ml{a$_1$\hastype A$_1$}, \ldots, \ml{a$_n$\hastype A$_n$}. The important points about tuples, so far as typing is
concerned, are:
\begin{numbered}
    \item the coordinates of a tuple need not be of the same type;
    \item the type of a tuple determines the number of its coordinates (that is, its
    dimension), and their types.
\end{numbered}


\subsection{Lists}

Given a type \ml{B}, \ml{[B]} is the type of lists whose entries are of type \ml{B}. More
specifically, an object of type \ml{[B]} must be
\begin{numbered}
    \item either the empty list, which is denoted by \ml{[\,]};
    \item or a non-empty list, formed by prefixing an object \ml{b\hastype B} to a list \ml{bs\hastype[B]},
    which is denoted by \ml{b:bs}.
\end{numbered}
If all the successive entries \ml{b$_1$}, \ldots, \ml{b$_k$} of a finite list are known, we may write
it using the notation
\begin{mlcoded}
[b$_1$, \ldots, b$_k$]
\end{mlcoded}
The important points about lists, so far as typing is concerned, are:
\begin{numbered}
    \item In contrast with the coordinates of a tuple, all entries of a list must be of
    the same type. For example, it would make no sense to form a list in which
    the entries were alternately characters and truth values. (We could in fact
    define a type of such entities, but they would not be lists.)
    \item In contrast with the dimension of a tuple, the length of a list is not
    determined by its type. Indeed, when programming in a lazy language,
    we may operate with infinite lists such as the list of positive integers.
    There is no requirement that a list must be built up from the empty list by
    a finite number of applications of the prefixing operation (\ml{b:bs}), or that a
    principle of well-founded induction on the structure of lists should be valid.
\end{numbered}

\subsection{Structured Types}

Tuple types and list types are both examples of structured types, which were
introduced in Chapter 4. As explained there, in Miranda the general form of a
declaration of an operator for forming structured types is:
\begin{mlcoded}
    name v$_1$ \ldots{} v$_k$ \typedecl{} \quad c$_1$ t$_{1,1}$ \ldots{} t$_{1,r_1}$ \\
    \phantom{this is spacing hack} | \ldots \\
    \phantom{this is spacing hack} | c$_m$ t$_{m,1}$ \ldots{} t$_{m,r_m}$
\end{mlcoded}
where $m \geq 1$, $r_i \geq 0$ for $1 \leq i \leq m$, and $k \geq 0$. Here $v_1$, \ldots, $v_k$ stand for schematic type variables, which in Miranda have the special form \ml{$\ast$}, \ml{$\ast\ast$}, \ml{$\ast\ast\ast$}, etc. Also,
\ml{t$_{1,1}$}, \ldots, \ml{t$_{m,r_m}$} are type expressions, built up using variables from the list \ml{v$_1$}, \ldots, \ml{v$_k$}
and names for type-forming operations which are either built-in or declared
elsewhere in the script.

For example, in the type declaration
\begin{mlcoded}
    tree $\ast$ \typedecl{} LEAF $\ast$ | BRANCH (tree $\ast$) (tree $\ast$)
\end{mlcoded}
\ml{v$_1$} is \ml{$\ast$}, \ml{c$_1$} is \ml{LEAF}, and \ml{t$_{1,1}$} is \ml{$\ast$}; \ml{c$_2$} is \ml{BRANCH}, and \ml{t$_{2,1}$}, \ml{t$_{2,2}$} are both \ml{(tree $\ast$)}.
\ml{tree$'$} is a type-forming operator since, given a type as `argument', it produces a
type as its `result'; for example, \ml{(tree char)}, \ml{(tree num)}, \ml{(tree (tree num))}. In
this sense, the built-in basic types (such as \ml{char}, \ml{num}, \ml{bool}) are simply type-forming operators which take no arguments.

A declaration with the form above means that an object of a type
\begin{mlcoded}
    name t$'_1$ \ldots t$'_k$
\end{mlcoded}
must have one of the constructed forms
\begin{mlcoded}
    c$_i$ x$_1$ \ldots x$_n$
\end{mlcoded}
where \ml{x$_j$:t$'_{i,j}$} for $1 \leq j \leq n$, and \ml{t$'_y$} denotes the result of simultaneously
substituting the type expressions \ml{t$'_1$}, \ldots, \ml{t$'_k$} for the type variables \ml{v$_1$}, \ldots, \ml{v$_k$} in
the type expression \ml{t$_{i,j}$}.

For example, here is an object of type \ml{(tree char)}:
\begin{mlcoded}
    BRANCH (LEAF 'a') (LEAF 'b')
\end{mlcoded}
In this case, \ml{t$'_1$} is \ml{char}; the form of the object is a \ml{BRANCH}, and \ml{x$_1$} is
\ml{(LEAF 'a')\hastype{}tree char}, \ml{x$_2$} is \ml{(LEAF 'b')\hastype{}tree char}.

\subsection{Functions}
Given types \ml{A} and \ml{B}, we use the notation:
\begin{mlcoded}
    A $\rightarrow$ B
\end{mlcoded}
to denote the type of functions \ml{f} applicable to objects \ml{a\hastype{}A}, whose values (\ml{f a})
are of type \ml{B}.

For example, \ml{(char $\rightarrow$ num)} is the type of integer-valued functions of
characters. The function `code' which maps a character to its ASCII code is of
this type.

\ml{(char $\rightarrow$ bool)} is the type of boolean-valued functions of characters. For example, the function
\begin{mlcoded}
    isdigit ch = (code '0' <= x) \& (x <= code '9') \\
    \phantom{isdigit ch =} where x = code ch
\end{mlcoded}
is a function of this type.

\ml{([char] $\rightarrow$ [num])} is the type of functions whose arguments are lists of characters, and whose values are lists of integers. The function which returns the list of ASCII codes corresponding to a character list is of this type.


(Note: in functional programming, we consider a function to belong to a
type \ml{(A $\rightarrow$ B)} even though it is not totally defined on the domain type \ml{A}. For
example, the partial function which assigns to every even number its successor
has type \ml{(num $\rightarrow$ num)}.)

The arrow in the function type notation \ml{(A $\rightarrow$ B)} is considered to be a
right-associative binary operator. So
\begin{mlcoded}
    A $\rightarrow$ B $\rightarrow$ C
\end{mlcoded}
means the same as
\begin{mlcoded}
    A $\rightarrow$ (B $\rightarrow$ C)
\end{mlcoded}
and
\begin{letalign}
    & (A $\rightarrow$ B $\rightarrow$ C) \\
    $\rightarrow$ & (A $\rightarrow$ B) \\
    $\rightarrow$ & A \\
    $\rightarrow$ & C \\
\end{letalign}
means the same as
\begin{mlcoded}
    (A $\rightarrow$ (B $\rightarrow$ C)) $\rightarrow$ ((A $\rightarrow$ B) $\rightarrow$ (A $\rightarrow$ C))
\end{mlcoded}
(We shall often lay out a large type expression over several lines, as above.)

The reason we choose \ml{$\rightarrow$} to be right associative can be seen by considering
a (curried) function \ml{f} of two arguments \ml{a\hastype{}A} and \ml{b\hastype{}B}. Then we have:
\begin{letalign}
    f &\hastype{} A $\rightarrow$ B $\rightarrow$ C \\
    (f a) &\hastype{} B $\rightarrow$ C \\
    (f a b) &\hastype{} C
\end{letalign}
If \ml{$\rightarrow$} were left associative, we would have to write
\begin{mlcoded}
    f \hastype{} A $\rightarrow$ (B $\rightarrow$ C)
\end{mlcoded}
which is less convenient, since it uses more brackets.

\section{Polymorphism}

Many of the functions we define in a functional program are to a greater or
lesser degree indifferent to the types of their arguments. This can be
illustrated with a few examples.

\subsection{The Identity Function}

The identity function \ml{id}, defined by
\begin{mlcoded}
    id x = x
\end{mlcoded}
works equally well on arguments of any type. For example, in
\begin{letalign}
    id 3 &= 3 \\
    id 'a' &= 'a' \\
    id (3, 'a') &= (3, 'a')
\end{letalign}
the function \ml{id} is used with the types
\begin{mlcoded}
    num $\rightarrow$ num \\
    char $\rightarrow$ char \\
    (num, char) $\rightarrow$ (num, char)
\end{mlcoded}
In this sense, \ml{id} is \textit{indifferent to the type of its arguments}. However, \ml{id} always
returns a result of the same type as its argument. We express this by saying
that \ml{id} is of type \ml{A $\rightarrow$ A}, for all types \ml{A}.

Sometimes we omit the `for all types \ml{A}' (the jargon for which is \textit{schematic
generality}; \ml{A} is said to be a \textit{schematic} (or \textit{generic}) \textit{variable}). When the
schematic variables are not given explicitly, every type variable is here to be
understood as a schematic variable.

To say that \ml{id} is of type \ml{(A $\rightarrow$ A)} for all types \ml{A} means that the name \ml{id} can
occur in a larger expression in any context suitable for a function whose type is
of that form. When we indicate a form by means of a type expression, we
should say which parts of the expression may vary, by indicating the schematic
variables. To say that a type \ml{T} \textit{is of the form}
\begin{mlcoded}
    \ldots A\ldots B\ldots A\ldots C\ldots
\end{mlcoded}
where \ml{A} and \ml{B} are the schematic variables, is to say that \ml{T} may be obtained by
substituting certain types \ml{TA} and \ml{TB} for the schematic variables. In other
words, \ml{T} is a \textit{substitution instance} of the indicated type. The types
\begin{mlcoded}
    num $\rightarrow$ num \\
    char $\rightarrow$ char \\
    (num, char) $\rightarrow$ (num, char)
\end{mlcoded}
are all substitution instances of the form
\begin{mlcoded}
    A $\rightarrow$ A
\end{mlcoded}
where it is understood that \ml{A} is the schematic variable.

For a final example, consider the expression:
\begin{mlcoded}
    id (code (id 'a'))
\end{mlcoded}
The first occurrence of \ml{id} must have type \ml{(num $\rightarrow$ num)}, and the second must
have type \ml{(char $\rightarrow$ char)}. Since these are both substitution instances of the
type of \ml{id}, \ml{(A $\rightarrow$ A)}, the expression is correctly typed.

Note: What we here call schematic type variables are called in Miranda
generic type variables and written using the special symbols \ml{$\ast$}, \ml{$\ast\ast$}, etc. to
distinguish them from ordinary (non-generic) names for types.)

\subsection{The Length Function}

The function which returns the length of a list may be defined by the equations
\begin{letalign}
    length [\,] &= 0 \\
    length (x:xs) &= (length xs) + 1
\end{letalign}

The function \ml{length} works equally well on any list, regardless of the type of its
entries. For example, in the equations:
\begin{letalign}
    length [7,1,4] &= 3 \\
    length ['7','1','4','z'] &= 4 \\
    length [(3,'a'),(26,'z')] &= 2 \\
    length [id, id] &= NN
\end{letalign}
the function is used with the types:
\begin{letalign}
    [num] &$\rightarrow$ num \\
    {}[char] &$\rightarrow$ num \\
    {}[(num, char)] &$\rightarrow$ num \\
    {}[(A $\rightarrow$ A)] &$\rightarrow$ num
\end{letalign}
respectively. We express the type of \ml{length} by
\begin{mlcoded}
    length \hastype{} [A] $\rightarrow$ num, \normalfont{for all types} A
\end{mlcoded}
which conveys that
\begin{numbered}
    \item \ml{length} is a function;
    \item its arguments are lists;
    \item its values are numbers;
    \item the type of the entries in the argument list does not matter.
\end{numbered}

\subsection{The Composition Function}

Let us represent the composition of two functions \ml{f} and \ml{g} with a right-associative infix dot, and define
\begin{mlcoded}
    (f . g) x = f (g x)
\end{mlcoded}
(We shall write the composition function `\ml{compose}' when we do not want to
indicate its arguments.) Composition is well defined so long as both its left-
and right-hand arguments are functions, and the type of arguments of its
left-hand argument is the same as the type of values of its right-hand
argument. For example, the following make perfect sense:
\begin{numbered}
    \item \ml{decode . succ . code}\\
where \ml{succ} denotes the successor of an integer. The expression denotes a
function which returns `\ml{b}' from `\ml{a}', `\ml{c}' from `\ml{b}', and so on. The composition function is used here with the type:
\begin{mlcoded}
    (num $\rightarrow$ char) $\rightarrow$ (char $\rightarrow$ num) $\rightarrow$ char $\rightarrow$ char
\end{mlcoded}
at its first occurrence, and with the type:
\begin{mlcoded}
    (num $\rightarrow$ num) $\rightarrow$ (char $\rightarrow$ num) $\rightarrow$ char $\rightarrow$ num
\end{mlcoded}
at its second.
    \item \ml{code . id} and \ml{id . code}\\
    where \ml{id} is the identity function discussed above. In these expressions, the
    composition function is used with the types:
    \begin{mlcoded}
        (char $\rightarrow$ num) $\rightarrow$ (char $\rightarrow$ char) $\rightarrow$ char $\rightarrow$ num \\
        (num $\rightarrow$ num) $\rightarrow$ (char $\rightarrow$ num) $\rightarrow$ char $\rightarrow$ num
    \end{mlcoded}
    respectively.
    \item \ml{isdigit . decode} \\
    which is the predicate of an integer which is itself the ASCII code of a
    decimal digit. Here the composition function is used with type:
    \begin{mlcoded}
        (char $\rightarrow$ bool) $\rightarrow$ (num $\rightarrow$ char) $\rightarrow$ num $\rightarrow$ bool
    \end{mlcoded}
\end{numbered}

\noindent
We can express the constraint on the types of the arguments of \ml{compose} by
saying:
\begin{mlcoded}
    compose \hastype{} (B $\rightarrow$ C) $\rightarrow$ (A $\rightarrow$ B) $\rightarrow$ A $\rightarrow$ C
\end{mlcoded}
where \ml{A}, \ml{B}, and \ml{C} are the schematic variables.

\subsection{The Function \ml{foldr}}

The function \ml{foldr} may be defined by the equation
\begin{mlcoded}
    foldr f b [\,] = b \\
    foldr f b (a:as) = f a (foldr f b as)
\end{mlcoded}
Again, \ml{foldr} is to a certain extent indifferent to the types of its arguments. For
example, the following make perfect sense:
\begin{numbered}
    \item \ml{foldr plus 0 [7,1,4]} \\
    where \ml{plus} means binary addition. The function \ml{foldr} is used here with the
    type:
    \begin{letalign}
        &(num $\rightarrow$ num $\rightarrow$ num) \\
        $\rightarrow$ & num \\
        $\rightarrow$ & [num] \\
        $\rightarrow$ & num \\
    \end{letalign}
    \item  \ml{foldr append [\,] ["str1","str2","str3"]}\\
    Here \ml{append} is the function which concatenates two lists. The function
    \ml{foldr} is being used here with type:
    \begin{letalign}
        & (string $\rightarrow$ string $\rightarrow$ string) \\
        $\rightarrow$ & string \\
        $\rightarrow$ & [string] \\
        $\rightarrow$ & string
    \end{letalign}
    \item \ml{foldr cons [\,] [5,4,1,4,1]} \\
    Here \ml{cons x y = x:y}. In this expression, \ml{foldr} is used with the type:
    \begin{letalign}
        & (num $\rightarrow$ [num] $\rightarrow$ [num])\\
         $\rightarrow$ &[num] $\rightarrow$ [num] $\rightarrow$ [num]
    \end{letalign}
\end{numbered}

\noindent
In general, \ml{foldr} may be used in any context which requires a type of the form:
\begin{letalign}
    &(A $\rightarrow$ B $\rightarrow$ B) \\
    $\rightarrow$ & B \\
    $\rightarrow$ & [A] \\
    $\rightarrow$ & B
\end{letalign}
where \ml{A} and \ml{B} are the schematic variables.

\subsection{What Polymorphism Means}

Polymorphism is a style of type discipline which seems to have been first
identified by Christopher Strachey [1967]. A programming language has a
polymorphic type discipline if it permits us to define functions which work
uniformly for arguments of different types. For example, in a polymorphic
language, we can define a single function \ml{length} of type:
\begin{mlcoded}
    [A] $\rightarrow$ num
\end{mlcoded}
In contrast, a language with a monomorphic type discipline forces the
programmer to define different functions to return the length of a list of
integers, a list of floating point numbers, a list of binary numerical functions,
and so on. Languages such as Pascal and Algol 68 are monomorphic.

Strachey distinguished between \textit{ad hoc polymorphism}, and \textit{parametric
polymorphism}. A type discipline exhibits \textit{ad hoc} polymorphism if it permits
the use of the same expression to denote distinct operations at distinct types,
such as the use of the addition symbol to denote addition of integers,
rationals, real numbers, ordinals, complex numbers, and so on. This characteristic of a language is often now described as the ability to \textit{overload}
expressions. On the other hand, parametric polymorphism is just polymorphism as explained above.

The words \textit{polymorphic} and \textit{monomorphic} are also sometimes used to
distinguish between objects whose types are described by expressions with
schematic type variables, and those whose type expressions have none. For
example, the empty list is polymorphic, the functions \ml{id}, \ml{compose}, \ml{length} and
\ml{foldr} are polymorphic, while the function \ml{decode} which returns from an
integer the character with that ASCII code is monomorphic.

A polymorphic object may take on different types at different occurrences,
where these different types are substitution instances of the schematic type of
the function. For example, we do not need to have different versions of \ml{foldr}
for each pair of types that instantiate \ml{A} and \ml{B} in the type expression
\begin{mlcoded}
    (A $\rightarrow$ B $\rightarrow$ B) $\rightarrow$ B $\rightarrow$ [A] $\rightarrow$ B
\end{mlcoded}
or to parameterize \ml{foldr} with the type variables \ml{A} and \ml{B}. Precisely the same
code is executed whatever the types \ml{A} and \ml{B} (at least in a naive implementation of the compiler), and it would be artificial to duplicate that code, or
name it differently for each pair of types.

The terminology is also sometimes (perhaps unfortunately) applied to
types themselves. For example, it is said that \ml{foldr} possesses a `polymorphic'
type, meaning that its type is expressed with schematic variables. (Going by
etymology, `polymorphic' should mean `of many forms', and it is precisely in
order to identify a single form that we use an expression with schematic
variables.)

A polymorphic type discipline was first worked out for the language ML
around 1976, and since then has been incorporated in a number of functional
and imperative languages. In pragmatic terms at least, polymorphism represents a significant advance over the type disciplines of languages such as
Pascal or Algol 68.

\section{Type Inference}

This type discipline is not only polymorphic; it has the property that the only
places in a program where we have to mention types at all are in the type
definitions themselves. The type-checker is able, as part of a single process,
\begin{enumerate}
    \item to determine whether the program is well-typed; and
    \item if the program \textit{is} well-typed, to determine the type of any expression in the
    program.
\end{enumerate}
(Of course, to make a program easier to understand we should almost always
accompany a definition with a specification of the type of the defined entity.)

Before delving into the details of type-checking, we should ask ourselves
how we can informally deduce the types of functions given only their defining
equations.

Consider the definition:
\begin{letalign}
    isdigit ch = &(code '0' <= x) \& (x <= code '9') \\
    &where x = code ch
\end{letalign}
From the right-hand side of the definition we can see that, if the function is
well defined at all, its value must be a truth-value, since the outermost
operator \& (conjunction) produces truth-values. Moreover, the infix operator
\ml{<=} which supplies its values as arguments to \& also produces truth-values. (So
we can see that \& is used consistently with its type.) The arguments to \ml{<=} must
both have the type \ml{num}, and this is clearly the case for the actual arguments,
namely (\ml{code '0'}) and (\ml{code '9'}). It follows that \ml{x} must be a number, and for
this to hold, \ml{ch} must have type \ml{char}. So the right-hand side of the definition is well typed, with type \ml{bool}, provided that the argument \ml{ch} has type \ml{char}. Since
the left-hand side of an equation must have the same type as the right-hand
side, we deduce that:
\begin{mlcoded}
    isdigit \hastype{} char $\rightarrow$ bool
\end{mlcoded}
Consider now the definition of \ml{length}, repeated here:
\begin{mlcoded}
    length [\,] = 0 \\
    length (x:xs) = (length xs) + 1
\end{mlcoded}
From the first equation, it is clear that the type of \ml{length} is of the form
\begin{mlcoded}
    [A] $\rightarrow$ num
\end{mlcoded}
We must also look at the second equation to see whether it constrains the
type \ml{A} any further. For example, if the second equation were something like
\begin{letalign}
    length (x:xs) &= (length xs) + 1, \quad x = 'a' \\
    &= length xs
\end{letalign}
(using a conditional expression), we would have to conclude that the type \ml{A} is
not in fact completely general, but completely specific: it is the type \ml{char}. But
in the case of the function \ml{length}, the second clause imposes no further
constraint, so we can say that
\begin{mlcoded}
    length \hastype{} [A] $\rightarrow$ num, \normalfont{for all types} A
\end{mlcoded}
Consider now the function \ml{foldr}, with definition
\begin{mlcoded}
    \begin{tabular}{lll}
    foldr f x = g where &g [] &= x \\
        &g (a:as) &= f a (g as)
    \end{tabular}
\end{mlcoded}
The local function \ml{g} is evidently a function on lists, since it is defined by cases
on the two constructors of list form. So suppose \ml{g} has type \ml{([A] $\rightarrow$ B)}.
Both \ml{x} and \ml{(f a (g as))} must be of type \ml{B}. Since \ml{(g as)} has type \ml{B}, \ml{f} must
have type \ml{(A $\rightarrow$ B $\rightarrow$ B)}. So, all in all,
\begin{mlcoded}
    foldr \hastype{} (A $\rightarrow$ B $\rightarrow$ B) $\rightarrow$ B $\rightarrow$ [A] $\rightarrow$ B
\end{mlcoded}

In general, by examining the context of an expression, we may be able to
deduce an expression for the form of the type of an object which can fit into
that context. By examining the expression itself, we may be able to deduce the
form of the types which that expression can take on. So we have two type
expressions that will usually contain variables, the first giving the form of the
type required by the context (deduced from the `outside'), and the second
giving the form of type which the object can take (deduced from the `inside').
For the whole expression to be well typed, these two type expressions must
match, in the sense that by substituting for the schematic variables of the type
expressions, they can be brought to the same form.

\section{The Intermediate Language}

The language for which we will construct a type-checker is the language of the
lambda calculus. We will use the form of that language in which recursion is
expressed using the \ml{letrec} construct rather than by using the  \ml{Y} combinator.
Briefly, the forms of expression are these:
\begin{numbered}
    \item Variables: \ml{x}, \ml{y}, etc.
    \item Lambda abstractions: \ml{\tlb{x}E}
    \item Application: \ml{E$_1$ E$_2$}
    \item Simultaneous definitions (let-expressions):
    \begin{letalign}
        let &x$_1$ = E$_1$ \\
        &\ldots \\
        &x$_k$ = E$_k$ \\
        in &E
    \end{letalign}
    \item Mutual recursion (\ml{letrec}-expressions):
    \begin{letalign}
        letrec &x$_1$ = E$_1$ \\
        &\ldots \\
        &x$_k$ = E$_k$ \\
        in E &
    \end{letalign}
\end{numbered}
The type-checker should be invoked when the source program has been
brought into this form, and before lambda-lifting, or transformation to a
supercombinator program (see Chapter 13). It is, however, important that
the program is subjected to the dependency analysis referred to in Section 6.2.8
before type-checking. This is for the following reason. If we include in a
\ml{letrec}-expression a definition whose right-hand side does not `really' depend
on the other names defined in the \ml{letrec}, we may not be able to type-check the
program at all. (For an explanation of this, see Mycroft [1984].)

The most conspicuous absentee from this list of constructs is anything
corresponding to function definitions by pattern-matching. But as is shown in
Chapters 4-6, we can replace such definitions by using instead built-in case
functions associated with the type-forming operations defined by the
programmer or supplied by the system. The names of these case functions,
and indeed of the associated discriminators and selectors, can be regarded as
the names of variables with predeclared types. Hence they are of no special
interest in the type-checker.

(In the same vein, we might have taken the easy way out in our treatment of
recursion, and used the \ml{Y} combinator, regarding this as having a priori the
predeclared type
\begin{mlcoded}
    Y \hastype{} (A $\rightarrow$ A) $\rightarrow$ A, for all types A
\end{mlcoded}

However, the issues involved in the problem of how a type discipline should
treat recursion are rather subtle. Although the solution we have adopted is in
fact precisely equivalent to adoption of the \ml{Y} combinator for the expression of recursion, we take the point of view that to do this would be to sweep the
problem under the carpet.)

The type-checking algorithm can still be developed when pattern-matching
is present in the language. Indeed, for practical reasons, it is better to type-check while the program is still close to the form in which it was entered, in
order that error messages can refer to program text that the programmer can
recognize.

\section{How to Find Types}

Presumably, when we construct an expression \ml{E} in a program, we reason to
ourselves that it is well typed. As a product of this reasoning, we are in a
position to say what the type is of any subexpression \ml{E$'$} of \ml{E}. We can, as it
were, label each subexpression with the type which we think it has. When we
enter that expression into the text of our program, that `labelling' has been
lost. It is the job of the type-checker to reason out the type structure of the
expression once again, and to recover the labelling.

If we accept that type-checking is a species of inference, this raises the
question as to what forms of inference we may validly employ in checking the
type of an expression. We shall not go so far as to try to state those forms of
inference explicitly (akin to an exercise in formal logic), but rather by
considering a sufficient variety of examples (as it were, particular syllogisms),
try to work up some confidence that we can tell the difference between right
and wrong inference.

\subsection{Simple Cases, and Lambda Abstractions}

In order to make enough space to expose the type structure of an expression,
let us lay it out as a tree, where at the top we have the variables and constants,
and as we proceed down towards the root, we pass through nodes labelled
with the constructors applied in the formation of the expression. For an
example containing both application and abstraction nodes, take the
expression
\begin{mlcoded}
    (\tlb{x}\tlb{y}\tlb{z}x z (y z))
\end{mlcoded}
Laid out as a tree this becomes
\begin{center}
\begin{forest}
    for tree={
        grow=north, % This makes the tree grow upwards
    }
    [\ml{\tlb{x}\tlb{y}\tlb{z}}
    [\ml{@}
    [\ml{@}
    [\ml{z}] [\ml{y}]
    ]
    [\ml{@}
    [\ml{z}] [\ml{x}]
    ]
    ]
    ]
\end{forest}
\end{center}
Each node in this tree corresponds to a subexpression of the original
expression and should therefore possess a type. Assign arbitrary type labels
\ml{T0}, \ml{T1}, \ldots, \ml{T7} to the nodes of the tree. Drawing the tree in a slightly
different way to use less space, we get:
\[
\inferrule*[right={\!\!@}]{
\inferrule*[right={\!\!@}]{\text{\ml{x\hastype{} T0}} \\ \text{\ml{z\hastype{} T1}}}
{ \text{\ml{T4}}} \\
\inferrule*[right={\!\!@}]{\text{\ml{y\hastype{} T2}} \\ \text{\ml{z\hastype{} T3}}}
{ \text{\ml{T5}}}
}{
\inferrule*[right={\!\!
\textnormal{\ml{\tlb{x}\tlb{y}\tlb{z}}}
}]
{\text{\ml{T6}}}
{
    \text{\ml{T7}}}
}
\]
In order to be sure that an expression (\ml{E$_1$ E$_2$}) of application form is well typed,
the function \ml{E$_1$} must have a functional type (\ml{A $\rightarrow$ B}), where \ml{E$_2$} is of type \ml{A},
and (\ml{E$_1$ E$_2$}) is of type \ml{B}. So whatever else is clear, the types of the subexpressions must be related by the following equations:
\begin{mlcoded}
    T0 = T1 $\rightarrow$ T4 \\
    T2 = T3 $\rightarrow$ T5 \\
    T4 = T5 $\rightarrow$ T6
\end{mlcoded}
Substituting back in the tree, we get
\[
\inferrule*[right={\!\!@}]{
    \inferrule*[right={\!\!@}]{\text{\ml{x\hastype{} T1 $\rightarrow$ T5 $\rightarrow$ T6}} \\ \text{\ml{z\hastype{} T1}}}
    { \text{\ml{T5 $\rightarrow$ T6}}} \\
    \inferrule*[right={\!\!@}]{\text{\ml{y\hastype{} T3 $\rightarrow$ T5}} \\ \text{\ml{z\hastype{} T3}}}
    { \text{\ml{T5}}}
}{
    \inferrule*[right={\!\!
            \textnormal{\ml{\tlb{x}\tlb{y}\tlb{z}}}
    }]
    {\text{\ml{T6}}}
    {
        \text{\ml{T7}}}
}
\]

Now what should we say about the abstraction? Certainly \ml{T7} will have the
form
\begin{mlcoded}
    (T1 $\rightarrow$ T5 $\rightarrow$ T6) $\rightarrow$ (T3 $\rightarrow$ T5) $\rightarrow \cdots $
\end{mlcoded}
but it is not immediately clear what to do about the two type labels \ml{T1} and \ml{T3}
for the two occurrences of the variable \ml{z}. It would be simple if we could see
some reason to say that the labels \ml{T1} and \ml{T3} must stand for the same type. For
then we could add two more equations to the set above, namely
\begin{mlcoded}
    T1 = T3 \\
    T7 = (T0 $\rightarrow$ T2 $\rightarrow$ T1 $\rightarrow$ T6)
\end{mlcoded}
and then on substituting back in the tree we would get
\[
\inferrule*[right={\!\!@}]{
    \inferrule*[right={\!\!@}]{\text{\ml{x\hastype{} T1 $\rightarrow$ T5 $\rightarrow$ T6}} \\ \text{\ml{z\hastype{} T1}}}
    { \text{\ml{T5 $\rightarrow$ T6}}} \\
    \inferrule*[right={\!\!@}]{\text{\ml{y\hastype{} T1 $\rightarrow$ T5}} \\ \text{\ml{z\hastype{} T1}}}
    { \text{\ml{T5}}}
}{
    \inferrule*[right={\!\!
            \textnormal{\ml{\tlb{x}\tlb{y}\tlb{z}}}
    }]
    {\text{\ml{T6}}}
    {
        \text{\ml{(T1 $\rightarrow$ T5 $\rightarrow$ T6) $\rightarrow$ (T1 $\rightarrow$ T5) $\rightarrow$ T1 $\rightarrow$ T6 }}}
}
\]
On the other hand, we have already seen in Section 8.2.3 expressions such as
\begin{mlcoded}
    I . code . I
\end{mlcoded}
which make perfect sense, but in which the two occurrences of the
composition function receive different types (to be sure, types sharing a
common form, but nonetheless different).

So it is not obvious that we should require all occurrences of a variable
bound by a lambda abstraction to have the same type. However, let us take
this requirement as an assumption, and explore its consequences using the
following example:
\begin{mlcoded}
    F = \tlb{f}\tlb{a}\tlb{b}\tlb{c}c (f a) (f b)
\end{mlcoded}
and laid out as a tree, the expression is:
\[
\inferrule*[right={\!\!@}]{
    \inferrule*[right={\!\!@}]{
        \text{\ml{c \hastype{} T2}}
        \\
        \inferrule*[right={\!\!@}]
        {\text{\ml{f \hastype{} T0}} \\ \text{\ml{a \hastype{} T1}}}
        {\text{\ml{T3}}}
    }
    { \text{\ml{T6}}}
    \\
    \inferrule*[right={\!\!@}]{\textnormal{\ml{f \hastype{} T4}} \\ \text{\ml{b \hastype{} T5}}}
    { \text{\ml{T7}}}
}{
    \inferrule*[right={\!\!\text{
            \text{\ml{\tlb{f}\tlb{a}\tlb{b}\tlb{c}}}
    }}]
    {\text{\ml{T8}}}
    {\text{\ml{T9}}}
   }
\]
From which we derive the equations:
\begin{mlcoded}
    T0 = T1 $\rightarrow$ T3 \\
    T2 = T3 $\rightarrow$ T6 \\
    T4 = T5 $\rightarrow$ T7 \\
    T6 = T7 $\rightarrow$ T8
\end{mlcoded}
If we now require that the different occurrences of \ml{f} have the same type, we
can add the equation \ml{T0 = T4} to the list above. But then we must also have
that \ml{T1 = T5} and \ml{T3 = T7}, which gives the tree:
\[
\inferrule*[right={\!\!@}]{
    \inferrule*[right={\!\!@}]{
        \text{\ml{c \hastype{} T3 $\rightarrow$ T3 $\rightarrow$ T8}}
        \\
        \inferrule*[right={\!\!@}]
        {\text{\ml{f \hastype{} T1 $\rightarrow$ T3}} \\ \text{\ml{a \hastype{} T1}}}
        {\text{\ml{T3}}}
    }
    { \text{\ml{T3 $\rightarrow$ T8}}}
    \\
    \inferrule*[right={\!\!@}]{\text{\ml{f \hastype{} T1 $\rightarrow$ T3}} \\ \text{\ml{b \hastype{} T1}}}
    { \text{\ml{T3}}}
}{
    \inferrule*[right={\!\!\textnormal{
            \ml{\tlb{f}\tlb{a}\tlb{b}\tlb{c}}
    }}]
    {\text{\ml{T8}}}
    {\text{\ml{(T1 $\rightarrow$ T3) $\rightarrow$ T1 $\rightarrow$ T1 $\rightarrow$ (T3 $\rightarrow$ T3 $\rightarrow$ T8) $\rightarrow$ T8}}}
}
\]

By demanding that both occurrences of \ml{f} should have the same type, we
have forced \ml{a} and \ml{b} to be of the same type. Renaming variables, the function \ml{F}
has type
\begin{mlcoded}
    (A $\rightarrow$ B) $\rightarrow$ A $\rightarrow$ A $\rightarrow$ (B $\rightarrow$ B $\rightarrow$ C) $\rightarrow$ C
\end{mlcoded}
according to our assumption.

It is not hard to think of contexts \ml{(F f a b)} which would make sense when \ml{a}
and \ml{b} are of different types. For example:
\begin{mlcoded}
    F I 0 'a'
\end{mlcoded}
seems to be the function which, when applied to a function \ml{c} of type
\ml{(num $\rightarrow$ char $\rightarrow$ A)}, returns the value \ml{(c 0 'a')}. On the other hand,
\begin{mlcoded}
    F code 0 'a' K
\end{mlcoded}
would certainly be an error, since it would result in the evaluation of \ml{(code 0)},
whereas the function \ml{code} is applicable only to characters. At last we can see
the point of the assumption. In order for an expression to be well typed, it is
not enough that it cannot `go wrong' when evaluated on its own, or in a
particularly favorable context. We have to make sure that it cannot `go wrong'
when plugged into \textit{any} well-typed context.

So we shall require that variables bound in a lambda abstraction receive the
same type at all their occurrences. Without `outside knowledge' of the
arguments to which an abstraction will be applied, we must assume the worst:
all occurrences of a variable bound by the same lambda abstraction must
share the same type.

To sum up, so far we have adopted the following rules:
\begin{numbered}
    \item The function part \ml{f} of an application \ml{(f a)} has a function type \ml{(A $\rightarrow$ V)},
    where \ml{A} is the type of the argument part \ml{a} and \ml{V} is the type of the
    application \ml{(f a)}.
    \item All occurrences of a \ml{\tl}-bound variable must have the same types.
\end{numbered}

Moreover, when solving a system of equations, we have used the following
rule:
\begin{mlcoded}
    \textnormal{If} (T1 $\rightarrow$ T2) = (T1$'$ $\rightarrow$ T2$'$), \textnormal{then} T1 = T1$'$ \textnormal{and} T2 = T2$'$
\end{mlcoded}
(This follows from a more general law which states that if two compound type
expressions are equal, then they must be formed with the same construction,
and their corresponding parts must be equal.)

\subsection{A Mistyping}

Consider the expression
\begin{mlcoded}
    \tlb{n}\tlb{a}\tlb{b}b n (n a b)
\end{mlcoded}
(This is sometimes used to define the successor function on natural numbers in
the type-free lambda calculus.) Written as a tree, the expression is:
\[
\inferrule*[right={\!\!@}]
{
    \inferrule*[right={\!\!@}]
    {\text{\ml{b \hastype{} T2}} \\ \text{\ml{n \hastype{} T3}}}
    { \text{\ml{T6}}}
    \\
     \inferrule*[right={\!\!@}]
     {
        \inferrule*[right={\!\!@}]
        {\text{\ml{n \hastype{} T0}} \\ \text{\ml{a \hastype{} T1}}}
        {\text{\ml{T4}}}
        \\
        \text{\ml{b \hastype{} T5}}
    }
    { \text{\ml{T7}}}
}
{
    \inferrule*[right={\!\!\textnormal{
            \ml{\tlb{n}\tlb{a}\tlb{b}}
    }}]
    {\text{\ml{T8}}}
    {\text{\ml{T9}}}
}
\]
From which we get the equations:
\begin{mlcoded}
    T0 = T1 $\rightarrow$ T4  \\
    T2 = T3 $\rightarrow$ T6
    T4 = T5 $\rightarrow$ T7 \\
    T6 = T7 $\rightarrow$ T8 \\
    T9 = T0 $\rightarrow$ T1 $\rightarrow$ T2 $\rightarrow$ T8 \\
    T3 = T0 \\
    T5 = T2
\end{mlcoded}
Eliminating \ml{T4} and \ml{T6}, these become
\begin{mlcoded}
    T0 = T1 $\rightarrow$ T5 $\rightarrow$ T7 \\
    T2 = T3 $\rightarrow$ T7 $\rightarrow$ T8 \\
    T9 = T0 $\rightarrow$ T1 $\rightarrow$ T2 $\rightarrow$ T8 \\
    T3 = T0 \\
    T5 = T2
\end{mlcoded}
Now note that these equations contain a circularity. If we try to use the last
two equations to eliminate \ml{T3} and \ml{T5}, we get

\vs
\begin{tabular}{llr}
    \ml{T0} & \ml{= T1 $\rightarrow$ T2 $\rightarrow$ T7} & (since \ml{T5 = T2}) \\
    & \ml{= T1 $\rightarrow$ (T3 $\rightarrow$ T7 $\rightarrow$ T8) $\rightarrow$ T7} & \\
    & \ml{= T1 $\rightarrow$ (T0 $\rightarrow$ T7 $\rightarrow$ T8) $\rightarrow$ T7} & \hspace{4cm} (since \ml{T3 = T0})
\end{tabular}
\vs

\noindent
So it is clear that the type \ml{T0} is not finite, and so neither is the type \ml{T9}.

Nevertheless, \ml{T9} possesses an infinite type, which may be expressed informally:
\begin{mlcoded}
    T0 $\rightarrow$ T1 $\rightarrow$ T2 $\rightarrow$ T8
\end{mlcoded}
where
\begin{mlcoded}
    T0 = T1 $\rightarrow$ (T0 $\rightarrow$ T7 $\rightarrow$ T8) $\rightarrow$ T7
\end{mlcoded}
There are many difficulties in dealing with infinite types. We shall simply
avoid them by imposing the rule:
\begin{quote}
    If \ml{T1 = \ldots T1\ldots}, where the type variable \ml{T1} occurs properly within the
    right-hand side of the equation, then the system of equations cannot be
    solved, and the expression from which the system was derived is ill-typed.
\end{quote}
As a consequence of this, the definition in Section 2.4.2 of the fixed-point
combinator \ml{Y} is ill-typed.

\subsection{Top-level lets}

Consider the expression
\begin{letalign}
    let &S = \tlb{x}\tlb{y}\tlb{z}x z (y z) \\
    &K = \tlb{x}\tlb{y}x \\
    in &S K K
\end{letalign}
It seems intuitively reasonable that we allow \ml{K} to take on different types at its
different occurrences in the body of the \ml{let}-expression. Indeed, it is hard to see
what polymorphism would mean if we insisted that variables introduced by a
\ml{let} definition should have the same type, as with variables bound by \ml{x}.

To examine the type structure of this expression, we need to extend the tree
notation to represent it:
\[
\inferrule*[right={\!\textnormal{\ml{let S,K.}}}]
{
    \inferrule{\text{\ml{Tree-S}} \\\\ \text{\ml{S \hastype{} TS}}}{}
    \\
    \inferrule{\text{\ml{Tree-K}} \\\\ \text{\ml{K \hastype{} TK}}}{}
    \\
    \inferrule*[right={\!\!@}]
    {
        \inferrule*[right={\!\!@}]
        {
            {\text{\ml{S \hastype{} T6 $\rightarrow$ T7 $\rightarrow$ T8}}}
            \\
            {\text{\ml{K \hastype{} T6}}}
        }
        {\text{\ml{T7 $\rightarrow$ T8}}}
        \\
        {\text{\ml{K \hastype{} T7}}}
    }
    {\text{\ml{T8}}}
}
{
   \text{\ml{T9}}
}
\]
Since we already know how to type-check the right-hand sides of the
definitions of \ml{S} and \ml{K}, we have merely indicated their type trees, to save
space. Moreover, we have skipped a few steps in representing the type
structure of \ml{(S K K)}. The equations for the type structure of the right-hand
sides of the definitions of \ml{S} and \ml{K} can be solved to yield:
\begin{mlcoded}
    TS = (T0 $\rightarrow$ T1 $\rightarrow$ T2) $\rightarrow$ (T0 $\rightarrow$ T1) $\rightarrow$ T0 $\rightarrow$ T2 \\
    TK = T3 $\rightarrow$ T4 $\rightarrow$ T3
\end{mlcoded}
The new constraints we have to consider are those relating \ml{T8} to \ml{T9}, and the
types \ml{TS} and \ml{TK} to the types of their occurrences in the body of the \ml{let}-
expression.

For the first constraint, plainly we should require that \ml{T8 = T9}. As for the
second, the constraint is that the type of the occurrence of \ml{S} should be an
instance of the type \ml{TS}, and the types of the two occurrences of \ml{K} should each
be an instance of the type \ml{TK}. But how should we represent such a requirement by means of an equation?

When working out the equations by hand, it is quite natural to proceed as
follows: \textit{refrain} from making any such representation at the outset. Instead,
obtain first a fully evaluated expression for the type of \ml{TS} and \ml{TK} (as we have
done). Then introduce new type labels for the instantiated variables at
each occurrence of \ml{S} and \ml{K} in the body of the \ml{let}-expression. (In this case,
there are three such variables in the type for \ml{S}, namely \ml{T0}, \ml{T1}, and \ml{T2}; and two
in the type of \ml{K}, namely \ml{T3} and \ml{T4}.) If we use a fresh set of variables for each
occurrence, then we can still work with equations, and leave the values of
those fresh variables to be worked out while we are exploring the type
structure of the body. So in this case we should add new variables \ml{T10}, \ml{T11},
and \ml{T12} to instantiate \ml{TS} at its first occurrence, \ml{T13} and \ml{T14} to instantiate \ml{TK} at the
first occurrence of \ml{K}, and \ml{T15} and \ml{T16} to instantiate \ml{TK} at the second
occurrence of \ml{K}. We then add the equations:
\begin{mlcoded}
    \begin{tabular}{llll}
        T6 $\rightarrow$ T7 $\rightarrow$ &T8 = & &(T10 $\rightarrow$ T11 $\rightarrow$ T12) \\
        & & $\rightarrow$ &(T10 $\rightarrow$ T11) $\rightarrow$ T10 $\rightarrow$ T12 \\
        & T6 = & T13 & $\rightarrow$ T14 $\rightarrow$ T13 \\
        & T7 = & T15 & $\rightarrow$ T16 $\rightarrow$ T15
    \end{tabular}
\end{mlcoded}

From the first of these we derive:
\begin{mlcoded}
    T6 = T10 $\rightarrow$ T11 $\rightarrow$ T12 \\
    T7 = T10 $\rightarrow$ T11 \\
    T8 = T10 $\rightarrow$ T12
\end{mlcoded}
reasoning that if \ml{(T1 $\rightarrow$ T2) = (T1' $\rightarrow$ T2')}, then \ml{T1 = T1'} and \ml{T2 = T2'}.
By the same reasoning, we have:
\begin{mlcoded}
    T10 = T13 \\
    T12 = T14 \\
    T10 = T15 \\
    T11 = T16 \\
    T15
\end{mlcoded}
which allows us to express the types of the two occurrences of \ml{K} as:
\begin{mlcoded}
    T6 = T10 $\rightarrow$ (T16 $\rightarrow$ T10) $\rightarrow$ T10 \\
    T7 = T10 $\rightarrow$ T16 $\rightarrow$ T10
\end{mlcoded}
and the type of the whole expression as:
\begin{mlcoded}
    T9 = T8 = T10 $\rightarrow$ T10
\end{mlcoded}

So the rule we adopt as the type-constraint for \ml{let}-expressions is that the
types of the occurrences of the defined names in the body must be instances of
the types of the corresponding right-hand sides. The procedure we adopt to
compute those instances is to instantiate the variables in the types of those
right-hand sides with new variables, making a fresh instance for each
occurrence of the defined name in the body of the \ml{let}. In fact, we shall not in
general be able to instantiate all the type variables, as we shall see shortly.

\subsection{Top-level \ml{letrecs}}

Turning now to \ml{letrecs}, it seems clear that a variable introduced by a \ml{letrec}
definition should be capable of taking on different types in the body of the
program governed by the \ml{letrec}, just as in the case of \ml{let}-definitions. So in
\begin{mlcoded}
    letrec f = (\ldots) \\
    in (\ldots f\ldots f\ldots f\ldots )
\end{mlcoded}
we expect \ml{f} to be capable of taking on different types throughout the
expression body. However, there is a new question we must answer. The
variable introduced by a recursive definition can also have many occurrences
in the right-hand side of its definition, as it were `while' it is being defined, as
well as `after'. In general, when there are several mutually recursive
definitions, as in
\begin{letalign}
    letrec &x$_1$ = (\ldots x$_1$\ldots x$_i$\ldots x$_k$\ldots) \\
    &x$_k$ = (\ldots x$_1$\ldots x$_j$\ldots x$_k$\ldots) \\
    in (\ldots &x$_1$\ldots x$_i$\ldots x$_j$\ldots x$_k$\ldots)
\end{letalign}
any one of the defined names \ml{x$_i$} can occur many times in many right-hand
sides, as well as in the body. Should we insist that all these occurrences have
the same type, in the sense of requiring equality to hold between the type
labels for the variable occurrences in the definitions? Or should we treat them
as we treat them in the body, and require only that at each such occurrence,
the type be an instance of the type of the corresponding right-hand side?
Unfortunately, in the nature of things, there is no obvious answer. Nevertheless,
to see what the question means, consider the example:
\begin{mlcoded}
    letrec Y = (\tlb{f}f (Y f)) in \ldots
\end{mlcoded}
Written out as a tree, the first definition is:
\[
\inferrule*[right={\!\textnormal{\ml{\tlb{f}}}}]
{
    \inferrule*[right={\!\!@}]
    {
        {\text{\ml{f \hastype{} T2}}}
        \\
        \inferrule*[right={\!\!@}]
        {
            {\text{\ml{Y \hastype{} T0}}}
            \\
            {\text{\ml{f \hastype{} T1}}}
        }
        {\text{\ml{T3}}}
    }
    {\text{\ml{T4}}}
}
{
    \text{\ml{Y \hastype{} T5}}
}
\]
The constraints we can write down straight away are these:
\begin{mlcoded}
    T1 = T2 \\
    T0 = T1 $\rightarrow$ T3 \\
    T2 = T3 $\rightarrow$ T4 \\
    T5 = T1 $\rightarrow$ T4
\end{mlcoded}
from which it follows that:
\begin{mlcoded}
    T0 = (T3 $\rightarrow$ T4) $\rightarrow$ T3
\end{mlcoded}
and
\begin{mlcoded}
    T5 = (T3 $\rightarrow$ T4) $\rightarrow$ T4
\end{mlcoded}
The question is, should we ask that \ml{T0 = T5}, or only that \ml{T0} be an instance
of \ml{T5}? In the former case, the only solution is \ml{T5 = ((T4 $\rightarrow$ T4) $\rightarrow$ T4)},
as we would expect of a fixed-point function. On the other hand,
the alternative requires only that \ml{T3} be an instance of \ml{T4}, so again
\ml{T5 = ((T4 $\rightarrow$ T4) $\rightarrow$ T4)} is a solution.

We shall adopt the (usual) approach according to which `during' such
definitions all occurrences of the defined variables must share the same type as
the right-hand side of their definitions. On the other hand, `after' the
definitions, the defined variables are polymorphic, and the type of such a
variable can be instantiated differently to satisfy the local constraints on
different occurrences of the variables in the body of the definition. If nothing
else, this approach has at least the merit of simplicity.

Some different approaches to the type-checking of recursive definitions
have been explored by Mycroft [1984]. In some (but not all) of these
approaches the problem of whether an expression is well typed becomes only
semi-decidable.

\subsection{Local Definitions}

We have presented type-checking as the search for the solution of a system of
constraints, represented by equations \ml{T$'$ = T} between type expressions. So
far, we know that when type-checking an expression of \ml{let} or \ml{letrec} form, we
should impose the constraint that the types of the occurrences of the defined
variables in the body should equal new instances of the types derived for their
right-hand sides. But just which type variables may be instantiated?

To understand this issue, we have to probe a little into the reason for our
conviction that a defined name can take on different types in the body of its
definition. The reason seems to be this:
\begin{quote}
An expression \ml{(let x = E in E$'$)} is well typed just in case the expression
\ml{E$'$[E/x]} is well typed, which is the expression obtained by substituting \ml{E} for
the free occurrences of \ml{x} in \ml{E$'$}.
\end{quote}
For each occurrence of \ml{x} in \ml{E$'$}, we should be able to instantiate the type
variables in the type tree for \ml{E} in such a way that it forms a subtree of the type
tree for \ml{E$'$[E/x]}. This instantiation is only possible if we do not thereby violate
the law that occurrences of a \ml{\tl}-bound variable must have the same type, or the
corresponding law for \ml{letrecs}.

Consider the expression \ml{(\tlb{x} let y = x in y y)}. By the principle above, this
is well typed just in case \ml{(\tlb{x}x x)} is well typed, which it blatantly is not. The
problem is that the type expression for \ml{y} contains (is!) a variable occurring in
the type of a more global \ml{\tl}-bound variable. We cannot instantiate that
variable differently at the different occurrences of \ml{y} in \ml{(y y)}.

Consider the partial expression:
\begin{mlcoded}
    \begin{tabular}{llll}
        \tlb{x} & & & \\
        &let &I &= \tlb{z}z \\
        & & prxl &= \tlb{c}(c x I) \\
        & & p1 &= \tlb{x}\tlb{y}x \\
        & & p2 &= \tlb{x}\tlb{y}y \\
        & in &\ldots
    \end{tabular}
\end{mlcoded}
Informally, the types of the defined names are:
\begin{letalign}
    I & \hastype{} A -> A \\
    prxl & \hastype{} (X -> (A -> A) -> B) -> B \\
    p1 & \hastype{} A -> B \\
    p2 & \hastype{} A -> B -> B
\end{letalign}
where \ml{A} and \ml{B} are schematic variables, and \ml{X} is the type of \ml{x}. If we take the
body of the \ml{let}-expression to be the expression
\begin{mlcoded}
    prxl p1 (prxl p1)
\end{mlcoded}
then it cannot be typed. For to satisfy the type constraints of this body, we
would have to instantiate \ml{X} differently at the different occurrences of \ml{prxl}. On
the other hand, if the body were
\begin{mlcoded}
    prxl p2 (prxl p2)
\end{mlcoded}
then the expression is well typed. For the structure of that expression does not
constrain \ml{X} to be instantiated differently at the different occurrences of \ml{prxl}.

When we are type-checking the body \ml{B} of a \ml{let} or \ml{letrec} definition, we must
therefore distinguish the type variables in the type derived for a defined name
according to whether they may or may not be differently instantiated at the
various occurrences of the name. Variables of the former kind are those that
do not occur in the type of any \textit{constrained} variable in the definition of the
name. A constrained variable is one which is a bound variable of a lambda
abstraction enclosing \ml{B}, or one defined in a \ml{letrec}-expression enclosing \ml{B} in
one of its right-hand sides.

This is one of the points at which the type regime of Miranda differs from
that of the type checker described here. The Miranda compiler requires that
all occurrences of a variable bound in a local definition share a single type.
This has the effect that local definitions cannot introduce new polymorphism
into a program. We will not explore the implications of this difference here—
the type-checking rules given in this and the following chapter are for a
standard implementation of the Milner type discipline.

We have used the notion of type trees to help elucidate the type structure of
expressions, and guide us towards a sharper view of the rules we use when
constructing and checking the types of expressions. In the next section, we
summarize those rules. With luck, the device will have served its purpose, and
we can then consider how to turn our intuitions into algorithms.

\section{Summary of Rules for Correct Typing}

The following rules are intended to describe the local `look' of the type
structure of a well-typed expression. To lighten the notational burden, we
shall sometimes simplify the expression whose type tree is depicted in the
figures. The simplifications are indicated in the commentary.

\subsection{Rule for Applications}

\plainbox{
\[
\inferrule*[right={\!\!@}]
{
    {\text{\ml{A $\rightarrow$ B}}}
    \\
    {\text{\ml{A}}}
}
{\text{\ml{B}}}
\]
}

\subsection{Rule for Lambda Abstractions}

\plainbox{
\[
    \inferrule*[vcenter, fraction={\cdot\cdots\cdot}]
    {
        \text{\ml{\ldots{} x \hastype{} A \ldots{} x \hastype{} A \ldots}}
    }
    {
         \inferrule*[fraction={---},right={\!\textnormal{\ml{\tlb{x}}}}]
        {
            \phantom{hi} \\\\
            \text{\ml{B}}
        }
        {
            \text{\ml{A $\rightarrow$ B}}
        }
%text
    }
\]
}
\noindent
Note that all occurrences of the variable \ml{x} bound by the abstraction must have
the same type.

\subsection{Rule for \ml{let}-expressions}

\plainbox{
    \[
\inferrule*[vcenter, right={\!\textnormal{\ml{let x.}}}]{
%\infer{
%    \[
    \inferrule*[vcenter, fraction={\cdot\cdots\cdot}]
    {
        \{ \text{\ml{\ldots{} y \hastype{} C \ldots{}}} \}
    }
    {
        \inferrule*[vcenter, fraction={\cdot\cdots\cdot}]
        {\phantom{text}}
        {\text{ \ml{x \hastype{} A} }}
    }
    \\
    \inferrule*[vcenter, fraction={\cdot\cdots\cdot}]
    {
        \text{\ml{ \ldots{} x \hastype{} A$'$ \ldots{} x \hastype{} A$''$ \ldots{}}}
    }
    {
       \text{\ml{B}}
    }
%    \]
}
{
    {
    \begin{tabular}{lcr}
        & \ml{B} \phantom{w} & \\
        $\{$\phantom{hacked} &  $\cdots$ \phantom{w} & \phantom{hack}$\}$ \\
        $\{$ &  ------\ml{\tlb{y}}  & $\}$ \\
        $\{$ &  $\cdots$ \phantom{w} & $\}$
    \end{tabular}
    }
}
\]
}

\noindent
Here we have shown only the case where just one definition is made in the
\ml{let}-expression: \ml{let x = E in E$'$}.

\textit{Restriction:} \ml{A$'$} and \ml{A$''$} are instances of \ml{A}. No variable may be instantiated
which occurs in the type of a variable bound in a more global lambda
abstraction or \ml{letrec}-expression (i.e., one further down the tree). The portions
of the figure in curly brackets indicate such a situation. Any type variables in \ml{A}
shared with \ml{C} may not be instantiated in forming \ml{A$'$} and \ml{A$''$}.

\subsection{Rule for \ml{letrec}-expressions}

\plainbox{
    \[
    \inferrule*[vcenter, right={\!\textnormal{\ml{letrec x.}}}]{
        %\infer{
            %    \[
            \inferrule*[vcenter, fraction={\cdot\cdots\cdot}]
            {
                \text{\ml{\ldots{} x \hastype{} A \ldots{} x \hastype{} A \ldots{}}}
            }
            {
                \inferrule*[vcenter, fraction={\cdot\cdots\cdot}]
                {\phantom{text}}
                {\text{ \ml{x \hastype{} A} }}
            }
            \\
            \inferrule*[vcenter, fraction={\cdot\cdots\cdot}]
            {
                \text{\ml{ \ldots{} x \hastype{} A$'$ \ldots{} x \hastype{} A$''$ \ldots{}}}
            }
            {
                \text{\ml{B}}
            }
            %    \]
        }
        {
            \text{\ml{B}}
        }
        \]
    }

\noindent
Here we have shown only the case where just one definition is made in the
\ml{letrec}-expression: \ml{letrec x = E in E$'$}. Note that the occurrences of \ml{x} within
the right-hand side of the definition must have the same type.

\textit{Restriction:} just as in the \ml{let} rule.

\section{Some Cautionary Remarks}

There is a beguiling similarity between the notion of type which we use in
mathematics, and the notion which we use in functional programming. It is all
too easy to transfer intuitions concerning the mathematical notion of type to
the notion used in programming. There are at least two important differences.

First, the types in a functional language are types of partial objects, whose
evaluation may not terminate. In contrast, the mathematical notion of type,
whose study began with Frege [Gaeck and Black, 1970] and Whitehead and
Russell [1910-1913], concerns total objects, whose definitions are well
founded. The purpose of the mathematical notion of type is to elucidate the
foundations of mathematics. The purpose of the notion in functional programming
is to assure us at compile-time that a program will not `go wrong',
where we do not count a program to have gone wrong if it does not terminate,
or a function is applied to arguments for which it has not been defined.

Second, in functional programming `recursion' is interpreted in a very
liberal sense, going far beyond recursion on well-founded structures, or
positive inductive definitions. As a direct result of this, the notion of a type in
functional programming cannot be the same notion that we use in
mathematics. For example, in a functional program we can define an integer
\ml{omega}, where
\begin{mlcoded}
    omega = omega + 1
\end{mlcoded}
and this cannot belong to the (mathematical) type of integers. Another
symptom of this liberal attitude to recursion is exhibited by the definition of
the algebraic type
\begin{mlcoded}
    D \typedecl{} LAMBDA (D $\rightarrow$ D)
\end{mlcoded}
in which the defined type occurs negatively (to the left of the arrow) on the
right-hand side of the definition. This is not to say that there is no mathematical
sense in the functional programming notions. On the contrary, there
is a rich and sophisticated theory (domain theory) which aims to give a
mathematical interpretation to just such constructs. But while constructing
that theory, and reasoning about the mathematical structures it involves, we
are using on the metalevel the ordinary mathematical notion of type.

We hope that this chapter has shown that a naive understanding of the
notion of type certainly gives us plenty to go on. We also hope to have
achieved another goal: that of showing that there are limits to the questions
we can settle on a naive basis alone.

\section*{References}

\begin{references}

\item Gaeck, P., and Black, M. (editors) 1970. Function and concept. In \textit{Translations
from the Philosophical Writings of Gottlob Frege}. Basil Blackwell.

\item Milner, R. 1978. A theory of type polymorphism in programming. \textit{Journal of
Computer and System Science}. Vol. 17, pp. 348-75.

\item Mycroft, A. 1984. Polymorphic type schemes and recursive definitions. In \textit{Proceedings
of the International Symposium on Programming, Toulouse}, pp. 217-39. LNCS
167. Springer Verlag.

\item Strachey, C. 1967. Fundamental concepts in programming languages. In \textit{Notes for the
International Summer School in Computer Programming, Copenhagen}.

\item Whitehead, A.N., and Russell, B.A.W. 1910-1913. \textit{Principia Mathematica}, 3
volumes. Cambridge University Press.

\end{references}