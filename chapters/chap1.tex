\chapter{Introduction}

This book is about implementing functional programming languages using
\textit{lazy graph reduction}, and it divides into three parts.

The first part describes how to translate a high-level functional language
into an intermediate language, called the lambda calculus, including detailed
coverage of pattern-matching and type-checking. The second part begins with
a simple implementation of the lambda calculus, based on graph reduction,
and then develops a number of refinements and alternatives, such as super-
combinators, full laziness and SK combinators. Finally, the third part
describes the G-machine, a sophisticated implementation of graph reduction,
which provides a dramatic increase in performance over the implementations
described earlier.

One of the agreed advantages of functional languages is their semantic
simplicity. This simplicity has considerable payoffs in the book. Over and
over again we are able to make semi-formal arguments for the correctness of
the compilation algorithms, and the whole book has a distinctly mathematical
flavor - an unusual feature in a book about implementations.

Most of the material to be presented has appeared in the published
literature in some form (though some has not), but mainly in the form of
conference proceedings and isolated papers. References to this work appear
at the end of each chapter.

\section{Assumptions}
This book is about implementations, not languages, so we shall make no
attempt to extol the virtues of functional languages or the functional
programming style. Instead we shall assume that the reader is familiar with
functional programming; those without this familiarity may find it heavy
going. A brief introduction to functional programming may be found in
Darlington [1984], while Henderson [1980] and Glaser et al. [1984] give more
substantial treatments. Another useful text is Abelson and Sussman [1985]
which describes Scheme, an almost-functional dialect of Lisp.

An encouraging consensus seems to be emerging in the basic features of
high-level functional programming languages, exemplified by languages such
as SASL [Turner, 1976], ML [Gordon et al., 1979], KRC [Turner, 1982],
Hope [Burstall et al., 1980], Ponder [Fairbairn, 1985], LML [Augustsson,
1984], Miranda [Turner, 1985] and Orwell [Wadler, 1985]. However, for the
sake of definiteness, we use the language Miranda as a concrete example
throughout the book (When used as the name of a programming language,
`Miranda' is a trademark of Research Software Limited.) A brief intro-
duction to Miranda may be found in the appendix, but no serious attempt is
made to give a tutorial about functional programming in general, or Miranda
in particular. For those familiar with functional programming, however, no
difficulties should arise.

Generally speaking, all the material of the book should apply to the other
functional languages mentioned, with only syntactic changes. The only
exception to this is that we concern ourselves almost exclusively with the
implementation of languages with \textit{non-strict semantics} (such as SASL, KRC,
Ponder, LML, Miranda and Orwell). The advantages and disadvantages of
this are discussed in Chapter 11, but it seems that graph reduction is probably
less attractive than the environment-based approach for the implementation
of languages with strict semantics; hence the focus on non-strict languages.
However, some functional languages are strict (ML and Hope, for example),
and while much of the book is still relevant to strict languages, some of the
material would need to be interpreted with care.

Generally speaking, all the material of the book should apply to the other
functional languages mentioned, with only syntactic changes. The only
exception to this is that we concern ourselves almost exclusively with the
implementation of languages with \textit{non-strict semantics} (such as SASL, KRC,
Ponder, LML, Miranda and Orwell). The advantages and disadvantages of
this are discussed in Chapter 11, but it seems that graph reduction is probably
less attractive than the environment-based approach for the implementation
of languages with strict semantics; hence the focus on non-strict languages.
However, some functional languages are strict (ML and Hope, for example),
and while much of the book is still relevant to strict languages, some of the
material would need to be interpreted with care.

The emphasis throughout is on an informal approach, aimed at developing
understanding rather than at formal rigor. It would be an interesting task to
rewrite the book in a formal way, giving watertight proofs of correctness at
each stage.

\section{Part I: Compiling High-level Functional Languages}
It has been widely observed that most functional languages are quite similar to
each other, and differ more in their syntax than their semantics. In order to
simplify our thinking about implementations, the first part of this book shows
how to translate a high-level functional program into an \textit{intermediate language}
which has a very simple syntax and semantics. Then, in the second and third
parts of the book, we will show how to implement this intermediate language
using graph reduction. Proceeding in this way allows us to describe graph
reduction in considerable detail, but in a way that is not specific to any
particular high-level language.

The intermediate language into which we will translate the high-level
functional program is the notation of the \textit{lambda calculus} (Figure~1.1). The
lambda calculus is an extremely well-studied language, and we give an introduction
to it in Chapter 2.


\boxedfigure{
\centering

\framebox{\
    \begin{minipage}{6cm}
        \centering

        High-level language program\
    \end{minipage}
}\\
\hspace{4cm}$\Bigg\downarrow$ \quad
\begin{minipage}{4cm}
    {Part I}
\end{minipage}\\
\framebox{\
    \begin{minipage}{6cm}
        \centering

        Program expressed in lambda notation
    \end{minipage}
    \ }\\
\hspace{4cm}$\Bigg\downarrow$  \quad
\begin{minipage}{4cm}
    {Parts II and III}
\end{minipage}\\
\framebox{\
    \begin{minipage}{6cm}
        \centering

        Concrete implementation
    \end{minipage}
    \ }\\

}{Implementing a functional program}



The lambda calculus is not only simple, it is also sufficiently expressive to
allow us to translate any high-level functional language into it. However,
translating some high-level language constructs into the lambda notation is
less straightforward than it at first appears, and the rest of Part I is concerned
with this translation.

Part I is organized as follows. First of all, in Chapter 3, we define a language
which is a superset of the lambda calculus, which we call the enriched lambda
calculus. The extra constructs provided by the enriched lambda calculus are
specifically designed to allow a straightforward translation of a Miranda
program into an expression in the enriched lambda calculus, and Chapter 3
shows how to perform this translation for simple Miranda programs.

After a brief introduction to pattern-matching, Chapter 4 then extends the
translation algorithm to cover more complex Miranda programs, and gives a
formal semantics for pattern-matching. Subsequently, Chapter 7 rounds out
the picture, by showing how Miranda's ZF expressions can also be translated
in the same way. (Various advanced features of Miranda are not covered,
such as algebraic types with laws, abstract data types, and modules.)

Much of the rest of Part I concerns the transformation of enriched lambda
calculus expressions into the ordinary lambda calculus subset, a process which
is quite independent of Miranda. This language-independence was one of the
reasons for defining the enriched lambda calculus language in the first place.
Chapter 5 shows how expressions involving pattern-matching constructs may
be transformed to use case-expressions, with a considerable gain in efficiency.
Then Chapter 6 shows how all the constructs of the enriched lambda calculus,
including case-expressions, may be transformed into the ordinary lambda
calculus.

Part I concludes with Chapter 8 which discusses type-checking in general,
and Chapter 9 in which a type-checker is constructed in Miranda.

\section{Part II: Graph Reduction}

The rest of the book describes how the lambda calculus may be implemented
using a technique called \textit{graph reduction}. It is largely independent of the later
chapters in Part I, Chapters 2--4 being the essential prerequisites.

As a foretaste of things to come, we offer the following brief introduction to
graph reduction. Suppose that the function \ml{f} is defined (in Miranda) like this:
\begin{mlcoded}
f\, x = (x + 1) $*$ (x -- 1)
\end{mlcoded}
This definition specifies that \ml{f} is a function of a single argument \ml{x}, which
computes `\ml{(x + 1) $*$ (x -- 1)}'. Now suppose that we are required to evaluate
\begin{mlcoded}
f\, 4
\end{mlcoded}
that is, the function \ml{f} applied to \ml{4}. We can think of the program like this:
\begin{center}
    \begin{forest}
        [\ml{@}
            [\ml{f}]
            [\ml{4}]
        ]
    \end{forest}
\end{center}
where the \ml{@} stands for function application. Applying \ml{f} to \ml{4} gives
\begin{center}
    \begin{forest}
        [\ml{$*$}
            [$+$
                [\ml{4}]
                [\ml{1}]
            ]
            [$-$
                [\ml{4}]
                [\ml{1}]
            ]
        ]
    \end{forest}
\end{center}
(Note: in the main text we will use a slightly different representation for
applications of \ml{$*$}, \ml{$+$} and \ml{$-$}, but this fact is not significant here.) We may now
execute the addition and the subtraction (in either order), giving
\begin{center}
    \begin{forest}
    [\ml{$*$}
        [\ml{5}]
        [\ml{3}]
    ]
\end{forest}
\end{center}
Finally we can execute the multiplication, to give the result
\begin{center}
\ml{15}
\end{center}

From this simple example we can see that:
\begin{numbered}
\item \textit{Executing} a functional program consists of \textit{evaluating} an expression.
\item A functional program has a natural representation as a \textit{tree} (or, more
generally, a \textit{graph}).
\item Evaluation proceeds by means of a sequence of simple steps, called
\textit{reductions}. Each reduction performs a local transformation of the graph
(hence the term \textit{graph reduction}).
\item Reductions may safely take place in a variety of orders, or indeed in
parallel, since they cannot interfere with each other.
\item Evaluation is complete when there are no further reducible expressions.
\end{numbered}
Graph reduction gives an appealingly simple and elegant model for the
execution of a functional program, and one that is radically different from the
execution model of a conventional imperative language.

We begin in Chapter 10 by discussing the representation of a functional
program as a graph. The next two chapters form a pair which discusses first the
question of deciding which reduction to perform next (Chapter 11), and then
the act of performing the reduction (Chapter 12).

Chapters 13 and 14 introduce the powerful technique of \textit{supercombinators},
which is the key to the remainder of the book. This is followed in Chapter 15
with a discussion of \textit{full laziness}, an aspect of lazy evaluation; this chapter can
be omitted on first reading since later material does not depend on it.

Chapter 16 then presents \textit{SK combinators}, an alternative implementation
technique to supercombinators. Hence, this chapter can be understood
independently of Chapters 13--15. Thereafter, however, we concentrate on
supercombinator-based implementations.

Part II concludes with a chapter on \textit{garbage collection}.

\section{Part Ill: Advanced Graph Reduction}

It may seem at first that graph reduction is inherently less efficient than more
conventional execution models, at least for conventional von Neumann
machines. The bulk of Part III is devoted to an extended discussion of the
G-machine, which shows how graph reduction can be compiled to a form that
is suitable for \textit{direct execution} by ordinary sequential computers.

In view of the radical difference between graph reduction on the one hand,
and the linear sequence of instructions executed by conventional machines on
the other, this may seem a somewhat surprising achievement. This (fairly
recent) development is responsible for a dramatic improvement in the speed
of functional language implementations.

Chapters 18 and 19 introduce the main concepts of the G-machine, while
Chapters 20 and 21 are devoted entirely to optimizations of the approach.
The book concludes with three chapters that fill in some gaps, and offer
some pointers to the future.

Chapter 22 introduces \textit{strictness analysis}, a compile-time program analysis
method which has been the subject of much recent work, and which is crucial
to many of the optimizations of the G-machine.

Perhaps the major shortcoming of functional programming languages,
from the point of view of the programmer, is the difficulty of estimating the
space and time complexity of the program. This question is intimately bound
up with the implementation, and we discuss the matter in Chapter 23.
graph reduction.

Finally, the book concludes with a chapter on parallel implementations of
graph reduction.

\pg

\section*{References}

\begin{references}
	\item Abelson, H., and Sussman, G.J. 1985. \textit{Structure and Interpretation of Computer
	Programs}. MIT Press.
	\item  Augustsson, L. 1984. A compiler for lazy ML. \textit{Proceedings of the ACM Symposium on
	Lisp and Functional Programming, Austin}. August, pp. 218--27.
	\item Burstall, R.M., MacQueen, D.B., and Sanella, D.T. 1980. \textit{Hope: an experimental
	applicative language}. CSR-62--80. Department of Computer Science, University of
	Edinburgh. May.
	\item  Darlington, J. 1984. Functional programming. In \textit{Distributed Computing}. Duce
(Editor). Academic Press.
	\item  Fairbairn, J. 1985. Design and implementation of a simple typed language based on the
lambda calculus. PhD thesis, \textit{Technical Report 75}. University of Cambridge. May.
	\item  Glaser, H., Hankin, C., and Till, D. 1984. \textit{Principles of Functional Programming}.
Prentice-Hall.
	\item  Gordon, M.J., Milner, A.J., and Wadsworth, C.P. 1979. \textit{Edinburgh LCF}. LNCS 78.
Springer Verlag.
	\item  Henderson, P. 1980. \textit{Functional Programming}. Prentice-Hall.
	\item  Turner, D.A. 1976. \textit{The SASL language manual}. University of St Andrews. December.
	\item  Turner, D.A. 1982. Recursion equations as a programming language. In \textit{Functional
Programming and Its Applications}, Darlington et al. (editors), pp. 1-28. Cam-
bridge University Press.
	\item  Turner, D.A. 1985. Miranda -- a non-strict functional language with polymorphic
types. In \textit{Conference on Functional Programming Languages and Computer
Architecture, Nancy}, pp. 1-16. Jouannaud (editor), LNCS 201. Springer Verlag.
	\item  Wadler, P. 1985. \textit{Introduction to Orwell}. Programming Research Group, University of
Oxford.
\end{references}
