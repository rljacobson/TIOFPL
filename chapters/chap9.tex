\chapter{A Type Checker}
\chapterauthors{Peter Hancock}
\vspace{3cm}

\noindent
In this chapter, we will construct a type-checker in Miranda, taking the rules
developed in the previous chapter as the basis for the type discipline.

Sections 9.1 and 9.2 show how the expressions of the intermediate language
and its type expressions can be represented as Miranda data types. Sections
9.3 to 9.6 are concerned with the basic mechanisms of the type-checker, which
is itself defined in Section 9.7.

\section{Representation of Programs}

Since we propose to write a type-checker in Miranda, we will have to
represent the program to be type-checked as a Miranda data structure, which
is passed as an argument to the type-checking function.

The program to be checked will be represented by an object of the
structured type \ml{vexp}, defined below. Each line of the type definition is derived
directly from the corresponding construct in the concrete syntax.
\begin{mlalign}
    vname == &[char] \\
    vexp \typedecl{} &VAR vname \\
     &| LAMBDA vname vexp \\
     &| AP vexp vexp \\
     &| LET [vname] [vexp] vexp \\
     &| LETREC [vname] [vexp] vexp
\end{mlalign}
\noindent
In a sense, this type encompasses slightly too much. We shall suppose that the
program is not `trivially' malformed: in a \ml{LET} or \ml{LETREC} construct, the list of
variables must have the same length as the list of right-hand sides; the
variable list in a \ml{LET} or \ml{LETREC} construct must not be empty, and should
contain no repetitions. Moreover, the free variables in an expression must be
among those associated with predeclared types, either because they are
supplied by the system or because their types can be deduced from type
definitions in the program. We can assure ourselves that these restrictions are
met in a simple recursive scan through the program.

To understand the representation, let us take for an example the following
trivial program:
\begin{letalign}
    let &S = \tlb{x}\tlb{y}\tlb{z}x z (y z) \\
    &K = \tlb{x}\tlb{y}x \\
    in &S K K
\end{letalign}
Considered as an object in the type \ml{vexp}, the program becomes:
\begin{mlcoded}
LET ["S", "K"] [rhs\_S, rhs\_K] main\\
where
\end{mlcoded}
\vspace{-\baselineskip}
\begin{mlcoded}
\begin{tabular}{rll}
\phantom{XX}  &var\_S &= VAR "S" \\
  &var\_K &= VAR "K" \\
&var\_x &= VAR "x" \\
&var\_y &= VAR "y" \\
&var\_z &= VAR "z" \\
&main &= AP (AP var\_S var\_K) var\_K \\
&rhs\_S &= plambda ["x", "y", "z"] body\_S \\
&rhs\_K &= plambda ["x", "y"] body\_K \\
&body\_S &= AP (AP var\_x var\_z) (AP var\_y var\_z) \\
&body\_K &= var\_x
\end{tabular}\\
plambda vs e = foldr LAMBDA e vs
\end{mlcoded}
\noindent
which the reader may write out without using `\ml{where}' if so inclined.

\section{Representation of Type Expressions}

To construct the type-checker, we will need to represent type expressions by
Miranda data structures. We need a type for the names of type variables and,
for the moment, we will take this to be the type of lists of characters. (For
technical convenience, we will revise this definition in Section 9.6.)
\begin{letalign}
    tvname &== [char] \\
    type\_exp  &\typedecl{}  TVAR tvname \\
     & \qquad | TCONS [char] [type\_exp]
\end{letalign}
This definition says that a type expression must be either a type variable or a
compound type (such as \ml{(A $\rightarrow$ B)}, \ml{[A]} or \ml{(A,B)}). We represent such
compound types by the name of the operator (e.g. ``\ml{arrow}'' for \ml{(A $\rightarrow$ B)},
``\ml{cross}'' for \ml{(A,B)}), and a list of the operands.

Whatever other type-forming operators we have, we will certainly need the
function type operator. So let us define:
\begin{mlcoded}
    arrow \hastype{} type\_exp $\rightarrow$ type\_exp $\rightarrow$ type\_exp \\
    arrow t1 t2 = TCONS "arrow" [t1,t2]
\end{mlcoded}
If \ml{t1} and \ml{t2} are of type \ml{type\_exp}, and we know what types they represent, then
\ml{(arrow t1 t2)} will represent the type of functions from \ml{t1} to \ml{t2}. Using Miranda's
dollar notation for infixes, we may write this in the form \ml{(t1 \$arrow t2)}, which
adheres more closely to the informal notation.

The other type-forming operations we have mentioned could be
represented in a similar way:
\begin{mlcoded}
\begin{tabular}{llll}
int &\hastype{} &type\_exp &\\
int & & \phantom{t2} = TCONS &"int" [] \\
&&&\\
cross &\hastype{} &type\_exp $\rightarrow$ &type\_exp $\rightarrow$ type\_exp \\
cross &t1 &t2 = TCONS &"cross" [t1,t2] \\
&&&\\
list & &type\_exp $\rightarrow$ &type\_exp \\
list t & & \phantom{t2} = TCONS &"list" [t]
\end{tabular}
\end{mlcoded}

The function \ml{tvars\_in} returns a list of the variable names that occur in a type
expression. (The list may contain repetitions.)
\begin{mlcoded}
    tvars\_in \hastype{} type\_exp $\rightarrow$ [tvname] \\
    tvars\_in t = tvars\_in$'$ t [\,] \\
    \qquad where \\
    \qquad \phantom{XX} tvars\_in$'$ (TVAR x) I = x : I \\
    \qquad \phantom{XX} tvars\_in$'$ (TCONS y ts) I = foldr tvars\_in$'$ I ts
\end{mlcoded}

\section{Success and Failure}

Since type-checking is something that can succeed or fail, we have to choose a
mechanism for representing success and failure within Miranda.

We shall use the type \ml{(reply $\ast$)} for the type of the values of a function which
may succeed (returning an object of type \ml{$\ast$}) or fail (returning no indication as
to why).
\begin{mlcoded}
    reply $\ast{}$ \typedecl{} OK $\ast{}$ | FAILURE
\end{mlcoded}
It would not be acceptable for a practical type-checker to return no indication
as to why a check has failed. One might then use a slightly more complicated
operator, such as
\begin{mlcoded}
    reply' $\ast{}$ $\ast\ast$ \typedecl{} OK$'$ $\ast$ | FAILURE$'$ $\ast\ast$
\end{mlcoded}
which is capable of returning error information. It is notoriously difficult to
write error-handling code without obscuring the code to handle correct cases,
so we will use instead the simpler, less informative operator. Any error
detected while type-checking will be propagated up to the top level without
further examination of the program. Here, too, there may be grounds for
complaint, which we counter with the same excuse.

(There is more than one way to represent success and failure. An
alternative approach to the one taken here is described by Wadler [1985].)

\section{Solving Equations}

Consider type-checking an application \ml{(AP e1 e2)}, where we have worked out
the type \ml{t1} for \ml{e1} and the type \ml{t2} for \ml{e2}. To do this, we try to `solve the equation'
\begin{mlcoded}
    t1 = t2 $\rightarrow$ (TVAR n)
\end{mlcoded}
where \ml{n} is a type variable name that has not been used before. As we have
seen, the structure of an expression gives rise to a system of such equations.

How should we represent solutions of systems of type equations? In
mathematics, the solution of simultaneous equations
\vspace{-0.25\baselineskip}
\begin{align*}
a_{1,1} \times x_1 + a_{1,2} \times x_2 &= b_1\\
a_{2,1} \times x_1 + a_{2,2} \times x_2 &= b_2
\end{align*}
%\vspace{-0.5\baselineskip}
is expressed by giving values for each of the unknowns $x_1$ and $x_2$ which satisfy
the equations. Analogously, an alleged solution of a system of type equations
can be expressed as a function from type variables (the unknowns) to type
expressions (their values). The allegation is that the equations are satisfied
when we replace (i.e. substitute) the unknowns by their values under the
function. We therefore take
\begin{mlcoded}
    subst == tvname $\rightarrow$ type\_exp
\end{mlcoded}
to be the type of substitutions. We shall see how to determine whether a set of
equations between type expressions has a solution, and if so how to construct
a substitution that satisfies them. We shall use identifiers such as \ml{phi}, \ml{phi$'$}, and \ml{psi},
as variables over substitutions.

\subsection{Substitutions}

Given a substitution function \ml{phi} and a type expression \ml{te}, we define
\ml{(sub\_type phi te)} to be the type expression obtained by performing the \ml{phi}
substitution on all the type variables in \ml{te}:
\begin{letalign}
    sub\_type \hastype{} subst $\rightarrow$ type\_exp& $\rightarrow$ type\_exp \\
    sub\_type phi (TVAR tvn) &= phi tvn \\
    sub\_type phi (TCONS tcn ts) &= TCONS tcn (map (sub\_type phi) ts)
\end{letalign}
Here \ml{map} is the function that applies a function to each entry in a list:
\begin{letalign}
    map \hastype{} ($\ast$ $\rightarrow$& $\ast\ast$) $\rightarrow$ [$\ast$] $\rightarrow$ [$\ast\ast$] \\
    map f [\,] &= [\,] \\
    map f (x:xs) &= f x : map f xs
\end{letalign}
Two substitutions can be composed to give a further substitution:
\begin{mlcoded}
    scomp \hastype{} subst $\rightarrow$ subst $\rightarrow$ subst \\
    scomp sub2 sub1 tvn = sub\_type sub2 (sub1 tvn)
\end{mlcoded}
The crucial property of \ml{scomp} is that
\begin{mlcoded}
sub\_type (scomp phi psi) = (sub\_type phi) . (sub\_type psi)
\end{mlcoded}
(Remember that function composition is represented by an infix dot.)

The identity substitution \ml{id\_subst} has the property that
\begin{mlcoded}
sub\_type id\_subst\ t = t
\end{mlcoded}
for all \ml{t\hastype{}type\_exp}. It can be defined by:
\begin{mlcoded}
    id\_subst \hastype{} subst \\
    id\_subst tvn = TVAR tvn
\end{mlcoded}

A \textit{delta substitution} is one that affects one variable only. We define:
\begin{letalign}
    delta \hastype{} tvname& $\rightarrow$ type\_exp $\rightarrow$ subst \\
    delta tvn t tvn$'$ &= t,\qquad \qquad\qquad \qquad tvn = tvn$'$ \\
    \phantom{delta tvn t} &= TVAR tvn$'$
\end{letalign}
Hence, \ml{(sub\_type (delta tvn t))} is the function that maps a type expression to
one that contains \ml{t} where before it had \ml{(TVAR tvn)}.

In fact, all the substitutions we need will be built up from the identity
substitution \ml{id\_subst} by composition on the left with substitutions of \ml{delta}.

In general, a substitution may associate a variable with a value which itself
contains variables. If those variables in turn are given values different from
themselves, then the substitution is not 'fully worked out'. When we work out
a set of equations
\[
x_1 = t_1;\; \ldots;\; x_k = t_k
\]
by substituting $t_i$ for $x_i$ at all of its occurrences in $t_1, \ldots, t_k$, we may have to
iterate the substitution many times before the equations stabilize to their final
forms. (Of course, this iterative process does not terminate if there is a
circularity in the equations.) In general, we are interested in obtaining `fully
worked out' substitutions, which do not have to be re-applied. The next
definition is intended to capture what we mean by such a substitution.

A substitution \ml{phi} is \textit{idempotent} if
\begin{mlcoded}
(sub\_type phi) . (sub\_type phi) = sub\_type phi
\end{mlcoded}
or equivalently, if \ml{(phi \$scomp phi) = phi}. In other words, if you apply the
substitution twice, you get nothing different the second time. A type
expression \ml{t} is a \textit{fixed point} of a substitution \ml{phi} if
\begin{mlcoded}
sub\_type phi\ t = t
\end{mlcoded}
In particular, if \ml{(TVAR x)} is a fixed point of \ml{phi}, then we say that \ml{x} is unmoved by \ml{phi}.

Note that if \ml{phi} is idempotent, and \ml{phi} moves \ml{tvn}, then
\begin{mlcoded}
sub\_type phi (VAR tvn)
\end{mlcoded}
is a fixed point of \ml{phi}, and hence cannot contain \ml{tvn}.

\subsection{Unification}
In this section, we will show how to construct a substitution which solves a
given set of type equations, using a process called \textit{unification}.

A system of type equations can be represented by a list of pairs of type
expressions, where each pair \ml{(t$_1$, t$_2$)} represents the equation
\begin{mlcoded}
t$_1$ = t$_2$
\end{mlcoded}
To solve the equations, we have to find a substitution \ml{phi} which \textit{unifies} the left-
and right-hand sides of all equations in the system, where \ml{phi} unifies the pair
\ml{(t$_1$, t$_2$)} if
\begin{mlcoded}
    sub\_type phi t$_1$ = sub\_type phi t$_2$
\end{mlcoded}
If this equation holds, \ml{phi} is said to be a \textit{unifier} of \ml{t$_1$} and \ml{t$_2$}. If \ml{phi} is a unifier of
each pair in the list representing a set of equations, we may think of \ml{phi} as
a simultaneous solution of the equations.

If the substitution \ml{phi} solves a system of equations, then clearly any
substitution \ml{psi$'$} of the form \ml{(psi \$scomp phi)} is also a solution, but \ml{phi} will
usually be a more general solution than \ml{psi$'$}. A substitution \ml{phi} is \textit{no less
general} than a substitution \ml{psi} if there is a substitution \ml{rho} such that
\begin{mlcoded}
    psi = rho \$scomp phi
\end{mlcoded}
If such an equation holds, then \ml{psi} is said to be an \textit{extension} of \ml{phi}.

If we have constructed a solution \ml{phi} of a system of type equations, and we
have done no more than is necessary to satisfy the equations, we will have a
solution which is \textit{maximally general}, in the sense that it is no less general than
any other solution.

For an example (in informal terms), consider the type expressions
\begin{mlcoded}
    T1 = (A $\rightarrow$ B) $\rightarrow$ C\\
    T2 = (B $\rightarrow$ A) $\rightarrow$ (A $\rightarrow$ B)
\end{mlcoded}
The substitutions \ml{phi1} and \ml{phi2}, where
\begin{mlcoded}
    phi1 A = B, phi1 C = (B $\rightarrow$ B)\\
    phi2 B = A, phi2 C = (A $\rightarrow$ A)
\end{mlcoded}
are both unifiers of \ml{T1} and \ml{T2}. In fact, they are examples of maximally general
unifiers: they each do (one version of) the minimum necessary to make \ml{T1} and
\ml{T2} equal, so that any other unifier of \ml{T1} and \ml{T2} is an extension of each of them.

The problem of unification is to find a maximally general idempotent unifier
of a set of pairs of expressions. The method we use is Robinson's [1965]
unification algorithm. It is convenient when coding the algorithm to concentrate
on the problem of \textit{extending} a given substitution, which solves a set of
equations
\[
t_1 = t_1';\; \ldots ;\; t_k = t_k'
\]
to one that solves an extended set
\[
t_1 = t_1';\; \ldots;\; t_k = t_k';\; t_{k+1} = t_{k+1}'
\]
So we shall pose the problem in the following way. Given a pair $(t_1, t_2)$ of type
expressions, and an idempotent substitution \ml{phi}, our algorithm should return
\ml{FAILURE} if there is no extension of \ml{phi} which unifies $(t_1, t_2)$, and it should return
\ml{(OK psi)}, where \ml{psi} is an idempotent unifier of $(t_1, t_2)$ which extends \ml{phi}. (In
fact, the one we construct will be maximally general among such extensions of
\ml{phi}.)

The simplest equation we can consider is one of the form
\begin{mlcoded}
TVAR tvn = t
\end{mlcoded}
To handle such cases in the unification algorithm, we will make use of the
following function:
\begin{mlcoded}
\begin{tabular}{lll}
extend \hastype{} subst $\rightarrow$ &tvname $\rightarrow$ type\_exp{}\,& $\rightarrow$ reply subst \\
extend phi tvn t &= OK phi, & t = TVAR tvn \\
&= FAILURE, &tvn \$in tvars\_in t \\
&= OK ((delta tvn t) &\$scomp phi)
\end{tabular}
\end{mlcoded}
An expression \ml{(extend phi tvn t)} will be evaluated only when:
\begin{numbered}
    \item \ml{phi} is an idempotent substitution (the solution we are trying to extend);
    \item \ml{t} is a fixed point of \ml{phi};
    \item \ml{tvn} is unmoved by \ml{phi} (\ml{tvn} does not already have a value under \ml{phi}).
\end{numbered}

The value of the expression is either \ml{FAILURE}, or of the form \ml{(OK phi$'$)},
where \ml{phi$'$} is an idempotent substitution extending \ml{phi}, such that
\begin{mlcoded}
\begin{tabular}{lll}
    sub\_type phi$'$ t$'$ &= t  & {\normalfont{if}} t$'$ = TVAR tvn\\
    &= sub\_type phi t$'$\phantom{hack} & {\normalfont{otherwise}}
\end{tabular}
\end{mlcoded}
In fact, \ml{phi$'$} is maximally general among extensions of \ml{phi} which solve the equation:
\begin{mlcoded}
    TVAR tvn = t
\end{mlcoded}
Note that if \ml{phi} is idempotent, \ml{t} is a fixed point of \ml{phi}, and \ml{tvn} \textit{is} moved by \ml{phi},
then \ml{tvn} can occur in neither \ml{(phi tvn)} nor \ml{t}.

We can code the unification algorithm as follows:
\begin{mlcoded}
    unify \hastype{} subst $\rightarrow$ (type\_exp, type\_exp) $\rightarrow$ reply subst \\
    \\
    unify phi ((TVAR tvn),t) \\
    \phantom{XX}= extend phi tvn phit, \hspace{2cm} phitvn = TVAR tvn \\
    \phantom{XX}= unify phi (phitvn, phit)\\
    \phantom{XX= }where \\
    \phantom{XXXX= }phitvn\ = phi tvn \\
    \phantom{XXXX= }phit\phantom{tvn} = sub\_type phi t \\
    \\
    unify phi ((TCONS tcn ts),(TVAR tvn)) \\
    \phantom{XX}= unify phi ((TVAR tvn),(TCONS tcn ts)) \\
    \\
    unify phi ((TCONS tcn ts),(TCONS tcn$'$ ts$'$)) \\
    \phantom{XX}= unifyl phi (ts \$zip ts$'$),  \hspace{1.75cm} tcn = tcn$'$ \\
    \phantom{XX}= FAILURE
\end{mlcoded}
The function \ml{zip}, which is generally useful, turns a pair of lists into a list of
pairs, whose length is the same as that of the shorter of the lists:
\begin{letalign}
    zip \hastype{} [$\ast$] $\rightarrow$ [$\ast\ast$]& $\rightarrow$ [($\ast$,$\ast\ast$)] \\
    zip [\,] xs &= [\,] \\
    zip (x:xs) [\,] &= [\,] \\
    zip (x:xs) (y:ys) &= (x,y):zip xs ys
\end{letalign}
The function \ml{unifyl} is defined such that \ml{(unifyl phi pts)} constructs a substitution
extending \ml{phi} which unifies corresponding entries in the list of pairs \ml{pts}. This
function is also generally useful, so it is defined globally too.
\begin{mlcoded}
    unifyl \hastype{} subst $\rightarrow$ [(type\_exp,type\_exp)] $\rightarrow$ reply subst
\end{mlcoded}
\begin{letalign}
    unifyl phi eqns = &foldr unify$'$ (OK phi) eqns \\
&where \\
&\phantom{XXXX}unify$'$ eqn (OK phi) = unify phi eqn \\
&\phantom{XXXX}unify$'$ eqn FAILURE = FAILURE
\end{letalign}
This completes the definition of the unification algorithm.

It is important to see why the unification algorithm terminates. After all, in
the definition above we have defined the value of \ml{(unify (TVAR tvn) t)} in terms
of \ml{(unify phitvn phit)} where \ml{phitvn = (phi tvn)} and \ml{phit = (sub\_type phi t)},
which may be very much larger expressions than \ml{(TVAR tvn)} and \ml{t}. However,
we only use that clause of the definition in circumstances when \ml{tvn} cannot
occur in \ml{phitvn} or \ml{phit}. Define the \textit{solution set} of \ml{phi} to be the set of variables
which occur in an expression \ml{(phi tvn$'$)}, where \ml{tvn$'$} is moved by \ml{phi}. We can
prove that \ml{(unify phi (t1,t2))} terminates, by a nested induction: the outer
induction is on the number of variables in \ml{t1} and \ml{t2} which are not in the
solution set of \ml{phi}, and the inner induction is on the combined length of \ml{t1} and
\ml{t2}.

The unification algorithm has many applications other than type-checking.
In particular, it is a key algorithm in the implementation of programming
languages such as Prolog.

\section{Keeping Track of Types}

When type-checking an expression with free variables, there are two ways to
proceed.

\subsection{Method 1: Look to the Occurrences}

We can find the constraints imposed on the types of the free variables by the
manner in which they occur in the expression. In a complete program, the free
variables must stand for the system's built-in functions or functions associated
with type definitions. We would then look to see whether the types deduced
for \textit{each occurrence} of a free variable can be instances of the type supplied a
priori for that variable. When type-checking a lambda abstraction \ml{(\tlb{x}E)}, we
would check that the types deduced for the various occurrences of \ml{x} within \ml{E}
can be unified to the same type expression, and we would handle occurrences
of defined variables in the right-hand sides of a \ml{letrec}-expression in the same
way.

It is quite possible to develop a type-checker along these lines: one is
presented in Damas [1985].

\subsection{Method 2: Look to the Variables}

It is technically rather a nuisance that distinct \textit{occurrences} of the same variable
in an expression are associated with different type expressions. Is there
something which we can associate with each \textit{variable} instead?

Suppose we wish to type-check a \ml{let}-expression. First of all, we type-check
the definitions of the \ml{let}, thus deducing a type for each variable defined by the
\ml{let}. Then it seems that we could associate each variable with its type, and
proceed to type-check the body of the \ml{let}-expression. At each occurrence of
one of these defined variables in the body, we should construct an \textit{instance} of
its associated type, substituting fresh type variables for the \textit{schematic}
variables in the type (see Section 8.5.3). However, as we discovered in
Section 8.5.5, some of the variables in the type are \textit{constrained} and should not
be substituted for, and the instantiation mechanism must take account of this.

What is needed, therefore, is to associate with each variable a kind of type
\textit{template}, in which the schematic variables are distinguished from the non-
schematic variables. Then the template can be instantiated by copying it,
substituting a fresh type variable for each occurrence of a schematic variable
(but copying non-schematic variables unchanged). This type template is
called a \textit{type scheme}. To summarize:
\begin{numbered}
    \item The \textit{schematic} type variables in a type scheme associated with a variable
    are those that may be freely instantiated to conform with the type
    constraints on the various occurrences of that variable.
    \item All the other (non-schematic) variables in a type scheme are constrained, and must not be instantiated when instantiating the type scheme. As we
    remarked in Section 9.4, they behave in a similar way to the unknowns of
    a mathematical equation. For example, consider the simultaneous
    equations
    \[
    a_{1,1} \times x_1 + a_{1,2} \times x_2 = b_1
    \]
    \[
    a_{2,1} \times x_1 + a_{2,2} \times x_2 = b_2
    \]
    We seek values for the unknowns $x_1$ and $x_2$, by solving the equations, but
    they must be consistently instantiated, so that $x_1$ stands for the same value
    wherever it occurs (and likewise $x_2$).\\
    \phantom{XX}By analogy, we will refer to the non-schematic variables of a type
    scheme as \textit{unknowns}. They are the type variables whose values we seek by solving the system of type constraints implied by the structure of an
    expression.
\end{numbered}



(In papers about type-checking, schematic variables are often called \textit{generic}
variables, and unknowns are called \textit{non-generic}. We mention this only to
make it easier to link up with the literature, and will not use that terminology
here.)

There is a partial analogy between type schemes and lambda abstractions.
The schematic variables of a type scheme correspond to the formal parameter
of a lambda abstraction, and the unknowns of a type scheme correspond to
the free variables of a lambda abstraction. Applying a lambda abstraction to
an argument involves constructing an instance of its body, substituting the
argument for occurrences of the formal parameter (but copying free variables
unchanged). This is very similar to the process of instantiating a type scheme,
which involves constructing an instance of the type scheme template,
substituting fresh type variables for occurrences of the schematic variables
(but copying unknowns unchanged).

We will represent type schemes by objects of the type
\begin{mlcoded}
    type\_scheme \typedecl{} SCHEME [tvname] type\_exp
\end{mlcoded}
A type variable occurring in a type scheme (\ml{SCHEME scvs e}) is schematic if
its name occurs in the list \ml{scvs}, otherwise it is an unknown.

\begin{mlcoded}
    unknowns\_scheme \hastype{} type\_scheme $\rightarrow$ [tvname]\\
    unknowns\_scheme (SCHEME scvs t) = tvars\_in t \$bar scvs
\end{mlcoded}
where
\begin{mlcoded}
    bar \hastype{} [$\ast$] $\rightarrow$ [$\ast$] $\rightarrow$ [$\ast$]\\
    bar xs ys = [ x $\leftarrow$ xs | $\sim$ (x \$in ys) ]\\
    in \hastype{} $\ast \rightarrow$ [$\ast$] $\rightarrow$ bool\\
    in x$'$ [\;] \quad \  = False\\
    in x$'$ (x:xs)  = True, \qquad x = x$'$\\
    \phantom{XXXXXXX\;\,\ }= x$'$ \$in xs
\end{mlcoded}


During the course of type-checking we will have occasion to apply a
substitution to a type scheme, to reflect additional information we have on its
unknowns. When doing this, we should take care that only the unknowns are
affected (remember that the schematic variables function like the formal
parameter of a lambda abstraction, and have only local significance):
\begin{mlcoded}
    sub\_scheme \hastype{} subst $\rightarrow$ type\_scheme $\rightarrow$ type\_scheme \\
    sub\_scheme phi (SCHEME scvs t)\\
    \phantom{XXXX} = SCHEME scvs (sub\_type (exclude phi scvs) t)
\end{mlcoded}
where
\begin{mlalign}
    exclude phi scvs tvn &= TVAR tvn, \quad tvn \$in scvs\\
    &= phi tvn
\end{mlalign}

In Section 2.2.6 we demonstrated the irritating problem of `name-capture',
whereby a free variable of a lambda abstraction could become bound by being
substituted inside another lambda abstraction. There is a similar problem
here with substitution into type schemes. We must take care that the
expression
\begin{mlcoded}
    sub\_scheme phi (SCHEME scvs t)
\end{mlcoded}
is only evaluated when the schematic variables \ml{scvs} are distinct from any
variables occurring in the result of applying the substitution \ml{phi} to any of the
unknowns of \ml{t}. Otherwise a type variable in the range of the substitution
(which is always an unknown) might surreptitiously be changed into a
schematic variable. The way in which we ensure this is to guarantee that
the names of the schematic type variables in the type scheme are always
distinct from those which can occur in the range of the substitution (which are
always unknowns).

\subsection{Association Lists}

Having decided to associate a type scheme with each free variable in an
expression, rather than a type expression with each occurrence of a free
variable, we now have to decide how this information should be provided to
the type-checker. There are two requirements on the data structure we use:
\begin{numbered}
    \item It should provide a mapping from the free variables of the expression to
    type schemes.
    \item We should be able to determine the \textit{range} of that mapping.
\end{numbered}
To understand the second point, consider type-checking (\ml{let x = E in E$'$}). We
start by deriving a type \ml{t} for \ml{E}, in a type environment
\begin{mlcoded}
x$_1$ \hastype{} ts$_1$, \; \ldots, \; x$_k$ \hastype{} ts$_k$
\end{mlcoded}
which associates a type scheme \ml{t$_i$} with each variable \ml{x$_i$} free in \ml{E}
(the \ml{t$_i$} thus constitute the range of the type environment). In other words, we attempt to
build a solution \ml{phi} to the type equations implied by the structure of \ml{E}, such
that
\begin{mlcoded}
    E \hastype{} t \quad {\normalfont provided that} \quad x$_1$ \hastype{} ts$_1'$, \ldots, x$_k$ \hastype{} ts$_k'$
\end{mlcoded}
where \ml{ts$_i$$'$} is the image of \ml{ts$_i$} under the substitution \ml{phi}. We then form the type
scheme \ml{ts} to be associated with \ml{x} when type-checking \ml{E$'$}, in the extended
environment
\begin{mlcoded}
    x$_1$ \hastype{} ts$_1'$, \ldots, x$_k$ \hastype{} ts$_k'$, x \hastype{} ts
\end{mlcoded}
The schematic variables of \ml{ts} are all of the type variables of \ml{t} except those that
are unknown (non-schematic) in any of the schemes \ml{ts$_1'$}, \ldots, \ml{ts$_k'$}. So
whatever data structure we choose to represent the environment of the
type-checker, it should give us ready access to the set of unknowns in its range
(the \ml{ts$_i'$}).

An association list provides us with a suitable data structure.
\begin{mlcoded}
    assoc\_list $\ast$ $\ast\ast$ == [($\ast$, $\ast\ast$)]
\end{mlcoded}
Here $\ast$ stands for the type of keys, and $\ast\ast$ for the type of associated values. A
key \ml{k} is associated with a value \ml{v} by means of the pair \ml{(k,v)}. The partial
function itself is represented by a list of such associations. We shall use \ml{al}, \ml{al$'$},
etc. as variables over association lists.
\begin{mlcoded}
    dom \hastype{} assoc\_list $\ast$ $\ast\ast$ $\rightarrow$ [$\ast$]\\
    dom al = [ k | (k,v) $\leftarrow$ al]
\end{mlcoded}
(\ml{dom al}) returns a list (possibly with duplications) of the keys associated with
values in the list, which is how we shall represent the domain of a partial
function.
\begin{mlcoded}
    val \hastype{} assoc\_list $\ast$ $\ast\ast$ $\rightarrow$ $\ast$ $\rightarrow$ $\ast\ast$ \\
    val al k = hd [ v | (k$'$,v) $\leftarrow$ al ; k = k$'$ ]
\end{mlcoded}
If \ml{k} is a key in (\ml{dom al}), then (\ml{val al k}) returns the first value in the list which is
associated with \ml{k}. When using this function, we should be careful to ensure
that the second argument belongs to the domain of the association list.
\begin{mlcoded}
    install al k v = (k,v):al
\end{mlcoded}
(\ml{install al k v}) returns an association list which implements the same partial
function as \ml{al}, except that the key \ml{k} is now mapped to the value \ml{v}.
\begin{mlcoded}
    rng \hastype{} assoc\_list $\ast$ $\ast\ast$ $\rightarrow$ [$\ast\ast$]\\
    rng al = map (val al) (dom al)
\end{mlcoded}
The property which \ml{rng} is intended to satisfy is that every entry in (\ml{rng al}) is a
value of (\ml{val al}).

We shall represent the information passed to the type-checker about the
types of the free variables of an expression by means of an object of the
following type:
\begin{mlcoded}
    type\_env == assoc\_list vname type\_scheme
\end{mlcoded}
We shall use \ml{gamma}, \ml{gamma$'$}, etc. as variables standing for type environ-
ments. The functions \ml{unknowns\_scheme} and \ml{sub\_scheme} can be extended to
act on type environments, in the obvious way:
\begin{mlcoded}
    unknowns\_te \hastype{} type\_env $\rightarrow$ [tvname] \\
    unknowns\_te gamma = appendlist (map unknowns\_scheme (rng gamma))\\
\\
    appendlist \hastype{} [\;[$\ast$]\;] $\rightarrow$ [$\ast$] \\
    appendlist lls = foldr (++) [\,] lls\\
\\
    sub\_te \hastype{} subst $\rightarrow$ type\_env $\rightarrow$ type\_env \\
    sub\_te phi gamma \\
    \phantom{XXX} = [ (x, sub\_scheme phi st) | (x,st) $<-$ gamma ]
\end{mlcoded}

\section{New Variables}

When type-checking a closed expression, we first assigned a distinct type
variable to each subexpression, and then wrote down equations expressing
the constraints on those variables imposed by the structure of the expression.
When type-checking an expression containing variables defined in a \ml{let}- or
\ml{letrec}-expression, we chose first to work out the schematic types of those
variables (i.e.\ we checked the definitions first). We then assigned to each
occurrence of such a variable a type expression obtained by substituting new
unknown variables for the schematic variables, using a distinct set of
unknowns for each distinct occurrence.

So we will need a mechanism that enables us to `make up' new type
variables, and guarantees that they are distinct from type variables we may
introduce in the future. There are many ways to provide such a mechanism.
The one we adopt here is to postulate that there is a type \ml{name\_supply}, and
functions
\begin{mlcoded}
    \begin{tabular}{lll}
    next\_name &\hastype{} name\_supply $\rightarrow$ &tvname \\
    deplete   &\hastype{} name\_supply $\rightarrow$ &name\_supply \\
    split     &\hastype{} name\_supply $\rightarrow$ &(name\_supply, name\_supply)
    \end{tabular}
\end{mlcoded}
such that if \ml{ns} is a name supply, then (\ml{next\_name ns}) is distinct from any name
supplied by (\ml{deplete ns}), and if (\ml{ns0,ns1}) = \ml{split ns}, then any name supplied
from \ml{ns0} is distinct from any name supplied by \ml{ns1}. One way to implement
such a type is to (re)define \ml{tvname}, thus:
\begin{mlcoded}
    \begin{tabular}{ll}
        tvname       &== [num] \\
        name\_supply  &== tvname \\
        next\_name ns &= ns \\
        deplete (n:ns) &= (n+2 : ns) \\
        split ns       &= (0:ns, 1:ns)
    \end{tabular}
\end{mlcoded}
For example, if we start with the name supply \ml{[0]}, then the names it will supply
are \ml{[0]}, \ml{[2]}, \ml{[4]}, \ldots, while the names supplied by splitting the supply into
\ml{[0,0]} and \ml{[1,0]} will be \ml{[0,0]}, \ml{[2,0]}, \ml{[4,0]}, \ldots, and \ml{[1,0]}, \ml{[3,0]}, \ml{[5,0]}, \ldots,
respectively. (The $+2$ in the definition of \ml{deplete} is only an artifice to ensure that the
two halves of a split name supply are forever distinct.)

The function \ml{name\_sequence} returns from a name supply an infinite
sequence of distinct names derived from that supply:
\begin{mlcoded}
    name\_sequence \hastype{} name\_supply $\rightarrow$ [tvname] \\
    name\_sequence ns = next\_name ns : name\_sequence (deplete ns)
\end{mlcoded}

In practice, it is probably better to adopt an approach other than the supply
of new variables, according to which variables are named by integers, and the
name supply represented by the name of the next variable to be allocated. The
type-checker would then take the name supply as an argument, and return the
depleted supply as part of its value. We have adopted an approach which
wastes large portions of the variable name space, in order not to encumber the
type-checker code with a further avoidable detail.

\section{The Type-checker}

Finally, we are in a position to define the type-checker. This will take the form
of a function \ml{(tc gamma ns e)} where
\begin{numbered}
    \item \ml{gamma} is a type environment, associating type schemes with each of the
    free variables of \ml{e}. When the type-checker is invoked upon a complete
    program, this type environment should be initialized to contain
    declarations of the types of the built-in system-supplied identifiers.
    \item \ml{ns} is a supply of type variable names.
    \item \ml{e} is the expression to be checked.
\end{numbered}
The value returned will be a reply, which in the case of success will return a
pair of the form \ml{(phi,t)} where
\begin{numbered}
    \item \ml{phi} is a substitution defined on the unknown type variables in \ml{gamma}.
    \item \ml{t} is a type derived for the expression \ml{e}, in the type environment
    \ml{(sub\_te phi gamma)}. It will in fact be a fixed point of the substitution \ml{phi}.
\end{numbered}
In other words, if
\begin{mlcoded}
    tc gamma ns e = OK (phi,t)
\end{mlcoded}
then \ml{e \hastype{} t} can be derived from \ml{gamma}, provided that each unknown \ml{tvn} in
\ml{gamma} has the value given it by \ml{phi}.

We shall define the function \ml{tc} by induction on the structure of the
expression, with a different clause for each form which an expression can
take:
\begin{mlcoded}
    \phantom{n}tc \hastype{} type\_env $\rightarrow$ name\_supply $\rightarrow$ vexp $\rightarrow$ reply (subst, type\_exp) \\
    \begin{tabular}{l l l}
        tc gamma ns (VAR x)        & = tcvar    &gamma ns x \\
        tc gamma ns (AP e1 e2)     & = tcap     &gamma ns e1 e2 \\
        tc gamma ns (LAMBDA x e)   & = tclambda &gamma ns x e \\
        tc gamma ns (LET xs es e)  & = tclet    &gamma ns xs es e \\
        tc gamma ns (LETREC xs es e) & = tcletrec &gamma ns xs es e
    \end{tabular}
\end{mlcoded}
We will describe each of these cases in a separate section, beginning at Section
9.7.2. First, however, we define a useful auxiliary function \ml{tcl}.

\subsection{Type-checking Lists of Expressions}

It is convenient to define a function \ml{(tcl es gamma n)} which applies to a list of
expressions \ml{es}, and will return in the case of success a similar result \ml{OK (phi,ts)},
where \ml{ts} is a list of types derived for corresponding components of the list \ml{es} in
the type environment \ml{(sub\_te phi gamma)}. \ml{phi} embodies all the constraints
on \ml{gamma} necessary to derive those types simultaneously. The function is
defined from \ml{tc} by the equations:
\begin{mlcoded}
    \phantom{n}tcl \hastype{} type\_env $\rightarrow$ name\_supply $\rightarrow$ [vexp] $\rightarrow$ reply (subst, [type\_exp]) \\
    \begin{tabular}{lll}
        tcl gamma ns [\,]  &= &OK (id\_subst, [\,]) \\
        tcl gamma ns (e:es) &= &tcl1 gamma ns0 es (tc gamma ns1 e) \\
        & &where (ns0,ns1) = split ns
    \end{tabular}
\end{mlcoded}

\begin{mlcoded}
    \begin{tabular}{lll}
    tcl1 gamma ns es FAILURE      &= &FAILURE \\
    tcl1 gamma ns es (OK (phi,t)) &= &tcl2 phi t (tcl gamma$'$ ns es) \\
    & &where gamma$'$ = sub\_te phi gamma
    \end{tabular}
\end{mlcoded}

\begin{mlcoded}
    \begin{tabular}{ll}
        tcl2 phi t FAILURE          & = FAILURE \\
        tcl2 phi t (OK (psi,ts))    & = OK (psi \$scomp phi, (sub\_type psi t) : ts)
    \end{tabular}
\end{mlcoded}
The substitution can be thought of as built up in two stages. In the first stage,
we type-check each entry in the list, in the type environment `seen' through
the substitutions derived for previous entries. Then in the second stage, we
form the substitution by cumulative composition, and ensure that each type
returned for an expression is a fixed point of the composite substitution.

\subsection{Type-checking Variables}

When type-checking a variable \ml{x} in a given type environment \ml{gamma}, with
name supply \ml{ns}, we look up the type scheme associated with that variable by
\ml{gamma}. Recall that in a type scheme, a type variable is \textit{either schematic}, in
which case we substitute a fresh type variable for it, or \textit{unknown}, in which case
we leave it as it is.

So we return a new instance of the schematic type associated with the
variable, in which the schematic variables have been replaced by fresh type
variables. In this way, the type constraints on different occurrences of a
variable \ml{x} can be resolved independently, as indicated by the schematic
variables in the type scheme associated with \ml{x}.
\begin{mlalign}
    tcvar \hastype{} type\_env &$\rightarrow$ name\_supply $\rightarrow$ vname \\
    &$\rightarrow$ reply (subst, type\_exp)
\end{mlalign}
\begin{mlcoded}
    tcvar gamma ns x \\
    \phantom{XXXX}= OK (id\_subst, newinstance ns scheme) \\
    \phantom{XXXXXX}where scheme = val gamma x
\end{mlcoded}
where
\begin{mlcoded}
    newinstance \hastype{} name\_supply $\rightarrow$ type\_scheme $\rightarrow$ type\_exp \\
    newinstance ns (SCHEME scvs t) \\
    \phantom{XXXX}= sub\_type phi t \\
    \phantom{XXXXXX}where al  = scvs \$zip (name\_sequence ns) \\
    \phantom{XXXXXXXXXXX}phi = al\_to\_subst al
\end{mlcoded}

Here we have built an association list between the schematic variables and
an initial segment of the name sequence built on the given name supply. Such
an association list can be made into a substitution, by means of the function:
\begin{mlcoded}
    \phantom{n}al\_to\_subst \hastype{} assoc\_list tvname tvname $\rightarrow$ subst \\
    \begin{tabular}{ll}
    al\_to\_subst al tvn &= TVAR (val al tvn),  tvn \$in dom al \\
    &= TVAR tvn
    \end{tabular}
\end{mlcoded}

\subsection{Type-checking Application}

When type-checking an expression \ml{(AP e1 e2)} with respect to a type
environment \ml{gamma}, we first of all try to construct a substitution \ml{phi} which
solves the type constraints on \ml{e1} and \ml{e2} together. Suppose that the types
\ml{t1} and \ml{t2} are derived for \ml{e1} and \ml{e2}. We then try to construct an extension of \ml{phi}
which satisfies the additional constraint
\begin{mlcoded}
    t1 = t2 $\rightarrow$ t$'$
\end{mlcoded}
where \ml{t$'$} is a new type variable. We obtain this extension, as usual, by unifying
\ml{t1} with \ml{t2 $\rightarrow$ t$'$}.
\begin{mlcoded}
    \begin{tabular}{ll}
    tcap \hastype{} type\_env &$\rightarrow$ name\_supply $\rightarrow$ vexp $\rightarrow$ vexp \\
        &$\rightarrow$ reply (subst, type\_exp)
    \end{tabular} \\
    \phantom{n}tcap gamma ns e1 e2 \\
    \begin{tabular}{rl}
         \phantom{XXX} = tcap1 tvn &(tcl gamma ns$'$ [e1,e2]) \\
        where tvn &= next\_name ns \\
        ns$'$ &= deplete ns
    \end{tabular}
\end{mlcoded}
\begin{mlalign}
    \phantom{n}tcap1 &tvn FAILURE \\
    &= FAILURE \\
    \phantom{n}tcap1 &tvn (OK (phi,[t1,t2])) \\
    &= tcap2 tvn (unify phi (t1, t2 \$arrow (TVAR tvn)))
\end{mlalign}
\begin{mlcoded}
    \begin{tabular}{ll}
tcap2 tvn FAILURE &= FAILURE \\
tcap2 tvn (OK phi) &= OK (phi, phi tvn)
    \end{tabular}
\end{mlcoded}

\subsection{Type-checking Lambda Abstractions}

When type-checking \ml{(LAMBDA x e)}, we know nothing at the outset about the
type of \ml{x}. So we associate \ml{x} with a scheme of the form
\ml{(SCHEME [ ] (TVAR tvn))}, where \ml{tvn} is a new type variable. Because this
scheme has no schematic type variables, the various occurrences of the
variable will be assigned the value of the same type variable. This is the formal
counterpart of our decision to insist that all occurrences of the same \ml{LAMBDA}-bound variable should have the same type.

\begin{mlcoded}
    \begin{tabular}{l@{\hspace{8pt}}l}
        tclambda \hastype{} type\_env &$\rightarrow$ name\_supply $\rightarrow$ vname $\rightarrow$ vexp \\
        &$\rightarrow$ reply (subst, type\_exp)
    \end{tabular} \\
    \begin{tabular}{r@{\hspace{8pt}}l}
        tclambda gamma ns x e& \\
        = tclambda1 tvn &(tc gamma$'$ ns$'$ e) \\
        where ns$'$ &= deplete ns \\
        gamma$'$ &= new\_bvar (x,tvn)\; :\; gamma \\
        tvn \hspace{.6cm} &= next\_name ns
    \end{tabular}
\end{mlcoded}

\begin{mlcoded}
    tclambda1 tvn FAILURE \\
    \hspace*{1.5cm} = FAILURE \\
    tclambda1 tvn (OK (phi,t)) \\
    \hspace*{1.5cm} = OK (phi, (phi tvn) \$arrow t)
\end{mlcoded}

\begin{mlcoded}
    new\_bvar (x, tvn) = (x, SCHEME [ ] (TVAR tvn))
\end{mlcoded}

\subsection{Type-checking \ml{let}-expressions}

When type-checking an expression \ml{(LET xs es e)}, we first of all type-check
the right-hand sides in the list \ml{es}. We then have to update the environment so
that it associates the appropriate schematic types with the names in the list \ml{xs},
and type-check the body \ml{e}. The details of constructing the `appropriate'
schematic types are slightly involved, so we shall hide them in the definition of
a function \ml{add\_decls}.

\begin{mlcoded}
    \begin{tabular}{l@{\hspace{6pt}}l}
        tclet \hastype{} &type\_env $\rightarrow$ name\_supply \\
        &$\rightarrow$ [vname] $\rightarrow$ [vexp] $\rightarrow$ vexp \\
        &$\rightarrow$ reply (subst, type\_exp) \\
    \end{tabular}\\
    \begin{tabular}{r@{\hspace{4pt}}l}
        tclet &gamma ns xs es e \\
        = &tclet1 gamma ns0 xs e (tcl gamma ns1 es) \\
        \vspace{.2cm}
        &where (ns0, ns1) = split ns\\
        tclet1 &gamma ns xs e FAILURE \\
        = &FAILURE \\
        tclet1 &gamma ns xs e (OK (phi, ts)) \\
        = &tclet2 phi (tc gamma$''$ ns1 e) \\
    \end{tabular}\\
    \begin{tabular}{l@{\hspace{4pt}}l@{\hspace{4pt}}l}
        \hspace*{1cm}where &gamma$''$ &= add\_decls gamma$'$ ns0 xs ts \\
        &gamma$'$  &= sub\_te phi gamma \\
    \vspace{.2cm}
        &(ns0, ns1) &= split ns
    \end{tabular}\\
    \begin{tabular}{r@{\hspace{4pt}}l}
        tclet2 &phi FAILURE \\
        = &FAILURE \\
        tclet2 &phi (OK (phi$'$,t)) \\
        = &OK (phi$'$ \$scomp phi, t)
    \end{tabular}
\end{mlcoded}
The purpose of \ml{add\_decls} is to update a type environment \ml{gamma} so that it
associates schematic types formed from the types \ml{ts} with the variables \ml{xs}. The
variables which become schematic variables are those that are not unknowns
in \ml{gamma}. The definition is slightly complicated by our obligation to ensure
that the names of the schematic variables are distinct from the names of any
unknown variables which can occur in the range of a substitution. We use the
name sequence \ml{ns} to supply new names for the schematic variables.

\begin{mlcoded}
    \begin{tabular}{l@{\hspace{6pt}}l}
        add\_decls \hastype{} &type\_env $\rightarrow$ name\_supply \\
        &$\rightarrow$ [vname] $\rightarrow$ [type\_exp] $\rightarrow$ type\_env
    \end{tabular}\\
    \begin{tabular}{r@{\hspace{4pt}}l}
        add\_decls &gamma ns xs ts \\
        = &(xs \$zip schemes) ++ gamma \\
        &where schemes \hspace{.25cm}= map (genbar unknowns ns) ts \\
        &\hspace{1.25cm}unknowns = unknowns\_te gamma
    \end{tabular}
\end{mlcoded}

\begin{mlcoded}
    \begin{tabular}{r@{\hspace{4pt}}l}
        genbar &unknowns ns t \\
        &= SCHEME (map snd al) t$'$ \\
    \end{tabular}\\
    \begin{tabular}{r@{\hspace{6pt}}l@{\hspace{6pt}}l}
        \hspace*{1.6cm}where &al &= scvs \$zip (name\_sequence ns) \\
        &scvs &= (nodups (tvars\_in t)) \$bar unknowns \\
        &t$'$   &= sub\_type (al\_to\_subst al) t
    \end{tabular}
\end{mlcoded}
Here \ml{snd} is a function which projects a pair to its second coordinate. The
projection functions for pairs are defined by

\begin{mlcoded}
        fst \hastype{} ($\ast$,\,$\ast\ast$) $\rightarrow$ $\ast$ \\
        \vspace{.2cm}fst (x, y) = x\\
        snd \hastype{} ($\ast$,\,$\ast\ast$) $\rightarrow$ $\ast\ast$ \\
        snd (x, y) = y
\end{mlcoded}

The function \ml{nodups} returns a list with the same set of entries as its
argument list, but without duplicates:

\begin{mlcoded}
    \begin{tabular}{l@{\hspace{6pt}}l@{\hspace{6pt}}l}
        nodups \hastype{}&[\,$\ast$\,] $\rightarrow$ [\,$\ast$\,] &\\
        nodups xs &= f [ ] xs &\\
        &where &\\
        &\hspace{0.5cm}f acc [ ] &= acc \\
        &\hspace{0.5cm}f acc (x:xs) &= f acc xs, \ x \$in acc \\
        &\hspace{0.5cm}f acc (x:xs) &= f (x:acc) xs
    \end{tabular}
\end{mlcoded}

\subsection{Type-checking letrec-expressions}

The definition of the function invoked to type-check expressions
\ml{(LETREC xs es e)} is rather intricate, as there are many things to do. In
outline, they are these:

\begin{numbered}
    \item Associate new type schemes with the variables \ml{xs}. These schemes will
    have no schematic variables, in accordance with our decision to insist
    that all occurrences of a defined name in the right-hand sides of a
    recursive definition should have the same type.
    \item Type-check the right-hand sides. If successful, this will yield a
    substitution and a list of types which may be derived for the right-hand
    sides if the type environment is constrained by the substitution.
    \item Unify the types derived for the right-hand sides with the types associated
    with the corresponding variables, in the context of that substitution. This
    is in accordance with our decision that the right-hand sides of recursive
    definitions must receive the same types as occurrences of the corresponding variables. Should the unification succeed, that constraint can be
    met.
    \item We are now in much the same situation as we were in with expressions of
    \ml{LET} form, when the definitions had been processed, and it remained to
    type-check the body \ml{e}, after updating the type environment with
    appropriate schematic types.
\end{numbered}

\begin{mlcoded}
    \begin{tabular}{@{}r@{\hspace{6pt}}l}
        tcletrec \hastype{} &type\_env $\rightarrow$ name\_supply \\
        &$\rightarrow$ [vname] $\rightarrow$ [vexp] $\rightarrow$ vexp \\
        &$\rightarrow$ reply (subst, type\_exp)
    \end{tabular}\\
    \begin{tabular}{@{}r@{\hspace{4pt}}l}
        tcletrec &gamma ns xs es e \\
        = &tcletrec1 gamma ns0 nbvs e (tcl (nbvs ++ gamma) ns1 es) \\
    \end{tabular}\\
    \begin{tabular}{r@{\hspace{4pt}}l@{\hspace{4pt}}l}
    \hspace*{1.25cm}where &(ns0,ns$'$) &= split ns \\
    &(ns1,ns2) &= split ns$'$ \\
    &nbvs &= new\_bvars xs ns2
    \end{tabular}\\
\end{mlcoded}

\begin{mlcoded}
    new\_bvars xs ns = map new\_bvar (xs \$zip (name\_sequence ns))
\end{mlcoded}

\begin{mlcoded}
    tcletrec1 gamma ns nbvs e FAILURE \\
    \hspace*{0.75cm}= FAILURE \\
    tcletrec1 gamma ns nbvs e (OK (phi,ts)) \\
    \hspace*{0.75cm}= tcletrec2 gamma$'$ ns nbvs$'$ e (unify1 phi (ts \$zip ts$'$)) \\
    \begin{tabular}{r@{\hspace{4pt}}l@{\hspace{4pt}}l}
        \hspace*{1cm}where &ts$'$   &= map old\_bvar nbvs$'$ \\
        &nbvs$'$ &= sub\_te phi nbvs \\
        &gamma$'$ &= sub\_te phi gamma
    \end{tabular}
\end{mlcoded}

\begin{mlcoded}
    old\_bvar (x, SCHEME [ ] t) = t
\end{mlcoded}

\begin{mlcoded}
    tcletrec2 gamma ns nbvs e FAILURE \\
    \hspace*{0.75cm}= FAILURE \\
    tcletrec2 gamma ns nbvs e (OK phi) \\
    \hspace*{0.75cm}= tclet2 phi (tc gamma$''$ ns1 e) \\
    \begin{tabular}{@{}r@{\hspace{4pt}}l@{\hspace{4pt}}l}
        \hspace*{1.2cm}where &ts       &= map old\_bvar nbvs$'$ \\
        &nbvs$'$   &= sub\_te phi nbvs \\
        &gamma$'$  &= sub\_te phi gamma \\
        &gamma$''$ &= add\_decls gamma$'$ ns0 (map fst nbvs) ts \\
        &(ns0,ns1) &= split ns
    \end{tabular}
\end{mlcoded}
The definition of the type-checker is now complete.

\section*{References}

\begin{references}

    \item Damas, L. 1985. \textit{Type Assignment in Programming Languages}. CST-33-35.
    Department of Computer Science, University of Edinburgh. April.

    \item Robinson, J.A. 1965. A machine-oriented logic based on the resolution principle.
    \textit{Journal of the ACM}. Vol. 12, no. 1, pp. 23–41.

    \item Wadler, P. 1985. How to replace failure by a list of successes. In \textit{Conference on
        Functional Programming Languages and Computer Architecture, Nancy}.
    Jouannaud (editor). LNCS 201. Springer Verlag.

\end{references}