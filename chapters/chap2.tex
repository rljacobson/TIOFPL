\chapter{The Lambda Calculus}

This chapter introduces the lambda calculus, a simple language which will be
used throughout the rest of the book as a bridge between high-level functional
languages and their low-level implementations. The reasons for introducing
the lambda calculus as an intermediate language are:
\begin{numbered}
\item It is a simple language, with only a few, syntactic constructs, and simple
semantics. These properties make it a good basis for a discussion of
implementations, because an implementation of the lambda calculus only
has to support a few constructs, and the simple semantics allows us to
reason about the correctness of the implementation.
\item It is an expressive language, which is sufficiently powerful to express all
functional programs (and indeed, all computable functions). This means
that if we have an implementation of the lambda calculus, we can
implement any other functional language by translating it into the lambda
calculus.
\end{numbered}
In this chapter we focus on the syntax and semantics of the lambda calculus
itself, before turning our attention to high-level functional languages in the
next chapter.

\section{The Syntax of the Lambda Calculus}

Here is a simple expression in the lambda calculus:
\begin{mlcoded}
(+ 4 5)
\end{mlcoded}
All function applications in the lambda calculus are written in \textit{prefix} form, so,
for example, the function + precedes its arguments 4 and 5. A slightly more
complex example, showing the (quite conventional) use of brackets, is
\begin{mlcoded}
(+ ($*$ 5 6) (+ 8 3))
\end{mlcoded}
In both examples, the outermost brackets are redundant, but have been
added for clarity (see Section 2.1.2).

From the implementation viewpoint, a functional program should be
thought of as an \textit{expression}, which is `executed' by \textit{evaluating} it. Evaluation
proceeds by repeatedly selecting a \textit{reducible expression} (or \textit{redex}) and
reducing it. In our last example there are two redexes: ($*$ 5 6) and (+ 8 3).
The whole expression \ml{(+ ($*$ 5 6) ($*$ 8 3))} is not a redex, since a \ml{+} needs to be
applied to two \textit{numbers} before it is reducible. Arbitrarily choosing the first
redex for reduction, we write
\begin{mlcoded}
(+ ($*$ 5 6) ($*$ 8 3)) $\to$ (+ 30 ($*$ 8 3))
\end{mlcoded}
where the \ml{$\to$} is pronounced `reduces to'. Now there is only one redex, \ml{(* 8 3)},
which gives
\begin{mlcoded}
(+ 30 ($*$ 8 3)) $\to$ (+ 30 24)
\end{mlcoded}
This reduction creates a new redex, which we now reduce
\begin{mlcoded}
(+ 30 24) $\to$ 54
\end{mlcoded}
\indent When there are several redexes we have a choice of which one to reduce
first. This issue will be addressed later in this chapter.

\subsection{Function Application and Currying}
In the lambda calculus, function application is so important that it is denoted
by simple juxtaposition; thus we write
\begin{mlcoded}
f x
\end{mlcoded}
to denote `the function \ml{f} applied to the argument \ml{x}'. How should we express
the application of a function to several arguments? We could use a new
notation, like \ml{(f (x,y))}, but instead we use a simple and rather ingenious
alternative. To express `the sum of 3 and 4' we write
\begin{mlcoded}
	((+ 3) 4)
\end{mlcoded}
The expression \ml{(+ 3)} denotes the function that adds \ml{3} to its argument. Thus
the whole expression means `the function \ml{+} applied to the argument \ml{3}, the
result of which is a function applied to \ml{4}'. (In common with all functional
programming languages, the lambda calculus allows a function to return a
function as its result.)

This device allows us to think of all functions as having a \textit{single argument
only}. It was introduced by Schonfinkel [1924] and extensively used by Curry
[Curry and Feys, 1958]; as a result it is known as \textit{currying}.

\subsection{Use of Brackets}

In mathematics it is conventional to omit redundant brackets to avoid
cluttering up expressions. For example, we might omit brackets from the
expression
\begin{mlcoded}
	(ab) + ((2c)/d)
\end{mlcoded}
to give
\begin{mlcoded}
	ab + 2c/d
\end{mlcoded}
The second expression is easier to read than the first, but there is a danger that
it may be ambiguous. It is rendered unambiguous by establishing conventions
about the precedence of the various functions (for example, multiplication
binds more tightly than addition)..

Sometimes brackets cannot be omitted, as in the expression:
\begin{mlcoded}
	(b + c)/a
\end{mlcoded}
Similar conventions are useful when writing down expressions in the
lambda calculus. Consider the expression:
\begin{mlcoded}
	((+ 3) 2)
\end{mlcoded}
By establishing the convention that function application associates to the left,
we can write the expression more simply as:
\begin{mlcoded}
	(+ 3 2)
\end{mlcoded}
or even
\begin{mlcoded}
	+ 3 2
\end{mlcoded}

We performed some such abbreviations in the examples given earlier. As a
more complicated example, the expression:
\begin{mlcoded}
	((f ((+ 4) 3)) (g x))
\end{mlcoded}
is fully bracketed and unambiguous. Following our convention, we may omit
redundant brackets to make the expression easier to read, giving:
\begin{mlcoded}
	f (+ 4 3) (g x)
\end{mlcoded}
No further brackets can be omitted. Extra brackets may, of course, be
inserted freely without changing the meaning of the expression; for example
\begin{mlcoded}
	(f (+ 4 3) (g x))
\end{mlcoded}
is the same expression again.

\subsection{Built-in Functions and Constants}
In its purest form the lambda calculus does not have built-in functions such as
\ml{+}, but our intentions are practical and so we extend the pure lambda calculus
with a suitable collection of such built-in functions.

These include arithmetic functions (such as \ml{+}, \ml{--}, $*$, \ml{/}\,) and constants (\ml{0}, \ml{1},
\ldots), logical functions (such as \ml{AND}, \ml{OR}, \ml{NOT}) and constants (\ml{TRUE},
\ml{FALSE}), and character constants (\ml{`a'}, \ml{`b'}, \ldots). For example
\begin{mlalign}
	-- 5 4 &$\to$ 1\\
AND TRUE FALSE &$\to$ FALSE
\end{mlalign}
We also include a conditional function, \ml{IF}, whose behavior is described by the
reduction rules:
\begin{mlcoded}
IF TRUE E\textsubscript{t} E\textsubscript{f} $\to$ E\textsubscript{t}\\
IF FALSE E\textsubscript{t} E\textsubscript{f} $\to$ E\textsubscript{f}
\end{mlcoded}

We will initially introduce data constructors into the lambda calculus by
using the built-in functions \ml{CONS} (short for \ml{CONSTRUCT}\,), \ml{HEAD} and \ml{TAIL}
(which behave exactly like the Lisp functions \ml{CONS}, \ml{CAR} and \ml{CDR}\,). The
constructor \ml{CONS} builds a compound object which can be taken apart with
\ml{HEAD} and \ml{TAIL}. We may describe their operation by the following rules:
\begin{mlalign}
HEAD (CONS a b) &$\to$ a\\
TAIL (CONS a b) &$\to$ b
\end{mlalign}
We also include \ml{NIL}, the empty list, as a constant. The data constructors will
be discussed at greater length in Chapter 4.

The exact choice of built-in functions is, of course, somewhat arbitrary, and
further ones will be added as the need arises.

\subsection{Lambda Abstractions}

The only functions introduced so far have been the built-in functions (such as
\ml{+} and \ml{CONS}\,). However, the lambda calculus provides a construct, called a
lambda abstraction, to denote new (non-built-in) functions. A lambda
abstraction is a particular sort of expression which denotes a function. Here is
an example of a lambda abstraction:
\begin{mlcoded}
	(\tlb{x}+ x 1)
\end{mlcoded}
The \tl says `here comes a function', and is immediately followed by a variable,
\ml{x} in this case; then comes a . followed by the \textit{body} of the function, \ml{(+ x 1)} in
this case. The variable is called the \textit{formal parameter}, and we say that the \tl
binds it. You can think of it like this:
\begin{quote}\setlength{\tabcolsep}{3pt}
\begin{tabular}{llllll}
	\ml{(\tl}	&\ml{x}  &\ml{.}  &\ml{+}  &\ml{x} &\ml{1}\,)  \\
	\;\;$\uparrow$	&$\uparrow$  &$\uparrow$  &$\uparrow$  &$\uparrow$ &$\uparrow$ \\
	That function of &\ml{x}  & which  &adds  &\ml{x} to &\ml{1} 
\end{tabular}
\end{quote}
A lambda abstraction \textit{always} consists of all the four parts mentioned: the \tl,
the formal parameter, the \ml{.} and the body.

A lambda abstraction is rather similar to a function definition in a
conventional language, such as C:
\begin{mlcoded}
	Inc( x )\\
	int x;\\
	\{ return( x + 1 ); \}
\end{mlcoded}
The formal parameter of the lambda abstraction corresponds to the formal
parameter of the function, and the body of the abstraction is an expression
rather than a sequence of commands. However, functions in conventional
languages must have a name (such as \ml{Inc}), whereas lambda abstractions are
`anonymous' functions.

The body of a lambda abstraction extends \textit{as far to the right as possible}, so
that in the expression
\begin{mlcoded}
	(\tlb{x}+ x 1) 4
\end{mlcoded}
the body of the \ml{\tl{}x} abstraction is \ml{(+ x 1)}, not just \ml{+}. As usual, we may add
extra brackets to clarify, thus
\begin{mlcoded}
	(\tlb{x}(+ x 1)) 4
\end{mlcoded}
When a lambda abstraction appears in isolation we may write it without any
brackets:
\begin{mlcoded}
	\tlb{x}+ x 1
\end{mlcoded}

\subsection{Summary}
We define a \textit{lambda expression} to be an expression in the lambda calculus, and
Figure 2. 1 summarizes the forms which a lambda expression may take. Notice
that a lambda \textit{abstraction} is not the same as a lambda \textit{expression}; in fact the
former is a particular instance of the latter.

<exp> :: = <constant>
Built-in constants
<variable>
Variable names
<exp> <exp>
A <variable>.<exp>
Applications
Lambda abstractions
This is the abstract syntax of lambda expressions. In order to write down
such an expression in concrete form we use brackets to disambiguate its
structure (see Section 2.1.2).
We will use lower-case letters for variables (e.g. x, f), and upper-case
letters to stand for whole lambda expressions (e.g. M, E).
The choice of constants is rather arbitrary; we assume integers and
booleans (e.g. 4, TRUE), together with built-in functions to manipulate
them (e.g. AND, IF, +). We also assume built-in list-processing functions
(e.g. CONS, HEAD).
Figure 2.1 Syntax of a lambda expression (in BNF)