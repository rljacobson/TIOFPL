\chapter{The Lambda Calculus}

This chapter introduces the lambda calculus, a simple language which will be
used throughout the rest of the book as a bridge between high-level functional
languages and their low-level implementations. The reasons for introducing
the lambda calculus as an intermediate language are:
\begin{numbered}
\item It is a simple language, with only a few, syntactic constructs, and simple
semantics. These properties make it a good basis for a discussion of
implementations, because an implementation of the lambda calculus only
has to support a few constructs, and the simple semantics allows us to
reason about the correctness of the implementation.
\item It is an expressive language, which is sufficiently powerful to express all
functional programs (and indeed, all computable functions). This means
that if we have an implementation of the lambda calculus, we can
implement any other functional language by translating it into the lambda
calculus.
\end{numbered}
In this chapter we focus on the syntax and semantics of the lambda calculus
itself, before turning our attention to high-level functional languages in the
next chapter.

\section{The Syntax of the Lambda Calculus}

Here is a simple expression in the lambda calculus:
\begin{mlcoded}
($+$ 4 5)
\end{mlcoded}
All function applications in the lambda calculus are written in \textit{prefix} form, so,
for example, the function $+$ precedes its arguments 4 and 5. A slightly more
complex example, showing the (quite conventional) use of brackets, is
\begin{mlcoded}
($+$ ($*$ 5 6) ($+$ 8 3))
\end{mlcoded}
In both examples, the outermost brackets are redundant, but have been
added for clarity (see Section 2.1.2).

From the implementation viewpoint, a functional program should be
thought of as an \textit{expression}, which is `executed' by \textit{evaluating} it. Evaluation
proceeds by repeatedly selecting a \textit{reducible expression} (or \textit{redex}) and
reducing it. In our last example there are two redexes: ($*$ 5 6) and ($+$ 8 3).
The whole expression \ml{($+$ ($*$ 5 6) ($*$ 8 3))} is not a redex, since a \ml{+} needs to be
applied to two \textit{numbers} before it is reducible. Arbitrarily choosing the first
redex for reduction, we write
\begin{mlcoded}
($+$ ($*$ 5 6) ($*$ 8 3)) $\to$ ($+$ 30 ($*$ 8 3))
\end{mlcoded}
where the \ml{$\to$} is pronounced `reduces to'. Now there is only one redex, \ml{(* 8 3)},
which gives
\begin{mlcoded}
($+$ 30 ($*$ 8 3)) $\to$ ($+$ 30 24)
\end{mlcoded}
This reduction creates a new redex, which we now reduce
\begin{mlcoded}
($+$ 30 24) $\to$ 54
\end{mlcoded}
\indent When there are several redexes we have a choice of which one to reduce
first. This issue will be addressed later in this chapter.

\subsection{Function Application and Currying}
In the lambda calculus, function application is so important that it is denoted
by simple juxtaposition; thus we write
\begin{mlcoded}
f x
\end{mlcoded}
to denote `the function \ml{f} applied to the argument \ml{x}'. How should we express
the application of a function to several arguments? We could use a new
notation, like \ml{(f (x,y))}, but instead we use a simple and rather ingenious
alternative. To express `the sum of 3 and 4' we write
\begin{mlcoded}
	(($+$ 3) 4)
\end{mlcoded}
The expression \ml{($+$ 3)} denotes the function that adds \ml{3} to its argument. Thus
the whole expression means `the function \ml{+} applied to the argument \ml{3}, the
result of which is a function applied to \ml{4}'. (In common with all functional
programming languages, the lambda calculus allows a function to return a
function as its result.)

This device allows us to think of all functions as having a \textit{single argument
only}. It was introduced by Schonfinkel [1924] and extensively used by Curry
[Curry and Feys, 1958]; as a result it is known as \textit{currying}.

\subsection{Use of Brackets}

In mathematics it is conventional to omit redundant brackets to avoid
cluttering up expressions. For example, we might omit brackets from the
expression
\begin{mlcoded}
	(ab) $+$ ((2c)/d)
\end{mlcoded}
to give
\begin{mlcoded}
	ab $+$ 2c/d
\end{mlcoded}
The second expression is easier to read than the first, but there is a danger that
it may be ambiguous. It is rendered unambiguous by establishing conventions
about the precedence of the various functions (for example, multiplication
binds more tightly than addition)..

Sometimes brackets cannot be omitted, as in the expression:
\begin{mlcoded}
	(b $+$ c)/a
\end{mlcoded}
Similar conventions are useful when writing down expressions in the
lambda calculus. Consider the expression:
\begin{mlcoded}
	(($+$ 3) 2)
\end{mlcoded}
By establishing the convention that function application associates to the left,
we can write the expression more simply as:
\begin{mlcoded}
	($+$ 3 2)
\end{mlcoded}
or even
\begin{mlcoded}
    $+$ 3 2
\end{mlcoded}

We performed some such abbreviations in the examples given earlier. As a
more complicated example, the expression:
\begin{mlcoded}
	((f (($+$ 4) 3)) (g x))
\end{mlcoded}
is fully bracketed and unambiguous. Following our convention, we may omit
redundant brackets to make the expression easier to read, giving:
\begin{mlcoded}
	f ($+$ 4 3) (g x)
\end{mlcoded}
No further brackets can be omitted. Extra brackets may, of course, be
inserted freely without changing the meaning of the expression; for example
\begin{mlcoded}
	(f ($+$ 4 3) (g x))
\end{mlcoded}
is the same expression again.

\subsection{Built-in Functions and Constants}
In its purest form the lambda calculus does not have built-in functions such as
\ml{+}, but our intentions are practical and so we extend the pure lambda calculus
with a suitable collection of such built-in functions.

These include arithmetic functions (such as \ml{+}, \ml{--}, $*$, \ml{/}\,) and constants (\ml{0}, \ml{1},
\ldots), logical functions (such as \ml{AND}, \ml{OR}, \ml{NOT}) and constants (\ml{TRUE},
\ml{FALSE}), and character constants (\ml{`a'}, \ml{`b'}, \ldots). For example
\begin{mlalign}
	$-$ 5 4 &$\to$ 1\\
AND TRUE FALSE &$\to$ FALSE
\end{mlalign}
We also include a conditional function, \ml{IF}, whose behavior is described by the
reduction rules:
\begin{mlcoded}
IF TRUE E\textsubscript{t} E\textsubscript{f} $\to$ E\textsubscript{t}\\
IF FALSE E\textsubscript{t} E\textsubscript{f} $\to$ E\textsubscript{f}
\end{mlcoded}

We will initially introduce data constructors into the lambda calculus by
using the built-in functions \ml{CONS} (short for \ml{CONSTRUCT}\,), \ml{HEAD} and \ml{TAIL}
(which behave exactly like the Lisp functions \ml{CONS}, \ml{CAR} and \ml{CDR}\,). The
constructor \ml{CONS} builds a compound object which can be taken apart with
\ml{HEAD} and \ml{TAIL}. We may describe their operation by the following rules:
\begin{mlalign}
HEAD (CONS a b) &$\to$ a\\
TAIL (CONS a b) &$\to$ b
\end{mlalign}
We also include \ml{NIL}, the empty list, as a constant. The data constructors will
be discussed at greater length in Chapter 4.

The exact choice of built-in functions is, of course, somewhat arbitrary, and
further ones will be added as the need arises.

\subsection{Lambda Abstractions}

The only functions introduced so far have been the built-in functions (such as
\ml{+} and \ml{CONS}\,). However, the lambda calculus provides a construct, called a
lambda abstraction, to denote new (non-built-in) functions. A lambda
abstraction is a particular sort of expression which denotes a function. Here is
an example of a lambda abstraction:
\begin{mlcoded}
	(\tlb{x}$+$ x 1)
\end{mlcoded}
The \tl says `here comes a function', and is immediately followed by a variable,
\ml{x} in this case; then comes a . followed by the \textit{body} of the function, \ml{($+$ x 1)} in
this case. The variable is called the \textit{formal parameter}, and we say that the \tl
binds it. You can think of it like this:
\begin{quote}\setlength{\tabcolsep}{3pt}
\begin{tabular}{llllll}
	\ml{(\tl}	&\ml{x}  &\ml{.}  &\ml{+}  &\ml{x} &\ml{1}\,)  \\
	\;\;$\uparrow$	&$\uparrow$  &$\uparrow$  &$\uparrow$  &$\uparrow$ &$\uparrow$ \\
	That function of &\ml{x}  & which  &adds  &\ml{x} to &\ml{1}
\end{tabular}
\end{quote}
A lambda abstraction \textit{always} consists of all the four parts mentioned: the \tl,
the formal parameter, the \ml{.} and the body.

A lambda abstraction is rather similar to a function definition in a
conventional language, such as C:
\begin{mlcoded}
	Inc( x )\\
	int x;\\
	\{ return( x $+$ 1 ); \}
\end{mlcoded}
The formal parameter of the lambda abstraction corresponds to the formal
parameter of the function, and the body of the abstraction is an expression
rather than a sequence of commands. However, functions in conventional
languages must have a name (such as \ml{Inc}), whereas lambda abstractions are
`anonymous' functions.

The body of a lambda abstraction extends \textit{as far to the right as possible}, so
that in the expression
\begin{mlcoded}
	(\tlb{x}$+$ x 1) 4
\end{mlcoded}
the body of the \ml{\tl\,x} abstraction is \ml{($+$ x 1)}, not just \ml{$+$}. As usual, we may add
extra brackets to clarify, thus
\begin{mlcoded}
	(\tlb{x}($+$ x 1)) 4
\end{mlcoded}
When a lambda abstraction appears in isolation we may write it without any
brackets:
\begin{mlcoded}
	\tlb{x}$+$ x 1
\end{mlcoded}

\subsection{Summary}
We define a \textit{lambda expression} to be an expression in the lambda calculus, and
Figure 2. 1 summarizes the forms which a lambda expression may take. Notice
that a lambda \textit{abstraction} is not the same as a lambda \textit{expression}; in fact the
former is a particular instance of the latter.

\boxedfigure{

	\begin{tabular}{lllll}
		<exp>	& ::=  & <constant> & & Built-in constants  \\
			& |  & <variable> & & Variable names  \\
			& |  & <exp> <exp> & & Applications  \\
			& |  & \ml{\tl} <variable>\ml{.}<exp> & & Lambda abstractions
	\end{tabular}
	\vspace{\baselineskip}

\noindent This is the \textit{abstract} syntax of lambda expressions. In order to write down
such an expression in \textit{concrete} form we use brackets to disambiguate its
structure (see Section 2.1.2).

We will use lower-case letters for variables (e.g. \ml{x}, \ml{f}), and upper-case
letters to stand for whole lambda expressions (e.g. \ml{M}, \ml{E}).

The choice of constants is rather arbitrary; we assume integers and
booleans (e.g. \ml{4}, \ml{TRUE}), together with built-in functions to manipulate
them (e.g. \ml{AND}, \ml{IF}, \ml{+}). We also assume built-in list-processing functions
(e.g. \ml{CONS}, \ml{HEAD}).
}{Syntax of a lambda expression (in BNF)}

In what follows we will use lower-case names for variables, and single
upper-case letters to stand for whole lambda expressions. For example we
might say `for any lambda expression \ml{E}, \dots,'. We will also write the names of
built-in functions in upper case, but no confusion should arise.

\section{The Operational Semantics of the Lambda Calculus}

So far we have described only the \textit{syntax} of the lambda calculus, but to dignify
it with the title of a `calculus' we must say how to `calculate' with it. We will do
this by giving three \textit{conversion rules} which describe how to convert one
lambda expression into another.

First, however, we introduce an important piece of terminology.

\subsection{Bound and Free Variables}

Consider the lambda expression
\begin{mlcoded}
	(\tlb{x}$+$ x y) 4
\end{mlcoded}
In order to evaluate this expression completely, we need to know the `global'
value of \ml{y}. In contrast, we do not need to know a `global' value for \ml{x}, since it is
just the formal parameter of the function, so we see that \ml{x} and \ml{y} have a rather
different status.

The reason is that \ml{x} occurs bound by the \ml{\tl\,x}; it is just a `hole' into which

\boxedfigure{
	\noindent An occurrence of a variable must be either free or bound.\\

	\textit{Definition of 'occurs free'}

	\hspace{-0.4em}\ml{x} occurs free in \ml{x} (but not in any other variable or constant)\\
	\begin{tabular}{llrl}
	  \ml{x} occurs free in \ml{(E F)} & <=> & & \ml{x} occurs free in \ml{E} \\
		& & \textit{or} & \ml{x} occurs free in \ml{F} \\
	  \ml{x} occurs free in \ml{\tlb{y} E} & <=> & & \ml{x} and \ml{y} are different variables \\
		& & \textit{and} & \ml{x} occurs free in \ml{E} \\
	\end{tabular}\\

	\textit{Definition of `occurs bound'}

	\begin{tabular}{llrl}
		\ml{x} occurs bound in \ml{(E F)} & <=> & & \ml{x} occurs bound in \ml{E}\\
		& & \textit{or} & \ml{x} occurs bound in \ml{F} \\
		\ml{x} occurs bound in \ml{\tlb{y}E} & <=> & & (\ml{x} and \ml{y} are the same variable\\
		& & & \textit{and} \ml{x} occurs free in \ml{E}) \\
		& & \textit{or} & \ml{x} occurs bound in \ml{E} \\
	\end{tabular}\\

	\noindent(No variable occurs bound in an expression consisting of a single constant
	or variable.)\\

	\noindent Note: '<=>' means 'if and only if'
}{Definitions of bound and free}


\noindent the
argument \ml{4} is placed when applying the lambda abstraction to its argument.
On the other hand, \ml{y} is not bound by any \ml{A}, and so occurs \textit{free} in the
expression. In general, the value of an expression depends only on the values
of its free variables.

An occurrence of a variable is bound if there is an enclosing lambda
abstraction which binds it, and is free otherwise. For example, \ml{x} and \ml{y} occur
bound, but \ml{z} occurs free in this example:
\begin{mlcoded}
	\tlb{x}$+$ ((\tlb{y}$+$ y z) 7) x
\end{mlcoded}
Notice that the terms `bound' and `free' refer to \textit{specific occurrences} of the
variable in an expression. This is because a variable may have both a bound
occurrence and a free occurrence in an expression; consider for example
\begin{mlcoded}
	$+$ x ((\tlb{x}$+$ x 1) 4)
\end{mlcoded}
in which \ml{x} occurs free (the first time) and bound (the second time). Each
individual occurrence of a variable must be either free or bound.

Figure 2.2 gives formal definitions for `free' and `bound', which cover the
forms of lambda expression given in Figure 2.1 case by case.

\subsection{Beta-conversion}

A lambda abstraction denotes a function, so we must describe how to apply it
to an argument. For example, the expression
\begin{mlcoded}
	(\tlb{x}$+$ x 1) 4
\end{mlcoded}
is the juxtaposition of the lambda abstraction \ml{(\tlb{x}$+$ x 1)} and the argument \ml{4},
and hence denotes the application of a certain function, denoted by the
lambda abstraction, to the argument \ml{4}. The rule for such function application
is very simple:
\begin{quote}
	The result of applying a lambda abstraction to an argument is an instance of
	the body of the lambda abstraction in which (free) occurrences of the
	formal parameter in the body are replaced with (copies of) the argument.
\end{quote}
Thus the result of applying the lambda abstraction \ml{(\tlb{x}$+$ x 1)} to the
argument \ml{4} is
\begin{mlcoded}
    $+$ 4 1
\end{mlcoded}
The \ml{($+$ 4 1)} is an instance of the body \ml{($+$ x 1)} in which occurrences of the
formal parameter, \ml{x}, are replaced with the argument, \ml{4}. We write the
reduction using the arrow `$\rightarrow$' as before:
\begin{mlcoded}
	(\tlb{x}$+$ x 1) 4 $\rightarrow$ $+$ 4 1
\end{mlcoded}
This operation is called \textit{\tb{}-reduction}, and much of this book is concerned with
its efficient implementation. We will use a series of examples to show in detail
how \tb{}-reduction works.

\subsubsection{Simple examples of beta-reduction}

The formal parameter may occur several times in the body:
\begin{mlalign}
	(\tlb{x}$+$ x x) 5  & $\rightarrow$ $+$ 5 5 \\
	 & $\rightarrow$ 10
\end{mlalign}

Equally, there may be no occurrences of the formal parameter in the body:
\begin{mlcoded}
	(\tlb{x}3) 5 $\rightarrow$ 3
\end{mlcoded}
In this case there are no occurrences of the formal parameter (\ml{x}) for which the
argument (\ml{5}) should be substituted, so the argument is discarded unused.

The body of a lambda abstraction may consist of another lambda
abstraction:
\begin{mlalign}
	(\tlb{x}(\tlb{y}$-$ y x)) 4 5 &$\rightarrow$ (\tlb{y}$-$ y 4) 5 \\
	& $\rightarrow$ 5 4 \\
	& $\rightarrow$ 1
\end{mlalign}
Notice that, when constructing an instance of the body of the \ml{\tl\,x} abstraction,
we copy the entire body including the embedded \ml{\tl\,x} abstraction (while
substituting for \ml{x}, of course). Here we see currying in action: the application
of the \ml{\tl\,x} abstraction returned a function (the \ml{\tl\,y} abstraction) as its result,
which when applied yielded the result (\ml{- 5 4}).

We often abbreviate
\begin{mlcoded}
	(\tlb{x}(\tlb{y}E))
\end{mlcoded}
to
\begin{mlcoded}
	(\tlb{x}\tlb{y}E)
\end{mlcoded}

Functions can be arguments too:
\begin{mlalign}
	(\tlb{f}f 3) (\tlb{x}$+$ x 1) & $\rightarrow$ (\tlb{x}$+$ x 1) 3 \\
    & $\rightarrow$  $+$ 3 1 \\
    & $\rightarrow$  4
\end{mlalign}
An instance of the \ml{\tl\,x} abstraction is substituted for \ml{f} wherever \ml{f} appears in the
body of the \ml{\tl{}f} abstraction.

\subsubsection{Naming}
Some slight care is needed when formal parameter names are not unique. For example,
\begin{mlalign}
    & \phantom{$\rightarrow$} (\tlb{x}(\tlb{x}$+$ ($-$ x 1)) x 3) 9  \\
    & $\rightarrow$ (\tlb{x}$+$ ($-$ x 1)) 9 3 \\
    & $\rightarrow$ $+$ ($-$ 9 1) 3 \\
    & $\rightarrow$ 11
\end{mlalign}
Notice that we did \textit{not} substitute for the inner \ml{x} in the first reduction, because
it was shielded by the enclosing \ml{\tl\,x}; that is, the inner occurrence of \ml{x} is not free
in the body of the outer \ml{\tl\,x} abstraction.


Given a lambda abstraction \ml{(\tlb{x}E)}, how can we identify exactly those
occurrences of \ml{x} which should be substituted for? It is easy: we should
substitute for \textit{those occurrences of \ml{x} which are free} in \ml{E}, because, if they are
free in \ml{E}, then they will be bound by the \ml{\tl\,x} abstraction \ml{(\tlb{x}E)}. So, when
applying the outer \ml{\tl\,x} abstraction in the above example, we examine its body
\begin{mlcoded}
    (\tlb{x}$+$ ($-$ x 1)) x 3
\end{mlcoded}
and see that only the second occurrence of \ml{x} is free, and hence qualifies for
substitution.

This is why the rule given above specified that only the \textit{free} occurrences of
the formal parameter in the body are to be substituted for. The nesting of the
scope of variables in a block-structured language is closely analogous to this
rule.

Here is another example of the same kind
\begin{mlalign}
    & \phantom{$\rightarrow$} (\tlb{x}\tlb{y}$+$ x ((\tlb{x}$-$ x 3) y)) 5 6 \\
    & $\rightarrow$ (\tlb{y}$+$ 5 ((\tlb{x}$-$ x 3) y)) 6 \\
    & $\rightarrow$ $+$ 5 ((\tlb{x}$-$ x 3) 6) \\
    & $\rightarrow$ $+$ 5 ($-$ 6 3) \\
    & $\rightarrow$ 8
\end{mlalign}
Again, the inner \ml{x} is not substituted for in the first reduction, since it is not free
in the body of the outer \ml{\tl\,x} abstraction.

\subsubsection{A larger example}
As a larger example, we will demonstrate the somewhat surprising fact that
data constructors can actually be modelled as pure lambda abstractions. We
define \ml{CONS}, \ml{HEAD} and \ml{TAIL} in the following way:
\begin{mlalign}
    CONS &= (\tlb{a}\tlb{b}\tlb{f}f a b) \\
    HEAD &= (\tlb{c}c (\tlb{a}\tlb{b}a)) \\
    TAIL &= (\tlb{c}c (\tlb{a}\tlb{b}b))
\end{mlalign}
These obey the rules for \ml{CONS}, \ml{HEAD} and \ml{TAIL} given in Section 2.1.3. For
example,
\begin{mlalign}
    \phantom{$\rightarrow$} & HEAD (CONS p q) \\
    $=$ & (\tlb{c}c (\tlb{a}\tlb{b}a)) (CONS p q) \\
    $\rightarrow$ & CONS p q (\tlb{a}\tlb{b}a) \\
    $=$ & (\tlb{a}\tlb{b}\tlb{f}f a b) p q (\tlb{a}\tlb{b}a) \\
    $\rightarrow$ & (\tlb{b}\tlb{f}f p b) q (\tlb{a}\tlb{b}a) \\
    $\rightarrow$ & (\tlb{f}f p q) (\tlb{a}\tlb{b}a) \\
    $\rightarrow$ & (\tlb{a}\tlb{b}a) p q \\
    $\rightarrow$ & (\tlb{b}p) q \\
    $\rightarrow$ & p
\end{mlalign}
This means, incidentally, that there is no essential need for the built-in
functions \ml{CONS}, \ml{HEAD} and \ml{TAIL}, and it turns out that all the other built-in functions can also be modelled as lambda abstractions. This is rather satisfying from a theoretical viewpoint, but all practical implementations support
built-in functions for efficiency reasons.

\subsubsection{Conversion, reduction and abstraction}
We can use the \ml{\tb{}}-rule backwards, to introduce new lambda abstractions, thus
\begin{mlcoded}
    $+$ 4 1 $\leftarrow$ (\tlb{x}$+$ x 1) 4
\end{mlcoded}
This operation is called \tb{}-\textit{abstraction}, which we denote with a backwards reduction arrow `$\leftarrow$'. \tb{}-\textit{conversion} means \tb{}-reduction or \tb{}-abstraction, and we denote it with a double-ended arrow `\conversion{\beta}'. Thus we write
\begin{mlcoded}
    $+$ 4 1 \conversion{\beta} (\tlb{x}$+$ x 1) 4
\end{mlcoded}
The arrow is decorated with \tb{} to distinguish \tb{}-conversion from the other forms of conversion we will meet shortly. An undecorated reduction arrow `$\rightarrow$' will stand for one or more \tb{}-reductions, or reductions of the built-in functions. An undecorated conversion arrow `$\leftrightarrow$' will stand for zero or more conversions, of any kind.

Rather than regarding \tb{}-reduction and \tb{}-abstraction as \textit{operations}, we can regard \tb{}-conversion as expressing the \textit{equivalence} of two expressions which `look different' but `ought to mean the same'. It turns out that we need two more rules to satisfy our intuitions about the equivalence of expressions, and we turn to these rules in the next two sections.

\subsection{Alpha-conversion}
Consider the two lambda abstractions
\begin{mlcoded}
    (\tlb{x}$+$ x 1)
\end{mlcoded}
and
\begin{mlcoded}
    (\tlb{y}$+$ y 1)
\end{mlcoded}
Clearly they `ought' to be equivalent, and \ta-\textit{conversion} allows us to change the name of the formal parameter of any lambda abstraction, so long as we do so consistently. So
\begin{mlcoded}
    (\tlb{x}$+$ x 1) \conversion{\alpha} (\tlb{y}$+$ y 1)
\end{mlcoded}
where the arrow is decorated with an \ta{} to specify an \ta-conversion. The newly introduced name must not, of course, occur free in the body of the original lambda abstraction. \ta-conversion is used solely to eliminate the sort of name clashes exhibited in the example in the previous section.

Sometimes \ta{}-conversion is essential (see Section 2.2.6).

\subsection{Eta-conversion}
One more conversion rule is necessary to express our intuitions about what lambda abstractions `ought' to be equivalent. Consider the two expressions
\begin{mlcoded}
    (\tlb{x}$+$ 1 x)
\end{mlcoded}
and
\begin{mlcoded}
    ($+$ 1)
\end{mlcoded}
These expressions behave in exactly the same way when applied to an argument: they add 1 to it. \te{}-conversion is a rule expressing their equivalence:
\begin{mlcoded}
    (\tlb{x}$+$ 1 x) \reduction{\eta} ($+$ 1)
\end{mlcoded}
More generally, we can express the \te{}-conversion rule like this:
\begin{mlcoded}
    (\tlb{x}F x) \reduction{\eta} F
\end{mlcoded}
provided \ml{x} does not occur free in \ml{F}, and \ml{F} denotes a function.

The condition that \ml{x} does not occur free in \ml{F} prevents false conversions. For example,
\begin{mlcoded}
    (\tlb{x}$+$ x x)
\end{mlcoded}
is not \te{}-convertible to
\begin{mlcoded}
    ($+$ x)
\end{mlcoded}
because \ml{x} occurs free in \ml{($+$ x)}. The condition that \ml{F} denotes a function prevents other false conversions involving built-in constants; for example:
\begin{mlcoded}
    TRUE
\end{mlcoded}
is not \te{}-convertible to
\begin{mlcoded}
    (\tlb{x}TRUE x)
\end{mlcoded}
When the \te{}-conversion rule is used from left to right it is called \te{}-reduction.

\subsection{Proving Interconvertibility}
We will quite frequently want to prove the interconvertibility of two lambda expressions. When the two expressions denote a function such proofs can become rather tedious, and in this section we will demonstrate a convenient method that abbreviates the proof without sacrificing rigor.

As an example, consider the two lambda expressions:
\begin{mlcoded}
    IF TRUE ((\tlb{p}p) 3)
\end{mlcoded}
and
\begin{mlcoded}
    (\tlb{x}3)
\end{mlcoded}
Both denote the same function, namely the function which always delivers the result \ml{3} regardless of the value of its argument, and we might hope that they were interconvertible. This hope is justified, as the following sequence of conversions shows:
\begin{mlalign}
    IF TRUE ((\tlb{p}p) 3) & \conversion{\beta} IF TRUE 3 \\
    & \conversion{\eta} (\tlb{x}IF TRUE 3 x) \\
    & $ \leftrightarrow $ (\tlb{x}3)
\end{mlalign}
The final step is the reduction rule for \ml{IF}.

An alternative method of proving convertibility of expressions denoting functions, which is often more convenient, is to apply both expressions to an arbitrary argument, \ml{w}, say
\begin{mlcoded}\setlength{\tabcolsep}{2pt}
\begin{tabular}{ll}
    IF TRUE ((\tlb{p}p) 3) w\hspace{2cm} & (\tlb{x}3) w \\
    $ \rightarrow $ (\tlb{p}p) 3  & $ \rightarrow $ 3 \\
    $ \rightarrow $ 3 &
\end{tabular}
\end{mlcoded}
Hence,
\begin{mlcoded}
    (IF TRUE ((\tlb{p}p) 3)) $\leftrightarrow$ (\tlb{x}3)
\end{mlcoded}
This proof has the advantage that it only uses reduction, and it avoids the explicit use of \te{}-conversion. If it is not immediately clear why the final step is justified, consider the general case, in which we are given two lambda expressions \ml{F$_1$} and \ml{F$_2$}. If we can show that
\begin{mlalign}
    F$_1$ w & $ \rightarrow $ E
\end{mlalign}
and
\begin{mlalign}
    F$_2$ w & $ \rightarrow $ E
\end{mlalign}
where \ml{w} is a variable which does not occur free in \ml{F$_1$} or \ml{F$_2$}, and \ml{E} is some expression, then we can reason as follows:
\begin{mlalign}
    F$_1$ & \conversion{\eta} (\tlb{w}F$_1$ w) \\
    & $ \leftrightarrow $ (\tlb{w}E) \\
    & $ \leftrightarrow $ (\tlb{w}F$_2$ w) \\
    & \conversion{\eta} F$_2$
\end{mlalign}
and hence \ml{F$_1$ $\leftrightarrow$ F$_2$}.

It is not always the case that lambda expressions which `ought' to mean the same thing are interconvertible, and we will have more to say about this point in Section 2.5.

\subsection{The Name-capture Problem}
As a warning to the unwary we now give an example to show why the lambda calculus is trickier than meets the eye. Fortunately, it turns out that none of our implementations will come across this problem, so this section can safely
be omitted on first reading.

Suppose we define a lambda abstraction TWICE thus:
\begin{mlcoded}
    TWICE $=$ (\tlb{f}\tlb{x}f (f x))
\end{mlcoded}
Now consider reducing the expression \ml{(TWICE TWICE)} using \tb{}-reductions:
\vspace{-\baselineskip}
\begin{mlalign}
    & TWICE TWICE \\
    $=$ & (\tlb{f}\tlb{x}f (f x)) TWICE \\
    $\rightarrow$ & (\tlb{x}TWICE (TWICE x))
\end{mlalign}
Now there are two \tb{}-redexes, \ml{(TWICE x)} and \ml{(TWICE (TWICE x))}, so let us
(arbitrarily) choose the inner one for reduction, first expanding the \ml{TWICE} to
its lambda abstraction:
\begin{mlcoded}
    $=$ \tlb{x}TWICE ((\tlb{f}\underline{\tlb{x}f (f x)}) x)
\end{mlcoded}
Now we see the problem. To apply \ml{TWICE} to \ml{x}, we must make a new instance
of the body of \ml{TWICE} (underlined) replacing occurrences of the formal
parameter, \ml{f}, with the argument, \ml{x}. But \ml{x} is \textit{already used as a formal parameter}
inside the body. It is clearly wrong to reduce to
\begin{mlalign}
    & \tlb{x}TWICE ((\tlb{f}\underline{\tlb{x}f (f x)}) x) \\
    $=$& \tlb{x}TWICE (\tlb{x}x (x x)) \hspace{3cm} \normalfont{\small \textit{wrong!}}
\end{mlalign}
because then the \ml{x} substituted for \ml{f} would be `captured' by the inner \ml{\tl{}\,x}
abstraction. This is called the \textit{name-capture} problem. One solution is to use
\ta{}-conversion to change the name of one of the \ml{\tl{}\,x}'s; for instance:
\begin{mlalign}
    & \tlb{x}TWICE ((\tlb{f}\underline{\tlb{x}f (f x)}) x)\\
    \conversion{\alpha} & \tlb{x}TWICE ((\tlb{f}\underline{\tlb{y}f (f y)}) x) \\
    $\rightarrow$ & \tlb{x}TWICE (\tlb{y}x (x y)) \hspace{3cm} \normalfont{\small \textit{right!}}
\end{mlalign}

We conclude:
\begin{numbered}
    \item \tb{}-reduction is only valid provided the free variables of the argument do
    not clash with any formal parameters in the body of the lambda
    abstraction.
    \item \ta{}-conversion is sometimes necessary to avoid (i).
\end{numbered}

\subsection{Summary of Conversion Rules}

We have now developed three conversion rules which allow us to interconvert
expressions involving lambda abstractions. They are
\begin{numbered}
    \item \textit{Name changing.} \ta{}-conversion allows us to change the name of the formal
    parameter of a lambda abstraction, so long as we do so consistently.
    \item \textit{Function application.} \tb{}-reduction allows us to apply a lambda abstraction to an argument, by making a new instance of the body of the abstraction, substituting the argument for free occurrences of the formal
    parameter. Special care needs to be taken when the argument contains
    free variables.
    \item \textit{Eliminating redundant lambda abstractions.} \te{}-reduction can sometimes
    eliminate a lambda abstraction.
\end{numbered}
Within this framework we may also regard the built-in functions as one more
form of conversion, \td{}-conversion. For this reason the reduction rules for
built-in functions are sometimes called \textit{delta rules}.

As we have seen, the application of the conversion rules is not always
straightforward, so it behooves us to give a formal definition of exactly what the
conversion rules are. This requires us to introduce one new piece of notation.
The notation
\begin{mlcoded}
    E[M/x]
\end{mlcoded}
means the expression \ml{E} with \ml{M} substituted for free occurrences of \ml{x}.

As a mnemonic, imagine `multiplying' \ml{E} by \ml{M/x}, giving \ml{M} where the \ml{x}'s
cancel out, so that \ml{x[M/x] = M}. This notation allows us to express
\tb{}-conversion very simply:
\begin{mlcoded}
    (\tlb{x}E) M \conversion{\beta} E[M/x]
\end{mlcoded}
and it is useful for \ta{}-conversion too.

Figures 2.3 and 2.4 give the formal definitions of substitution and
conversion. They are rather forbidding, but all the complexity arises because
of the name-capture problem described in Section 2.2.6 which will not arise at
all in our implementations. Hence \ta{}-conversion will not be necessary, \tb{}-
reduction can proceed by simple substitution, and \te{}-reduction will prove to
be a compile-time technique only.

To summarize our progress so far, we now have:
\begin{numbered}
    \item a set of formal rules for constructing expressions (Figure 2.1);
    \item a set of formal rules for converting one expression into an equivalent one
    (Figures 2.2-2.4).
\end{numbered}
\boxedfigure{
\begin{tabular}{lll}
    \ml{x [M/x]} &$=$ \ml{M} \\
    \ml{c [M/x]} &where \ml{c} is any variable or constant other than \ml{x}\\
    &$=$ \ml{c}\\
    \ml{(E F)[M/x]} &$=$ \ml{E[M/x] F[M/x]}\\
    \ml{(\tlb{x}E)[M/x]} &$=$ \ml{\tlb{x}E}\\
    \ml{(\tlb{y}E)[M/x]} &where \ml{y} is any variable other than \ml{x}\\
    &$=$ \ml{\tlb{y}E[M/x]}\;\;\; if \ml{x} does not occur free in \ml{E}\\
    &\phantom{$=$ \ml{\tlb{y}E[M/x]}\;\;\; }or \ml{y} does not occur free in \ml{M}\\
    &$=$ \ml{\tlb{z}(E[z/y])[M/x]} otherwise\\
    &\phantom{$=$ }where \ml{z} is a new variable name which does not\\
    &\phantom{$=$ }occur free in \ml{E} or \ml{M}
\end{tabular}
}{Definition of \ml{E[M/x]}}
It turns out that this small formal base is sufficient to build a large and complex theory of interconvertibility; the standard work is Barendregt [1984]. While this book is very well written, it is not intended for the casual reader, and Stoy [1981] gives a less comprehensive but more readable treatment. Curry and Feys also give a clear account of the historical origins and basic properties of the lambda calculus [Curry and Feys, 1958]. The lambda calculus was originally invented by Church [1941].

We will not take the lambda calculus any further as an end in itself; rather we will simply appropriate the fruits of the theory as and when we need them.

\boxedfigure{
\vspace{\baselineskip}
\begin{tabular}{ll}
    \ta{}-conversion:\hspace{1cm} & if \ml{y} is not free in \ml{E} then \\
                      & \ml{(\tlb{x}E) \conversion{\alpha} (\tlb{y}E[y/x])} \\
    \tb{}-conversion: & \ml{(\tlb{x}E) M \conversion{\beta} E[M/x]}\\
    \te{}-conversion: & if \ml{x} is not free in \ml{E}\\
                      & \quad and \ml{E} denotes a function then\\
                      & \ml{(\tlb{x}E x) \conversion{\eta} E}\\
\end{tabular}

\vspace{\baselineskip}
\noindent When used left to right, the \tb{} and \te{} rules are called reductions, and may be written with a `$\rightarrow$' arrow.
\vspace{\baselineskip}
}{Definitions of \ta{}-, \tb{}- and \te{}-conversions}

\section{Reduction Order}

If an expression contains no redexes then evaluation is complete, and the expression is said to be in \textit{normal form}. So the evaluation of an expression consists of successively reducing redexes until the expression is in normal form.

However, an expression may contain more than one redex, so \textit{reduction can proceed by alternative routes}. For example, the expression
\ml{(+ (* 3 4) (* 7 8)) }
can be reduced to normal form with the sequence
\begin{mlalign}
                  & (+ (* 3 4) (* 7 8))\\
    $\rightarrow$ & (+ 12 (* 7 8))\\
    $\rightarrow$ & (+ 12 56)\\
    $\rightarrow$ & 68
\end{mlalign}
or the sequence
\begin{mlalign}
    & (+ (* 3 4) (* 7 8))\\
    $\rightarrow$ & (+ (* 3 4) 56)\\
    $\rightarrow$ & (+ 12 56)\\
    $\rightarrow$ & 68
\end{mlalign}

Not every expression has a normal form; consider for example
\begin{mlcoded}
    (D D)
\end{mlcoded}
where \ml{D} is \ml{(\tlb{x}x x)}. The evaluation of this expression would not terminate since \ml{(D D)} reduces to \ml{(D D)}:
\begin{mlalign}
    (\tlb{x}x x) (\tlb{x}x x) & $\rightarrow$ (\tlb{x}x x) (\tlb{x}x x)\\
    & $\rightarrow$ (\tlb{x}x x) (\tlb{x}x x)
\end{mlalign}
This situation corresponds directly to an imperative program going into an infinite loop.

Furthermore, \textit{some} reduction sequences may reach a normal form while \textit{others do not}. For example, consider
\begin{mlcoded}
    (\tlb{x}3) (D D)
\end{mlcoded}
If we first reduce the application of \ml{(\tlb{x}3)} to \ml{(D D)} (without evaluating \ml{(D D)}) we get the result \ml{3}; but if we first reduce the application of \ml{D} to \ml{D}, we just get \ml{(D D)} again, and if we keep choosing the \ml{(D D)} the evaluation will fail to terminate.

\subsection{Normal Order Reduction}

These complications raise an embarrassing question: can two different reduction sequences lead to different normal forms? Fortunately the answer is `no'. This is a consequence of a profound and powerful pair of theorems, the \textit{Church-Rosser Theorems I and II}, which save the day.

\theorembox{Church-Rosser Theorem I \normalfont{(CRT I)}}{
If \ml{E$_1$ $\leftrightarrow$ E$_2$}, then there exists an expression \ml{E}, such that
\begin{mlcoded}
E$_1$ $\rightarrow$ E \normalfont{and} E$_2$ $\rightarrow$ E
\end{mlcoded}
}

\noindent The following corollary is an easy consequence:
\begin{pcorollary}
No expression can be converted to two distinct normal forms (that is, normal forms that are not \ta{}-convertible).
\end{pcorollary}
\vspace{-23pt}
\begin{pproof}
Suppose that \ml{E $\leftrightarrow$ E$_2$} and \ml{E $\leftrightarrow$ E$_2$}, where \ml{E$_1$} and \ml{E$_2$} are in normal form. Then, \ml{E$_1$  $\leftrightarrow$ E$_2$} and, by CRT I, there must exist an expression \ml{F}, such that \ml{E$_1$ $\rightarrow$ F} and \ml{E$_2$ $\rightarrow$ F}. But \ml{E$_1$} and \ml{E$_2$} have no redexes, so \ml{E$_1$ = F = E$_2$}.
\end{pproof}
\noindent Informally, the corollary says that all reduction sequences which terminate will reach the same result. The second Church-Rosser Theorem concerns a particular reduction order, called \textit{normal order}.

\theorembox{Church-Rosser Theorem II \normalfont{(CRT II)}}{
    If \ml{E$_1$ $\leftrightarrow$ E$_2$}, and \ml{E$_2$} is in normal form, then there exists a \textit{normal order} reduction sequence from \ml{E$_1$} to \ml{E$_2$}.
}

\noindent This is as much as we can hope for; there is at most one possible result, and normal order reduction will find it if it exists. Notice that no reduction sequence can give the `wrong' answer -- the worst that can happen is non-termination.

\plainbox{
Normal order reduction specifies that the \textit{leftmost outermost redex} should be reduced first.
}

\noindent Thus, in our example above \ml{((\tlb{x}3) (D D))}, we would choose the \ml{\tl{}x}-redex first, not the \ml{(D D)}. This rule embodies the intuition that \textit{arguments to functions may be discarded}, so we should apply the function \ml{(\tlb{x}3)} first, rather than first evaluating the argument \ml{(D D)}.

The shortest proofs of the Church-Rosser Theorem I (which is the harder one) are in Welch [1975] and Rosser [1982].

\subsection{Optimal Reduction Orders}

While normal order reduction guarantees to find a normal form (if one exists), it does \textit{not} guarantee to do so in the fewest possible number of reductions. In fact, for tree reduction (see Section 12.1.1) it is provably least favorable, but fortunately for graph reduction (see Section 12.1.1) it seems that normal order is `almost optimal', and that it probably takes more time to find the optimal redex than to pursue normal order. Some work has been done on finding more nearly optimal reduction orders that preserve the desirable properties of normal order [Levy, 1980].

For SK-combinator reduction (see Chapter 16), normal order graph reduction has been shown to be optimal. This result, among many others on graph reduction, is shown in Staples' series of papers [Staples, 1980a, 1980b, 1980c]. A more accessible treatment of this work is given by Kennaway [1984].

\section{Recursive Functions}

We began by saying that we propose to translate all functional programs into the lambda calculus. One pervasive feature of all functional programs is recursion, and this throws the viability of the whole venture into doubt, because the lambda calculus appears to lack anything corresponding to recursion.

In the remainder of this section, therefore, we will show that the lambda calculus is capable of expressing recursive functions without further extension. This is quite a remarkable feat, as the reader may verify by trying it before reading the following sections.

\subsection{Recursive Functions and \ml{Y}}

Consider the following recursive definition of the factorial function:
\begin{mlcoded}
FAC $=$ (\tlb{n}IF ($=$ n 0) 1 ($*$ n (FAC ($-$ n 1))))
\end{mlcoded}
The definition relies on the ability to name a lambda abstraction, and then to refer to this name inside the lambda abstraction itself. No such construct is provided by the lambda calculus. The problem is that lambda abstractions are \textit{anonymous} functions, so they cannot name (and hence refer to) themselves.

We proceed by simplifying the problem to one in which recursion is expressed in its purest form. We begin with a recursive definition:
\begin{mlcoded}
FAC $=$ \tlb{n}(\,$\ldots$\,FAC\,$\ldots$\,)
\end{mlcoded}
(We have written parts of the body of the lambda abstraction as `$\ldots$' to focus attention on the recursive features alone.)

By performing a \tb{}-abstraction on \ml{FAC}, we can transform its definition to:
\begin{mlcoded}
    FAC $=$ (\tlb{fac}(\tlb{n}(\,$\ldots$\,fac\,$\ldots$\,))) FAC
\end{mlcoded}
We may write this definition in the form:
%\begin{equation}
%    \text{\hspace{-0.9em}\ml{FAC $=$ H FAC}}
%\end{equation}
\begin{mlnumbered}
FAC $=$ H FAC
\end{mlnumbered}
where
\begin{mlcoded}
    H $=$ (\tlb{fac}(\tlb{n}(\,$\ldots$\,fac\,$\ldots$\,)))
\end{mlcoded}
The definition of \ml{H} is quite straightforward. It is an ordinary lambda abstraction and does not use recursion. The recursion is expressed solely by definition (2.1).

The definition (2.1) is rather like a mathematical equation. For example, to solve the mathematical equation
\begin{equation*}
    x^{2} - 2 = x
\end{equation*}
we seek values of $x$ which satisfy the equation (namely $x = -1$ and $x = 2$). Similarly, to solve (2.1) we seek a lambda expression for \ml{FAC} which satisfies (2.1). As with mathematical equations, there may be more than one solution.

The equation (2.1)
\begin{mlcoded}
    FAC $=$ H FAC
\end{mlcoded}
states that when the function \ml{H} is applied to \ml{FAC}, the result is \ml{FAC}. We say that \ml{FAC} is a \textit{fixed point} (or \textit{fixpoint}) of \ml{H}. A function may have more than one fixed point. For example, both 0 and 1 are fixed points of the function
\begin{mlcoded}
    \tlb{x}$*$ x x
\end{mlcoded}
which squares its argument.

To summarize our progress, we now seek a fixed point of \ml{H}. It is clear that
this can depend on \ml{H} only, so let us invent (for now) a function \ml{Y} which takes a
function and delivers a fixed point of the function as its result. Thus \ml{Y} has the
behavior that
\begin{mlcoded}
    Y H $=$ H (Y H)
\end{mlcoded}
and as a result \ml{Y} is called a \textit{fixpoint combinator}. Now, if we can produce such a
\ml{Y}, our problems are over. For we can now give a solution to (2.1), namely
\begin{mlcoded}
    FAC $=$ Y H
\end{mlcoded}
which is a non-recursive definition of \ml{FAC}. To convince ourselves that this
definition of \ml{FAC} does what is intended, let us compute \ml{(FAC 1)}. We recall the
definitions for \ml{FAC} and \ml{H}:
\begin{mlalign}
    FAC $=$ &Y H\\
    H   $=$ &\tlb{fac}\tlb{n}IF ($=$ n 0) 1 ($*$ n (fac ($-$ n 1)))
\end{mlalign}
So
\begin{mlalign}
    &FAC 1\\
    $=$ &Y H 1 \\
    $=$ &H (Y H) 1 \\
    $=$ &(\tlb{fac}\tlb{n}IF ($=$ n 0) 1 ($*$ n (fac ($-$ n 1)))) (Y H) 1 \\
    $\rightarrow$ &(\tlb{n}IF ($=$ n 0) 1 ($*$ n (Y H ($-$ n 1)))) 1 \\
    $\rightarrow$ &IF ($=$ 1 0) 1 ($*$ 1 (Y H ($-$ 1 1))) \\
    $\rightarrow$ &$*$ 1 (Y H 0) \\
    $=$ &$*$ 1 (H (Y H) 0) \\
    $=$ &$*$ 1 ((\tlb{fac}\tlb{n}IF ($=$ n 0) 1 ($*$ n (fac ($-$ n 1)))) (Y H) 0) \\
    $\rightarrow$ &$*$ 1 ((\tlb{n}IF ($=$ n 0) 1 ($*$ n (Y H ($-$ n 1)))) 0) \\
    $\rightarrow$ &$*$ 1 (IF ($=$ 0 0) 1 ($*$ 0 (Y H ($-$ 0 1)))) \\
    $\rightarrow$ &$*$ 1 1\\
    $\rightarrow$ &1
\end{mlalign}

\subsection{\ml{Y} Can Be Defined as a Lambda Abstraction}

We have shown how to transform a recursive definition of \ml{FAC} into a non-recursive one, but we have made use of a mysterious new function \ml{Y}. The
property that \ml{Y} must possess is
\begin{mlcoded}
    Y H $=$ H (Y H)
\end{mlcoded}
and this seems to express recursion in its purest form, since we can use it to
express all other recursive functions. Now here comes the magic: \ml{Y} can be
defined as a lambda abstraction, without using recursion!
\begin{mlcoded}
    Y $=$ (\tlb{h}(\tlb{x}h (x x)) (\tlb{x}h (x x)))
\end{mlcoded}

To see that \ml{Y} has the required property, let us evaluate
\begin{mlalign}
    &Y H \\
    $=$ &(\tlb{h}(\tlb{x}h (x x)) (\tlb{x}h (x x))) H \\
    $\leftrightarrow$ &(\tlb{x}H (x x)) (\tlb{x}H (x x)) \\
    $\leftrightarrow$ &H ((\tlb{x}H (x x)) (\tlb{x}H (x x))) \\
    $\leftrightarrow$ &H (Y H)
\end{mlalign}
and we are home and dry.

For those interested in \textit{polymorphic typing} (see Chapter 8), the only respect
in which \ml{Y} might be considered an `improper' lambda abstraction is that the
subexpression \ml{(\tlb{x}h (x x))} does not have a finite type.

The fact that \ml{Y} can be defined as a lambda abstraction is truly remarkable
from a mathematical point of view. From an implementation point of view,
however, it is rather inefficient to implement \ml{Y} using its lambda abstraction,
and most implementations provide \ml{Y} as a built-in function with the reduction
rule
\begin{mlcoded}
    Y H $=$ H (Y H)
\end{mlcoded}

We mentioned above that a function may have more than one fixed point,
so the question arises of which fixed point \ml{Y} produces. It seems to be the `right'
one, in the sense that the reduction sequence of \ml{(FAC 1)} given above does
mirror our intuitive understanding of recursion, but this is hardly satisfactory
from a mathematical point of view. The answer is to be found in \textit{domain
theory}, and the solution produced by \ml{(Y H)} turns out to be the unique \textit{least
fixpoint} of \ml{H} [Stoy, 1981], where `least' is used in a technical domain-theoretic
sense.

\section{The Denotational Semantics of the Lambda Calculus}

There are two ways of looking at a function: as an algorithm which will
produce a value given an argument, or as a set of ordered argument-value
pairs.

The first view is `dynamic' or \textit{operational}, in that it sees a function as a
sequence of operations in time. The second view is `static' or \textit{denotational}: the
function is regarded as a fixed set of associations between arguments and the
corresponding values.

In the previous three sections we have seen how an expression may be
evaluated by the repeated application of reduction rules. These rules
prescribe purely \textit{syntactic} transformations on permitted expressions, without
reference to what the expressions `mean'; and indeed the lambda calculus can
be regarded as a formal system for manipulating syntactic symbols. Nevertheless, the development of the conversion rules was based on our intuitions about abstract functions, and this has, in effect, provided us with an
\textit{operational semantics} for the lambda calculus. But what reason have we to
suppose that the lambda calculus is an accurate expression of the idea of an
abstract function?

To answer this question requires us to give a \textit{denotational semantics} for the
lambda calculus. The framework of denotational semantics will be useful in
the rest of the book, so we offer a brief sketch of it in the remainder of this
section.

\subsection{The \metafn{Eval} Function}

The purpose of the denotational semantics of a language is to assign a \textit{value} to
every \textit{expression} in that language. An expression is a \textit{syntactic} object, formed
according to the syntax rules of the language. A value, by contrast, is an
\textit{abstract} mathematical object, such as `the number 5', or `the function which
squares its argument'.

We can therefore express the semantics of a language as a (mathematical)
function, \metafn{Eval}, from expressions to values:

\begin{center}
        \framebox{\strut\ Expressions\ } $\xrightarrow{\qquad\text{\normalsize \metafn{Eval}{}}\qquad}$ \framebox{\strut\ Values\ }
\end{center}

We can now write equations such as
\begin{mlcoded}
    \metafnbb{Eval}{+ 3 4} = 7
\end{mlcoded}
This says `the meaning (i.e. value) of the expression \ml{(+ 3 4)} is the abstract
numerical value \ml{7}'. We use bold double square brackets to enclose the
argument to \metafn{Eval}{}, to emphasize that it is a syntactic object. This convention is
widely used in denotational semantics. We may regard the expression \ml{(+ 3 4)}
as a \textit{representation} or \textit{denotation} of the value \ml{7} (hence the term \textit{denotational semantics}).

We will now give a very informal development of the \metafn{Eval}{} function for the
lambda calculus. The task is to give a value for \metafnbb{Eval}{E}, for every lambda
expression \ml{E}, and we can proceed by direct reference to the syntax of lambda
expressions (Figure 2.1), which gives the possible forms which \ml{E} might take.

For the moment we will omit the question of constants and built-in
functions, returning to it in Section 2.5.3. Suppose, then, that \ml{E} is a variable,
\ml{x}. What should be the value of
\begin{mlcoded}
    \metafnbb{Eval}{x}
\end{mlcoded}
where \ml{x} is a variable? Unfortunately, the value of a variable is given by its
surrounding context, so we cannot tell its value in isolation. We can solve this
problem by giving \metafn{Eval}{} an extra parameter, \tr{}, which gives this contextual
information. The argument \tr{} is called an environment, and it is a function
which maps variable names on to their values. Thus
\begin{mlcoded}
    \metafnbb{Eval}{x} \tr{} $=$ \tr{} x
\end{mlcoded}


The notation \ml{(\tr{} x)}, on the right-hand side, means 'the function \ml{(\tr{})} applied to
the argument \ml{(x)}'.

Next we treat applications. It seems reasonable that the value of \ml{(E$_1$ E$_2$)}
should be the value of \ml{(E$_1$)} applied to the value of \ml{(E$_2$)}:
\begin{mlcoded}
    \metafnbb{Eval}{E$_1$ E$_2$} \tr{} $=$ (\metafnbb{Eval}{E$_1$} \tr{}) (\metafnbb{Eval}{E$_2$} \tr{})
\end{mlcoded}
The final case is that of a lambda abstraction. What should be the value of
(\metafnbb{Eval}{\tlb{x. E}} \tr{})? It is certainly a function, and so we can fully define it by
giving its value when applied to an arbitrary argument, \ml{(a)}:
\begin{mlcoded}
    (\metafnbb{Eval}{\tlb{x} E} \tr{}) a
\end{mlcoded}
(Following our usual conventions about currying, we will omit the brackets in
future.) The following statement sums up our intuitions about lambda
abstractions:
\begin{quote}
The value of a lambda abstraction, applied to an argument, is the value of
the body of the lambda abstraction, in a context where the formal
parameter is bound to the argument.
\end{quote}
Formally, we write
\begin{mlcoded}
    \metafnbb{Eval}{\tlb{x}E} \tr{} a $=$ \metafnbb{Eval}{E} \tr{}[x$=$a]
\end{mlcoded}
where the notation \ml{(\tr{}[x=a])} means `the function \ml{(\tr{})} extended with the
information that the variable \ml{(x)} is bound to the value \ml{(a)}'. More precisely:
\begin{mlalign}
    \tr{}[x$=$a] x & $=$ a \\
    \tr{}[x$=$a] y & $=$ \tr{} y
\end{mlalign}
if \ml{(y)} is a different variable from \ml{(x)}.

That's it! Apart from constants and built-in functions, each of which require
individual treatment, we have now provided a simple denotational semantics
for the lambda calculus. Figure 2.5 summarizes our progress.

Needless to say, this account is greatly simplified (though hopefully not
misleading). The main component that is missing is a description of the
collection of all possible values which \metafn{Eval}{} can produce. This collection is
called a \textit{domain}, and it is quite a complicated structure, since it includes all the functions and data values that can be denoted by a lambda expression.

\boxedfigure{
    \small
\begin{tabular}{ll}
    \metafnbb{Eval}{k} \tr{} & $=$ <see Section 2.5.3> \\
    \metafnbb{Eval}{x} \tr{} & \ml{$=$ \tr{} x} \\
    \metafnbb{Eval}{E$_1$ E$_2$} \tr{} & \ml{$=$ (\metafnbb{Eval}{E$_1$} \tr{})\quad (\metafnbb{Eval}{E$_2$} \tr{})} \\
    \ml{\metafnbb{Eval}{\tlb{x. E}} \tr{} a} & \ml{$=$ \metafnbb{Eval}{E} \tr{}[x=a]}
\end{tabular}
\vs

\begin{tabular}{rll}
    \hspace{2em}where\hspace{1em} &\ml{k} &is a constant or built-in function\\
          &\ml{x} &is a variable\\
          &\ml{E}, \ml{E$_1$}, \ml{E$_2$} &are expressions
\end{tabular}
}{Denotational semantics of the lambda calculus}

\noindent The
really serious complication is that, in view of the self-application required in
the \tl{} abstraction for \ml{Y}, the domain must include its own function space. Giving a sound theory to such domains is the purpose of \textit{domain theory} [Scott 1981].

We will take the existence and soundness of domain theory and denotational semantics for granted, and the framework they provide will prove to be
quite useful. They are rich and beautiful areas of computer science, and Stoy
[1981] is a good starting-point for further reading.

\textit{A note on notation}: as we have seen, the environment \tr{} is an essential
argument to \metafn{Eval}{}. Nevertheless, in all the situations where we use \metafn{Eval}{} in the
rest of this book, \tr{} plays no significant role. For the sake of simplicity, we will
therefore omit the argument \tr{} from now on -- it could be restored by adding \tr{}
to every call of \metafn{Eval}{}. For example, we will write

\begin{mlalign}
    \metafnbb{Eval}{E$_1$} & $=$ \metafnbb{Eval}{E$_2$}
\end{mlalign}

where we should more correctly write

\begin{mlalign}
    \metafnbb{Eval}{E$_1$} \tr{} & $=$ \metafnbb{Eval}{E$_2$} \tr{}
\end{mlalign}

\subsection{The Symbol $\bot$}

One of the most useful features of the theory we have described in this section
is that it gives us a way to reason about the termination (or otherwise) of
programs.

As remarked in Section 2.3, the reduction of an expression may not reach a
normal form. What value should the semantics assign to such programs? All
that we have to do is to include an element $\bot$, pronounced `bottom', in the
value domain, which is the value assigned to an expression without a normal
form:

\begin{mlcoded}
    \metafnbb{Eval}{{\normalfont <expression with no normal form>}} $= \bot$
\end{mlcoded}

\noindent$\bot$ has a perfectly respectable mathematical meaning in domain theory, and,
like the symbol \ml{0} (which also stands for `nothing'), its use often allows us to
write down succinct equations instead of rambling words. For example,
instead of saying `the evaluation of the expression \ml{E} fails to terminate', we can
write

\begin{mlalign}
    \metafnbb{Eval}{E} & $= \bot$
\end{mlalign}

\subsection{Defining the Semantics of Built-in Functions and Constants}

In this section we will see how to define the value of \metafnbb{Eval}{k}, where \ml{k} is a
constant or built-in function.

For example, what is the value of \metafnbb{Eval}{$*$}? It is certainly a function of two arguments, and we can define it by giving the value of this function
applied to arbitrary arguments:
\begin{mlcoded}
    \metafnbb{Eval}{$*$} a b $=$ a $\times$ b
\end{mlcoded}
This gives the meaning of the lambda calculus $*$ in terms of the mathematical
operation of multiplication $\times$. The distinction between the $*$ and $\times$ is crucial:
the $*$ is a syntactic expression in the lambda calculus, while $\times$ is the abstract
mathematical operation. In the case of multiplication, the mathematical
notation $\times$ differs from the program notation $*$, but in the case of addition (for
example) the symbol $+$ is used by both. This is a ready source of confusion,
and we must keep a clear head!

We will use lower-case letters, such as $a$ and $b$, to stand for values in
semantic equations.

The equation given above is, however, an incomplete specification for $*$
We must define what $*$ does to each possible argument, including $\bot$. The full
set of equations should therefore be:
\begin{mlcoded}
    \metafnbb{Eval}{$*$} a b $=$ a $\times$ b \qquad {\normalfont if} a $\neq \bot$ {\normalfont and} b $\neq \bot$ \\
    \metafnbb{Eval}{$*$} $\bot$ b $= \bot$ \\
    \metafnbb{Eval}{$*$} a $\bot = \bot$
\end{mlcoded}
The two new equations complete the definition of $*$, by specifying that if
either argument of $*$ fails to terminate, then so does the application of $*$.
They are not the only possible set of equations for a multiplication
operator. For example, here are the equations for a more `intelligent'
multiplication operator, \textit{\#}:
\begin{mlcoded}
    \hspace{-2em}
    \begin{tabular}{ll}
        \metafnbb{Eval}{\textit{\#}} a b $=$ a $\times$ b &{\normalfont if} a $\neq \bot$ {\normalfont and} a $\neq$ 0 {\normalfont and} b $\neq \bot$\\
    \metafnbb{Eval}{\textit{\#}} 0 b $=$ 0 &\\
    \metafnbb{Eval}{\textit{\#}} a $\bot = \bot$ &{\normalfont if} a $\neq$ 0 \\
    \metafnbb{Eval}{\textit{\#}} $\bot$ b $= \bot$ &
    \end{tabular}
\end{mlcoded}
These equations imply that \textit{\#} should evaluate its first argument and, if it is
zero, return the result zero without examining the second argument at all;
otherwise it behaves just like $*$. Using \textit{\#} instead of $*$ would cause the
evaluation of some expressions to terminate when they would not have done
so before.

The point of the example is that the semantic equations for a built-in
function enable us to express subtle variations in its behavior, with a precision
that is hard to achieve by giving reduction rules. The semantic equations for a
function both specify the meaning of the function and imply its operational
behavior (reduction rules).

Strictly speaking we should also provide equations such as
\begin{mlalign}
    \metafnbb{Eval}{6} & $=$ 6
\end{mlalign}
where the `\ml{6}' on the left-hand side is a lambda expression, and the `\ml{6}' on the
right-hand side is the abstract mathematical object. Ideally, we should distinguish the two kinds of `\ml{6}' typographically, but common practice is to
write them in the same way and distinguish them only by context. This applies
to all constants and built-in functions. Thus we write
\begin{mlcoded}
    \metafnbb{Eval}{TRUE} $=$ TRUE \\
    \metafnbb{Eval}{IF} $=$ IF \\
    \metafnbb{Eval}{$+$} $= +$
\end{mlcoded}
and so on.

This is sloppy, but it saves clutter. For example, using this more relaxed
notation, we could write the following semantic equations for the built-in
function \ml{IF}:
\begin{mlcoded}
    \setlength{\tabcolsep}{2pt}
    \begin{tabular}{lll}
    &IF TRUE &a b $=$ a \\
    &IF FALSE &a b $=$ b \\
    &IF $\bot$ &a b $=$ $\bot$
    \end{tabular}
\end{mlcoded}
The use of $=$ and the occurrence of $\bot$ continue to remind us that we are looking
at semantic equations rather than reduction rules.

\subsection{Strictness and Laziness}

We say that a function is \textit{strict} if it is sure to need the value of its argument.
This is a concept that will arise repeatedly in the book. Can we give a
denotational definition of strictness?

If a function, \ml{f}, is sure to need the value of its argument, and the evaluation
of the argument will not terminate, then the application of \ml{f} to the argument
will certainly fail to terminate. This verbose, operational argument suggests
the following concise, denotational, definition of strictness:

\definitionbox{\vspace{-1.2\baselineskip}}{
{\centering
A function \ml{f} is \textit{strict} if and only if\\
\ml{f $\bot = \bot$}

}
}

The definition generalizes easily to functions of several arguments. For
example, if \ml{g} is a function of three arguments, then \ml{g} is strict in its second
argument if and only if
\begin{mlcoded}
    g a $\bot$ c $= \bot$
\end{mlcoded}
for all values of \ml{a} and \ml{c}.

If a function is non-strict, we say that it is \textit{lazy}. Technically, this is an abuse
of terminology, since lazy evaluation is an implementation technique which
implements non-strict semantics. However, `lazy' is such an evocative term
that it is often used where `non-strict' would be more correct.

\subsection{The Correctness of the Conversion Rules}

The conversion rules given earlier in this chapter express equivalences
between lambda expressions. It is vital that these equivalences are mirrored in
the denotational world. For example, using \ta-conversion we may write
\begin{mlcoded}
    (\tlb{x}+ x 1) \conversion{\alpha} (\tlb{y}+ y 1)
\end{mlcoded}
Our hope is that both of these expressions mean the same thing or, more
precisely, denote the same function, so that
\begin{mlcoded}
    \metafnbb{Eval}{\tlb{x} + x 1} = \metafnbb{Eval}{\tlb{y}+ y 1}
\end{mlcoded}
In general, we hope that \textit{conversion preserves meaning}, which we may state
as follows:
\begin{mlcoded}
    E$_1$ $\rightarrow$ E$_2$
\end{mlcoded}
implies
\begin{mlcoded}
    \metafnbb{Eval}{E$_1$} $=$ \metafnbb{Eval}{E$_2$}
\end{mlcoded}
In other words, if E$_1$ is convertible to E$_2$ then the meaning of E$_1$ is certainly the
same as the meaning of E$_2$. (As we will see in the next section, however, the
reverse is not always true.) There is a burden of proof here, to show that the
above statement always holds, given the conversion rules and the semantic
function \metafn{Eval}. We will content ourselves with observing that proof is required,
leaving the hard work to [Stoy 1981].
Since the reduction rules (\tb-reduction and \te-reduction) are a subset of the
conversion rules, we certainly know that
\begin{mlcoded}
    E$_1$ \reduction{\beta} E$_2$
\end{mlcoded}
implies
\begin{mlcoded}
    E$_1$ $\rightarrow$ E$_2$
\end{mlcoded}
and hence
\begin{mlcoded}
    E$_1$ \reduction{\beta} E$_2$
\end{mlcoded}
implies
\begin{mlcoded}
    \metafnbb{Eval}{E$_1$} $=$ \metafnbb{Eval}{E$_2$}
\end{mlcoded}

\subsection{Equality and Convertibility}

In the previous section we saw that conversion preserves equality. But is the
reverse true? In particular, does the equality of two expressions imply their
interconvertibility? The answer is `no', as the following example shows.
Consider the two lambda abstractions, which we will call F$_1$ and F$_2$:
\begin{mlalign}
    F$_1$ & $=$ \tlb{x}$+$ x x \\
    F$_2$ & $=$ \tlb{x}$*$ x 2
\end{mlalign}
It is clear that F$_1$ cannot be converted into F$_2$ using the conversion rules of the
lambda calculus. To a mathematician, however, a function is a `black box',
and two functions are the same if (and only if) they give the same result for
each possible argument. This sort of equality of functions is called \textit{extensional
equality}. The function denoted by F$_1$ and that denoted by F$_2$ are certainly
(extensionally) equal, so we may write
\begin{mlcoded}
    \metafnbb{Eval}{F$_1$} $=$ \metafnbb{Eval}{F$_2$}
\end{mlcoded}
So F$_1$ and F$_2$ are not interconvertible, but they do denote the same function.

To summarize the main conclusion:

\hspace{-1.5em}
\begin{tabular}{ll}
if & \ml{E$_1$ \conversion{\alpha} E$_2$} \\
then  \quad\quad & \ml{\metafnbb{Eval}{E$_1$} $=$ \metafnbb{Eval}{E$_2$}}\\
\end{tabular}\\
but not necessarily the other way around.

We can therefore regard conversion as a weak form of reasoning about the
equality of expressions. It can never cause us to believe that two expressions
are equal when they are not, but it may not allow us to prove the equality of
two expressions which are in fact equal. From this point of view, reduction is a
still weaker form of inference.

\section{Summary}
A working understanding of the lambda calculus will prove extremely useful
for the rest of the book, and in this chapter we have tried to give a compact
summary of the material we will require. The treatment has necessarily been
rather superficial, and the reader is again referred to Stoy [1981] or
Barendregt [1984] for fuller treatments.

\section*{References}
\begin{references}
\item Barendregt, H.P. 1984. \textit{The Lambda Calculus -- Its Syntax and Semantics}, 2nd edition.
North-Holland.
\item Church, A. 1941. \textit{The Calculi of Lambda Conversion}. Princeton University Press.
\item Curry, H.B., and Feys, R. 1958. \textit{Combinatory Logic}, Vol. 1. North-Holland.
\item Kennaway, J.R. 1984. \textit{An Outline of Some Results of Staples on Optimal Reduction
Orders in Replacement Systems}. CSA/19/1984. School of Information Systems,
University of East Anglia March.
\item Levy, J.J. 1980. Optimal reductions in the lambda calculus. In \textit{Essays on Combinatory
Logic}, pp. 159-92. Hindley and Seldin (editors). Academic Press.
\item Rosser, J.B. 1982. Highlights of the history of the lambda calculus. \textit{Proceedings of the
ACM Symposium on Lisp and Functional Programming}, Pittsburgh, pp. 216-25.
\item Schonfinkel, M. 1924. Uber die Bausteine der mathematischen Logik. \textit{Mathematische
Annalen.} Vol. 92, pp. 305-16.
\item Scott, D. 1981. \textit{Lectures on a Mathematical Theory of Computation}. PRG-19.
Programming Research Group, Oxford. May.
\item Staples, J. 1980a. Computation on graph-like expressions. \textit{Theoretical Computer Science}. Vol. 10, pp. 171-85.
\item Staples, J. 1980b. Optimal evaluations of graph-like expressions. \textit{Theoretical Computer Science}. Vol. 10, pp. 297-316.
\item Staples, J. 1980c. Speeding up subtree replacement systems. \textit{Theoretical Computer Science}. Vol. 11, pp. 39-47.
\item Stoy, J.E. 1981. \textit{Denotational Semantics}. MIT Press.
\item Welch, P. 1975. \textit{Some Notes on the Martin-L\"of Proof of the Church Rosser Theorem as Rediscovered by Park}. Computer Lab., University of Kent. October.
\end{references}